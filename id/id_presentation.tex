\documentclass{../common/latex_classes/pdf_presentation}

\newcommand{\renderPartialCode}[2]{
	\lstinputlisting[firstnumber=#1,firstline=#1,lastline=#2]{./common/IIOPSSLSocketFactory.java}
}

\title{ID Presentation}

\subject{A brief presentation of the Inspection Document}
% This is only inserted into the PDF information catalog. Can be left
% out. 

% Let's get started
\begin{document}
	\titleToc{}
		\section{Class Name and Pattern Explanation}
		
		\begin{frame}{IIOPSSLSocketFactory - Overview}
			The class we have been assigned is named \textbf{IIOPSSLSocketFactory}.
			It is included in the \textbf{org.glassfish.enterprise.iiop.impl} package.
			\textbf{IIOPSSLSocketFactory} implements the \textbf{ORBSocketFactory} interface, which is part of the \textbf{CORBA} (Common Object Request Broker Architecture) standard declined to \textbf{Java Enterprise Edition}.
		\end{frame}
		
		\begin{frame}{IIOPSSLSocketFactory - ORB}
			The \textbf{ORBSocketFactory} interface is included in the \textbf{com.sun.corba.ee.spi.transport} package, and is an interface that abstracts some parts of the \textbf{ORB} (Object Request Broker) middleware related to sockets creation.\\
			As the name suggests, the \textbf{ORBSocketFactory} interface is a \textbf{Factory} of \textbf{Sockets} for the \textbf{ORB} middleware.
		\end{frame}
		
		\begin{frame}{IIOPSSLSocketFactory - IIOP}
			The \textbf{IIOP} (Internet Inter-ORB Protocol, a concrete protocol) is an implementation of the \textbf{GIOP} (General Inter-ORB Protocol, an abstract protocol) that \textbf{ORB}s use to communicate over the Internet, and provides a mapping between \textbf{GIOP} messages and the \textbf{TCP/IP} layer.\\
			Our class is therefore a \textbf{Factory} of \textbf{SSLSockets} for \textbf{IIOP}, and is an \textbf{Implementation} of the relevant part of the \textbf{IIOP} included in the \textbf{Enterprise} facilities of the \textbf{GlassFish} server.
		\end{frame}
		
		\section{Functional Role of Assigned Class}
		
		\subsection{ORB Middleware}
		
		\begin{frame}{ORB Middleware - Overview}
			The \textbf{O}bject \textbf{R}equest \textbf{B}roker, using the \textbf{IIOP} protocol, allows method calls to be made from one computer to another via Internet. \\
			This \textbf{middleware} allows objects to interact in a \textbf{heterogeneous}, \textbf{distributed}, \textbf{language} and \textbf{platform independent} environment.	
		\end{frame}
			
		\begin{frame}{ORB Middleware - Actors}
			\textbf{Two} main actors are involved in every remote method call, following the so called \textbf{CORBA Client - Server architecture}.
			\begin{itemize}
				\item \textbf{Client:}
					It requests a method call to an object which interface is exposed by the \textbf{Server} and is known to the \textbf{Client}.\\
					The \textbf{Client} has the capability of sending some parameters to the Server for executing the given method call and the capability of receiving back the return value of the called method, if any.
				\item \textbf{Server:}
					It exposes the interfaces of the objects that can be called by the various \textbf{Clients} allowed to make remote method calls.\\
					Through those interfaces, the \textbf{Clients} can make remote method calls, passing objects as parameters if necessary, and receiving a return value, if any.
			\end{itemize}
		\end{frame}
		
		\begin{frame}{CORBA Client - Server Architecture}
			\showPercentImage{./section_2/iiop.jpg}{CORBA Client - Server Architecture Overview}{0.9}
		\end{frame}
		
		\subsection{IIOPSSLSocketFactory Role}
		
		\begin{frame}{Class functionalities - Socket Creation}
			The first main class functionality is the creation of two types of \textbf{Sockets}.
			\begin{itemize}
				\item \textbf{Plain Socket:}
					Plain Text Socket (\textbf{java.net.Socket}) with \textit{Nagle's algorithm disabled}.
				\item \textbf{Secure Socket:}
					Encrypted Socket (\textbf{javax.net.ssl.SSLSocket}) that uses \textbf{Secure Socket Layer} or \textbf{Transport Secure Layer}.\\
					Its characteristics are defined during the creation of a \textbf{IIOPSSLSocketFactory} object and stored into a specific private attribute of type \textbf{IIOPSSLSocketFactory.SSLInfo}.\\
					In particular, a secure socket built by \textbf{IIOPSSLSocketFactory} can use either one of the following cryptographic protocols for data encryption: \textbf{SSL1}, \textbf{SSL2}, \textbf{SSL3}, \textbf{TLS}.
			\end{itemize}
		\end{frame}
		
		\begin{frame}{Class functionalities - Server Socket Creation}
			The second and last main class functionality allows the creation of two types of \textbf{Server Sockets}.
			\begin{itemize}
				\item \textbf{Plain Server Socket:}
					A Server Socket (\textbf{java.net.ServerSocket}) that accepts incoming \textbf{Plain Socket} connections from \textbf{Clients}.
				\item \textbf{Secure Server Socket:}
					A Secure Server Socket (\textbf{javax.net.ssl.SSLSocket}) that accepts incoming \textbf{Secure Socket} connections.
					A \textbf{Server Socket} can be opened for each \textbf{IIOP Listener} registered on the \textbf{Server} at \textbf{IIOPSSLSocketFactory} creation time, because the constructor stores in a dedicated \textbf{java.util.Map} attribute the association between a given \textbf{IIOP Listener TCP Port} and the relevant \textbf{IIOP Listener configuration}, encapsulated into a \textbf{SSLInfo} object.
					When the \textbf{Server Socket} related to a specific \textbf{IIOP Listener} is opened, the \textbf{Server} can accept incoming connections from the remote \textbf{Clients} of \textbf{Enterprise Beans} and from other \textbf{CORBA} based \textbf{Clients}.
			\end{itemize}
		\end{frame}
		
		\begin{frame}{Class Context}
			The entire class behavior depends on the type of process wherein the \textbf{IIOPSSLSocketFactory} object is built.\\
			This \textbf{extremely bad behavior} is obtained through the usage of a \textbf{conditional} structure in the class \textbf{constructor}.\\
			Doing inference on the source code available we discovered that there exists two different types of processes in which this class could be instantiated.
			\begin{itemize}
				\item \textbf{EJB container:}
					A \textbf{Server Process} that is the interface between \textbf{enterprise beans} and the \textbf{Java EE server}.
				\item \textbf{Application Client Container:}
					A \textbf{Client Process} that is the interface between \textbf{Java EE application clients} and the \textbf{Java EE server}.
			\end{itemize}			
		\end{frame}
			
		\begin{frame}{Class Usage}
			The functional role of \textbf{IIOPSSLSocketFactory} is either one of the following two.
			\begin{itemize}
				\item \textbf{EJB Container Functional Role:}
					The \textbf{Server Socket Creation} functionality is used to give a \textbf{IIOP Listener} the capability of accepting incoming \textbf{Plain Text Socket} and \textbf{Secure Socket} connections, in order to receive \textbf{Remote Method Calls} through \textbf{IIOP}.
					On the other side, the \textbf{Socket Creation} capability is exploited when the \textbf{EJB Container} needs to make \textbf{Remote Method Calls} using \textbf{IIOP} to another remote \textbf{EJB Container}.
				\item \textbf{Application Client Container Functional Role:}
					The \textbf{Socket Creation} functionality is used to connect to a remote \textbf{IIOP Listener} that is running into an \textbf{EJB Container} in order to deliver a \textbf{Remote Method Call}, and receive the \textbf{Return Value}, if any.
			\end{itemize}
		\end{frame}
	
	\section{Issues Found}
	
	\subsection{Class Issues}
	
	\begin{frame}{Class Issues - Overview}
		In this subsection are listed the issues related to the whole class and not only to a specific method. \par
		In the next slides we propose only few \textbf{selected}, \textbf{particular} and \textbf{meaningful} issues that we found interesting to mention.
	\end{frame}
	\begin{frame}{Class Issues - Checklist[1] Naming Conventions}
		The class has the capability of creating also \textbf{Plain Text Socket} and \textbf{Plain Text Server Socket}, even if the name \textbf{IIOPSSLSocketFactory} clearly underlines that the class has to be a Factory of \textbf{Secure Sockets and Server Sockets}. \\
		In order to give the architecture the ability of creating \textbf{Plain Text Socket} and \textbf{Plain Text Server Socket}, a separate class should be designed.
	\end{frame}
	\begin{frame}{Class Issues - Checklist[18] Comments}
		The class is not adequately commented at all.
	\end{frame}
	\begin{frame}{Class Issues - Checklist[20] Java Source File}
		The main class contains a non private inner class named \textbf{SSLInfo}.
	\end{frame}
	\begin{frame}{Class Issues - Checklist[27] Class and Interface Declarations}
		The constructor \textbf{IIOPSSLSocketFactory()} is \textbf{75 lines of code long} and it is not easily comprehensible.\\
		Furthermore it is full of if-else structures with no meaningful conditions.\\
		A such long method worsens the readability and the fast comprehension of the method role in the code.\\
		In addition it would be better to \textbf{refactor} the code by moving repeated atomic parts of the constructor logic in additional methods.
	\end{frame}
	\begin{frame}{Class Issues - Checklist[28] Initialization and Declarations}
		\textbf{Lines 94 to 97:} The private variables \textbf{TLS, SSL3, SSL2, SSL} are declared simple strings, instead of being members of a dedicated \textbf{enumeration} that could better encapsulate their logical bond.
		%\renderPartialCode{94}{97}
	\end{frame}
	\subsection{Method Issues}
	
	\begin{frame}{Method Issues - Overview}
		The \textbf{checklist} has also been checked analyzing the assigned source code file method by method.
		We were assigned \textbf{four methods} of this class:
		\configureJava{}
		\renderPartialCode{275}{276}
		\renderPartialCode{310}{310}
		\renderPartialCode{368}{369}
		\renderPartialCode{438}{438}
	\end{frame}
	
	\begin{frame}{createServerSocket() - Main Issues}
		Below we show the method issues we want to underline:
		\begin{itemize}
			\itemBold{Indention CL[8]} The whole method 	lacks a level of indentation and in general is provided of a bad layout (blank useless lines...).
			\itemBold{File Organization CL[12]} Lines 277, 289, 296 and 299 are blank without a clear reason.
			\itemBold{Comments CL[18]} The provided JavaDoc is too short and not really explicative. It does not completely explain the method functionalities.
			\itemBold{Computation, Comparisons and Assignments CL[44]} Lines 283 to 284: The if condition is not explicit and requires inference to be fully understood. In addition it is error prone. The whole problem should have been faced using an enumeration instead of String constants.
			\itemBold{Initialization and Declarations} In the method code we can find some useless variable initialization to \textit{"null"} value.
		\end{itemize}
	\end{frame}
	
	\begin{frame}{createSocket() - Main Issues}
		Below we show the method issues we want to underline:
		\begin{itemize}
			\itemBold{Indention CL[8]} The whole method 	lacks a level of indentation and in general is provided of a bad layout (blank useless lines...).
			\itemBold{File Organization CL[13]} Lines 304, 310, 317, 329 are too long.
			\itemBold{Comments CL[18]} The provided JavaDoc is too short and not really explicative. It does not completely explain the method functionalities.
			\itemBold{Initialization and Declaration CL[28]} At line 401 the variable \textit{ss} should have been declared of type \textbf{SSLServerSocket} instead of plain \textbf{ServerSocket}.
			\itemBold{Initialization and Declaration CL[29]} At lines 332, 332 the local variables host and port could have been used instead for better code readability.
			\itemBold{Exception CL[52]} Lines 340 to 345: The catch block is actually catching a generic Exception instead of the generated ones.
		\end{itemize}
	\end{frame}
	
	\begin{frame}{createSSLServerSocket() - Main Issues}
		Below we show the method issues we want to underline:
		\begin{itemize}
			\itemBold{Naming Conventions CL[1]} The difference between \textbf{port} and \textbf{iport} should be more highlighted through the naming choices (\textbf{iport} is used to search inside the \textbf{HashMap}).%367.377
			\itemBold{Indention CL[8]} The whole method lacks a level of indentation and in general is provided of a bad layout (blank useless lines...).
			\itemBold{File Organization CL[13]} Lines 368, 372, 380, 411 are too long.
			\itemBold{Comments CL[18]} The provided JavaDoc is too short and not really explicative. It does not completely explain the method functionalities.
			\itemBold{Initialization and Declaration CL[33]} Lines 393 and 401: The \textit{cs} and \textit{ss} variables are declared at the middle of the method code. They have to be declared at the beginning of it.
			\itemBold{Exception CL[52]} Lines 422 to 425: The catch block is actually catching a generic Exception instead of the generated ones.
			\itemBold{Output Format CL[42]} In the catch blocks the caught exceptions are not explained to the user, they are only printed out.
		\end{itemize}
	\end{frame}
	
	\begin{frame}{createSSLSocket() - Main Issues}
		Below we show the method issues we want to underline:
		\begin{itemize}
			\itemBold{Naming Conventions CL[1]} The following piece of code contains a variable named e2 whose name is not meaningful.%468.469
			\itemBold{Indention CL[8]} The whole method 	lacks a level of indentation and in general is provided of a bad layout (blank useless lines...).
			\itemBold{File Organization CL[13]} Lines 90, 99 are too long.
			\itemBold{Comments CL[18]} Comments and JavaDoc provided in this method are completely useless.
			\itemBold{Exception CL[52]} Lines 462 to 475: The catch block is actually catching a generic Exception instead of the generated ones.
			\itemBold{Initialization and Declarations CL[33]} Lines 450, 451, 452, 468: The local variables are not initialized at the beginning of their relevant blocks.
		\end{itemize}
	\end{frame}
	
	\section{Other Problems}
	
	\begin{frame}{Other Problems - Nagle's Algorithm Disabling}
		By default, \textbf{Nagle's algorithm} is disabled for all and only the plain sockets built. For encrypted sockets, Nagle's algorithm is not disabled and this could lead to severe performance issue. \par This is because \textbf{Nagle's algorithm is essentially delaying the delivery of TCP Packets in order to avoid the delivery of several small packets (which obviously increases the overhead to data ratio), preferring less but bigger packets}. Thus, a server response that is generated very fast could be delivered later to the client because of this policy. \par \textbf{For us, the missed disabling of Nagle's algorithm in secure sockets is a major bug.}
	\end{frame}

	\begin{frame}{Other Problems - Obtain Data from Global Variables}
		The characteristics of \textbf{Secure Sockets} and \textbf{Secure Server Sockets} creation are fetched from \textbf{global variables} using \textbf{public static methods} and stored into two relevant attributes by \textbf{IIOPSSLSocketFactory} constructor at object creation time.\\
		Obtaining \textbf{global data} from \textbf{static methods} instead of using a proper \textbf{object oriented factory}, makes the class functionality mostly dependent on the surrounding context, less predictable and bug prone.
	\end{frame}
	
	\begin{frame}{Other Problems - Class Context Dependency}
		The entire \textbf{class behavior} depends on the \textbf{type of process} in which context the IIOPSSLSocketFactory \textbf{object is built}. \par In fact, the Secure Socket Creation is both used in the \textbf{Application Client Container} (as Client of a EJB Container) and the \textbf{EJB Container} (as Client of another EJB Container), but the Secure Server Socket Creation is used and can only be used (if contrary a \textit{RuntimeException} is thrown) by the \textbf{EJB Container} (as Server of Application Client Containers and EJB Containers). \par \textbf{This is a major mess in the software architecture, the solution should have been designed in a completely different way, following the OOP principles.}
	\end{frame}
	
	\begin{frame}{Other Problems - No Generics Used}
		There is no usage of generics in the \textbf{java.util.Map} attribute \textbf{portToSSLInfo}.%113
	\end{frame}
\end{document}