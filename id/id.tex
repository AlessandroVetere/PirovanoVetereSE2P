\documentclass{../common/latex_classes/pdf_document}

\titleSubtitle{Inspection Document}{Version 1.0}

%TODO Sonarqube?

%TODO Video https://www.youtube.com/watch?v=1m3eRFeCInY

%TODO http://research.microsoft.com/pubs/180283/ICSE%202013-codereview.pdf

\begin{document}
	\titleToc{}
	\newpage
	Software	Engineering	2 Project,	AA	2015-2016
	Assignment 3:	Code	Inspection
	Code	inspection	(e.g.,	code	analysis,	visual	inspection,	reverse	engineering,	etc.)
	is	 systematic	 examination	 (often	 known	 as	 peer	 review)	 of	 computer	 source	
	code.	It	is	intended	to	find	mistakes	overlooked	in	the	initial	development	phase,	
	improving	 both	 the	 overall	 quality	 of	 software	 and	 the	 developers'	 skills.	
	Reviews	 are	 done	 in	 various	 forms	 such	 as	 pair	 programming,	 informal	
	walkthroughs,	 and	 formal	 inspections.	 You	 are to	 apply	 Code Inspection	
	techniques	(supported	by	 the	review	checklist at	 the	end	of	 this	document)	for	
	the	 purpose	 of	 evaluating	 the	 general	 quality	 of	 selected	 code	 extracts	 from	 a	
	release	of	the	Glassfish	4.1	application	server (see	the	information	at	the	end	of	
	the	document).	
	In	 the	 scope	 of	 this	assignment,	 you	will	 be	given	a	 selected	number	 of	classes	
	extracted	 from	 said	 software	 release.	Said	 selection	is	done	 systematically,	and	
	will	assign	different	sets	of	classes	to	different	groups.	The	systematic	selection	
	makes	sure	that	the	difficulty	of	the	assignment	is	homogeneous	per	every	group
	and	 takes	into	account	 the	 fact	 that	we	have	groups	of	different	sizes. We	have	
	allocated	about	80 lines	of	code	per	person1.	
	\newpage
	\configureJava{}
	\lstinputlisting{./common/IIOPSSLSocketFactory.java}
	\newcommand{\renderPartialCode}[2]{
		\lstinputlisting[firstline=#1,lastline=#2]{./common/IIOPSSLSocketFactory.java}
	}
	\newpage	
	\inputSection{1}
	\inputSection{2}
	\inputSection{3}
	\inputSection{4}
\end{document}               % End of document.

