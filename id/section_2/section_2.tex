\section{Functional Role of Assigned Classes}
%TODO <Elaborate on the functional role you have identified for the class cluster that was assigned to you, also, elaborate on how you managed to understand this role and provide the necessary evidence, e.g., javadoc, diagrams, etc.>

Starting from the considerations made in the previous section, the \textbf{IIOPSSLSocketFactory} functional role is further analysed .

\subsection{ORB Middleware Actors Overview}
First of all, an overview of the \textbf{ORB} middleware is given, because it is the component that uses the \textbf{IIOP} protocol to communicate over the internet.
The \textbf{O}bject \textbf{R}equest \textbf{B}roker allows method calls to be made from one computer to another via network, and it provides that for each remote method call there are two main actors exchanging informations:
\begin{itemize}
	\item \textbf{Client:}
		It requests a method call to an object which interface is exposed by the \textbf{Server} and is known to the \textbf{Client}.
		The \textbf{Client} has the capability of sending some parameters to the Server for executing the given method call and the capability of receiving back the return value of the called method, if any.
	\item \textbf{Server:}
		It exposes the interfaces of the objects that can be called by the various \textbf{Clients} allowed to make remote method calls.
		Through those interfaces, the \textbf{Clients} can make remote method calls, passing objects as parameters if necessary, and receiving a return value, if any.
\end{itemize}

\subsection{IIOPSSLSocketFactory functionalities}
In order to give each actor the capability of performing the actions provided by the middleware using the \textbf{IIOP} protocol, the \textbf{IIOPSSLSocketFactory} main functionalities are the following ones.
\begin{itemize}
	\item \textbf{Socket Creation:}
		This functionality allows the creation of a \textbf{Socket} with some specific characteristics.
		\begin{itemize}
			\item \textbf{Plain Socket:}
				Plain Text Socket (\textbf{java.net.Socket}) with \textit{Nagle's algorithm disabled}.
				It is created using \textbf{java.nio.channels.SocketChannel}, if so is specified in the \textbf{ORB} object, whose reference is passed at runtime to a \textbf{IIOPSSLSocketFactory} object using a setter).
			\item \textbf{Secure Socket:}
				Encrypted Socket (\textbf{javax.net.ssl.SSLSocket}) that uses \textbf{Secure Socket Layer} or \textbf{Transport Secure Layer}.
				Its characteristics are defined during the creation of a \textbf{IIOPSSLSocketFactory} object by \textit{obtaining data from global variables} (which seems to be a bad behaviour) and storing those data into a specific private attribute of type \textbf{IIOPSSLSocketFactory.SSLInfo}.
				This private attribute is never modified after \textbf{IIOPSSLSocketFactory} object creation and thus it can be considered as final, although it is not final nor immutable.
				In particular, a secure socket built by \textbf{IIOPSSLSocketFactory} can use either one of the following cryptographic protocols for data encryption.
				\begin{itemize}
					\item \textbf{SSL1} 
					\item \textbf{SSL2}
					\item \textbf{SSL3}
					\item \textbf{TLS}
				\end{itemize}
		\end{itemize}
	\item \textbf{Server Socket Creation:}
		This functionality allows the creation of a \textbf{Server Socket}.
		\begin{itemize}
			\item \textbf{Plain Server Socket:}
				A Server Socket (\textbf{java.net.ServerSocket}) that accepts incoming \textbf{Plain Socket} connections from \textbf{Clients}. If the \textbf{ORB} object set into the given \textbf{IIOPSSLSocketFactory} object is configured accordingly, the \textbf{Server Socket} is created using \textbf{java.nio.ServerSocketChannel}.
			\item \textbf{Secure Server Socket:}
				A Secure Server Socket (\textbf{javax.net.ssl.SSLSocket}) that accepts incoming \textbf{Secure Socket} connections on a certain \textbf{Port} of a given \textbf{IP} address.
				The \textbf{SSLInfo} object necessary to have the informations about how to build the \textbf{Secure Server Socket} are contained into an \textbf{IIOPSSLSocketFactory} attribute of type \textbf{java.util.Map} that associates a given \textbf{TCP Port} to the relevant \textbf{SSLInfo} object.
				This \textbf{java.util.Map} is \textit{initialized from global variables} (which seems a bad habit again) at \textbf{IIOPSSLSocketFactory} object creation time and stores the association of every \textbf{IIOP Listener Port} to the relevant \textbf{IIOP Listener} configuration.
				An \textbf{IIOP listener}, using \textbf{Server Sockets}, accepts incoming connections from the remote \textbf{Clients} of \textbf{Enterprise Beans} and from other \textbf{CORBA} (Common Object Request Broker Architecture) based \textbf{Clients}.
		\end{itemize}
\end{itemize}
%TODO Spiegare quando vengono usate le diverse funzioni dal client e dal server

%TODO I SERVER POSSONO PARLARE FRA DI LORO MA I SERVER NON POSSONO APRIRE CONNESSIONI VERSO I CLIENT!!!! LA MAPPA PORT TO SSLINFO E' SALVATA SOLO NEL SERVER, QUINDI UN CLIENT NON PUOI USARE IL METODO createSSLServerSocket (CHE SFRUTTA LE INFORMAZIONI CONTENUTE IN QUESTA MAPPA) E QUINDI NON PUO' ACCETTARE CONNESSIONI!!! LA CLASSE E' USATA NEL CONTESTO DI UN PROCESSO SERVER COME PUNTO DI COMUNICAZIONE BIDIREZIONALE , MENTRE IN UN PROCESSO CLIENT E' USATA SOLO PER APRIRE SOCKET VERSO UN SERVER!

%TODO Il client è in realtà l'application server. Quindi è il server che CHIAMA metodi sul server che contiene l'EJB a cui è associato lo IIOPListener.

The entire class behaviour depends on the type of process in which context the \textbf{IIOPSSLSocketFactory} object is built (A extremely bad behaviour).
There can exists two different types of 
So, using inference (the class documentation is not enough) on what has been discussed about so far, it can be concluded that the class functional role is 
%TODO Completare

\subsection{References}
%TODO Wikipedia, OMG, glassfish, java...
To study the functional role of \textbf{IIOPSSLSocketFactory} and further topics, some references have been consulted:
\begin{itemize}
	\item Wikipedia, the free encyclopedia (https://www.wikipedia.org/)
	\item Expectations, Outcomes, and Challenges Of Modern Code Review by Alberto Bacchelli and Christian Bird (http://research.microsoft.com/pubs/180283/ICSE\%202013-codereview.pdf)
	\item Object Management Group (OMG) website (http://www.omg.org/)
	\item Java Platform, Standard Edition 7 API Specification (http://docs.oracle.com/javase/7/docs/api/overview-summary.html)
	\item Java RMI over IIOP (http://docs.oracle.com/javase/7/docs/technotes/guides/rmi-iiop/index.html)
	\item GrepCode for navigating the source code (http://grepcode.com/file/repo1.maven.org/maven2/org.glassfish.main.orb/orb-iiop/4.0/org/glassfish/enterprise/iiop/impl/IIOPSSLSocketFactory.java)
	\item Class JavaDoc
	\item GlassFish Server Administration Guide: Administering the Object Request Broker (ORB) (http://docs.oracle.com/cd/E26576_01/doc.312/e24928/orb.htm#GSADG00018)
	\item https://httpd.apache.org/docs/trunk/ssl/ssl_howto.html
	\item SSL/TLS Strong Encryption: An Introduction (https://httpd.apache.org/docs/trunk/ssl/ssl_intro.html)
\end{itemize}



