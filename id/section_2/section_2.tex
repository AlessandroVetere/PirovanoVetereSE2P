\section{Functional Role of Assigned Classes}
%TODO <Elaborate on the functional role you have identified for the class cluster that was assigned to you, also, elaborate on how you managed to understand this role and provide the necessary evidence, e.g., javadoc, diagrams, etc.>

The class we have been assigned is named IIOPSSLSocketFactory.
It is included in the org.glassfish.enterprise.iiop.impl package.
IIOPSSLSocketFactory implements the ORBSocketFactory interface, which is part of the CORBA (Common Object Request Broker Architecture) standard declined to Java Enterprise Edition.
%included in the com.sun.corba.ee.spi.transport package.
ORBSocketFactory is an interface that abstracts some parts of the ORB middleware related to sockets creation.
The IIOP (Internet Inter-Orb Protocol, a concrete protocol) is an implementation of the GIOP (General Inter-ORB Protocol, an abstract protocol) that ORBs (Object Request Brokers) use to communicate over the Internet, and provides a mapping between GIOP messages and the TCP/IP layer.

\subsection{ORB Middleware}
An ORB Middleware provides that there are two main actors exchanging informations:
\begin{itemize}
	\item \textbf{Client:}
		It requests a method call to an object which interface is exposed by the Server and is known by the Client.
		The Client has the capability of sending some parameters to the Server for executing the given method call and needs to have back the return value of the method, if any.
	\item \textbf{Server:}
		It exposes the interfaces of the objects that can be called by the various Clients allowed to make remote method calls.
		Through those interfaces, the Clients can make remote method calls, passing objects as parameters if necessary, and receive a return value, if any.
\end{itemize}

\subsection{IIOPSSLSocketFactory functionalities}
In order to give each actor the capability of performing the actions provided by the middleware using the IIOP protocol, the IIOPSSLSocketFactory main functionalities are:
\begin{itemize}
	\item \textbf{Socket Creation:}
		This functionality allows the creation of a socket with some specific characteristics.
		\begin{itemize}
			\item \textbf{Plain Socket:}
				Plain text socket (java.net.Socket) with the option of using a java.nio.channels.SocketChannel (if so is specified in the ORB object, whose reference is passed at runtime to a IIOPSSLSocketFactory object using a setter).
			\item \textbf{Secure Socket:}
				Encrypted socket (javax.net.ssl.SSLSocket) that uses Secure Socket Layer or Transport Secure Layer.
				Its characteristics are defined during the creation of a IIOPSSLSocketFactory object by obtaining data from global variables (which seems to be a bad behaviour) and storing those data into a specific private attribute of type IIOPSSLSocketFactory.SSLInfo.
				This private attribute is never modified after IIOPSSLSocketFactory object creation and thus it can be considered as final, although it is not final nor immutable.
				In particular, a secure socket built by IIOPSSLSocketFactory can use either one of the following cryptographic protocols for data encryption.
				\begin{itemize}
					\item \textbf{SSL1} 
					\item \textbf{SSL2}
					\item \textbf{SSL3}
					\item \textbf{TLS}
				\end{itemize}
		\end{itemize}
		By default, Nagle's algorithm is disabled for \textbf{all and only} the plain sockets built: this is done by setting the "TCP No Delay" property of the java.net.Socket to true, using the given setter.
		For encrypted sockets, Nagle's algorithm is not disabled and this could lead to severe performance issue.
		This is because Nagle's algorithm is essentially delaying the delivery of TCP Packets in order to avoid the delivery of several small packets (which obviously increases the overhead to data ratio), preferring less but bigger packets.
		Thus, a server response that is generated very fast could be delivered later to the client because of this policy.
		\textbf{For us, the missed disabling of Nagle's algorithm in secure sockets is a major bug.}
	\item \textbf{Server Socket Creation:}
		This functionality allows the creation of a server socket.
		\begin{itemize}
			\item \textbf{Plain Server Socket:}
				A java.net.ServerSocket that accepts incoming \textbf{Plain Socket} connections from clients. If the ORB object set into the given IIOPSSLSocketFactory object is configured accordingly, the server socket is created using java.nio.channels.ServerSocketChannel.
			\item \textbf{Secure Server Socket:}
				A javax.net.ssl.SSLSocket that accepts incoming \textbf{Secure Socket} connections on a certain port of a given IP address.
				The SSLInfo object necessary to have the informations about how to build the secure server socket are contained into an IIOPSSLSocketFactory attribute of type java.util.Map that associates a given TCP port to the relevant SSLInfo object.
				This java.util.Map is initialized from global variables (which seems again a bad habit) at IIOPSSLSocketFactory object creation time and stores the association of every IIOP Listener port to the relevant IIOP Listener configuration.
				An IIOP listener is a server socket that accepts incoming connections from the remote clients of enterprise beans and from other CORBA (Common Object Request Broker Architecture) based clients.
		\end{itemize}
\end{itemize}
%TODO Spiegare quando vengono usate le diverse funzioni dal client e dal server

%TODO I SERVER POSSONO PARLARE FRA DI LORO MA I SERVER NON POSSONO APRIRE CONNESSIONI VERSO I CLIENT!!!! LA MAPPA PORT TO SSLINFO E' SALVATA SOLO NEL SERVER, QUINDI UN CLIENT NON PUOI USARE IL METODO createSSLServerSocket (CHE SFRUTTA LE INFORMAZIONI CONTENUTE IN QUESTA MAPPA) E QUINDI NON PUO' ACCETTARE CONNESSIONI!!! LA CLASSE E' USATA NEL CONTESTO DI UN PROCESSO SERVER COME PUNTO DI COMUNICAZIONE BIDIREZIONALE , MENTRE IN UN PROCESSO CLIENT E' USATA SOLO PER APRIRE SOCKET VERSO UN SERVER!

%TODO Il client è in realtà l'application server. Quindi è il server che CHIAMA metodi sul server che contiene l'EJB a cui è associato lo IIOPListener.

The entire class behaviour depends on the type of process in which context the IIOPSSLSocketFactory object is built.
%TODO Completare











