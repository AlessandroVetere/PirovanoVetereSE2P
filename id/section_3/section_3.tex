\section{Issues Found}
%TODO <Report the classes/code fragments that do not fullfill some points in the check list. Explain which point is not fullfilled and why>

Code	inspection	checklist
Naming	Conventions
1. All	class	names,	interface	names,	method	names,	class	variables,	method	
variables,	and	constants	used	should	have	meaningful	names	and	do	what	
the	name	suggests.
2. If	 one-character	 variables	 are	 used,	 they	 are	 used	 only	 for	 temporary	
“throwaway”	variables,	such	as	those	used	in	for	loops.
3. Class	names	are	nouns,	in	mixed	case,	with	the	first	letter	of	each	word	in	
capitalized.	Examples:		class	Raster;		class	ImageSprite;	
4. Interface	names	should	be	capitalized	like	classes.
5. Method	names	should	be	verbs,	with	the	first	letter	of	each	addition	word	
capitalized.	Examples:		getBackground();		computeTemperature().
6. Class	 variables,	 also	 called	 attributes,	 are	 mixed	 case,	 but	 might	 begin	
with	 an	 underscore	 (‘\_’)	 followed	 by	 a	 lowercase	 first letter.	 	 All	 the	
remaining	words	in	 the	 variable	 name have	 their	 first	letter capitalized.		
Examples:			\_windowHeight,		timeSeriesData.
7. Constants	 are	 declared	 using	 all	 uppercase	 with	 words	 separated	 by	 an	
underscore.		Examples:			MIN\_WIDTH;		MAX\_HEIGHT;
Indention
8. Three	or	four	spaces	are	used	for	indentation	and	done	so	consistently
9. No	tabs	are	used	to	indent
Braces
10. Consistent	bracing	style	is	used,	either	the	preferred	“Allman”	style	(first	
brace	goes	underneath	the	opening	block)	or	the	“Kernighan	and	Ritchie”	
style (first	brace	is	on	the	same	line	of	the	instruction	that	opens	the	new	
block).
11. All	 if,	 while,	 do-while,	 try-catch,	 and	 for	 statements	 that	 have	 only	 one	
statement	to	execute	are	surrounded	by	curly	braces.		Example:
Avoid	this:
if (	condition	)
doThis();
Instead	do	this:
if	(	condition	)	
{
	doThis();
}
File	Organization
12. Blank	 lines	 and	 optional	 comments	 are	 used	 to	 separate	 sections	
(beginning	 comments,	 package/import	 statements,	 class/interface	
declarations	 which	 include	 class	 variable/attributes	 declarations,	
constructors,	and	methods).
13. Where	practical,	line	length	does	not	exceed	80	characters.
14. When	 line	 length	 must	 exceed	 80	 characters,	 it	 does	 NOT	 exceed	 120	
characters.	
Wrapping	Lines
15. Line	break	occurs	after	a	comma	or	an	operator.
16. Higher-level	breaks	are	used.
17. A	 new	 statement	 is	 aligned	 with	 the	 beginning	 of	 the	 expression	 at	 the	
same	level	as	the	previous	line.
Comments
18. Comments	 are	 used	 to	 adequately	 explain what	 the	 class,	 interface,	
methods,	and	blocks	of	code	are	doing.
19. Commented	 out	 code	 contains	a	 reason	 for	 being	 commented	 out	and	a	
date	it	can	be	removed	 from	 the	source	 file	if	determined	it	is	no	longer	
needed.
Java	Source	Files
20. Each	Java	source	file	contains	a	single	public	class	or	interface.
21. The	public	class	is	the	first	class	or	interface	in	the	file.
22. Check	that the	external	program	interfaces are	implemented	consistently	
with	what	is	described	in	the	javadoc.
23. Check	that the	javadoc is complete (i.e.,	it covers all	classes	and	files	part	
of	the	set	of	classes	assigned	to	you).
Package	and	Import	Statements
24. If	 any	 package	 statements	 are	 needed,	 they	 should	 be	 the	 first	 noncomment
statements.		Import	statements	follow.
Class	and	Interface	Declarations
25. The	class	or	interface	declarations	shall	be	in	the	following	order:
A. class/interface	documentation	comment
B. class	or	interface	statement
C. class/interface	implementation	comment,	if	necessary
D. class	(static)	variables
a. first	public class	variables
b. next	protected class	variables
c. next	package	level	(no	access	modifier)
d. last	private class	variables
E. instance	variables
a. first	public	instance	variables
e. next	protected	instance	variables
f. next	package	level	(no	access	modifier)
g. last	private	instance	variables
F. constructors
G. methods	
26. Methods	 are	 grouped	 by	 functionality	 rather	 than	 by	 scope	 or	
accessibility.
27. Check	 that the	 code is free	 of	 duplicates,	 long	 methods,	 big	 classes,	
breaking	encapsulation,	as	well	as	if	coupling	and	cohesion	are	adequate.
Initialization	and	Declarations
28. Check	that variables	and	class	members	are	of	the	correct	type.	Check	that	
they	have	the	right	visibility	(public/private/protected)
29. Check	that variables	are	declared	in	the	proper	scope	
30. Check	that constructors	are called	when	a	new	object	is	desired
31. Check	that all	object	references are initialized	before	use	
32. Variables	are	initialized	where	they	are	declared,	unless	dependent	upon	
a	computation
33. Declarations	 appear	 at	 the	 beginning	 of	 blocks	 (A	 block	 is	 any	 code	
surrounded	by	curly	braces	“{“	and	“}”	).		The	exception	is	a	variable	can	
be	declared	in	a	‘for’	loop.
Method	Calls
34. Check	that parameters are	presented	in	the	correct	order
35. Check	 that the	correct	method is being	called,	or	should	it	be	a	different	
method	with	a	similar	name
36. Check	that method	returned values	are	used	properly	
Arrays
37. Check	 that there	 are	 no	 off-by-one	 errors	 in	 array	 indexing (that	 is,	 all	
required	array	elements are correctly	accessed	through	the	index)
38. Check	 that all	 array	 (or	 other	 collection)	 indexes	 have	 been	 prevented	
from	going	out-of-bounds
39. Check	that constructors	are called	when	a	new	array	item	is	desired
Object	Comparison
40. Check	that all	objects	(including	Strings)	are	compared	with	"equals"	and	
not with "=="
Output	Format
41. Check	that displayed	output	is	free	of	spelling	and	grammatical	errors
42. Check	that error	messages	are	comprehensive	and	provide	guidance	as to	
how	to	correct	the	problem
43. Check	that the	output is formatted	correctly	in	terms	of	line	stepping	and	
spacing
Computation,	Comparisons	and	Assignments
44. Check	 that the	 implementation	 avoids “brutish	 programming:	 (see	
http://users.csc.calpoly.edu/~jdalbey/SWE/CodeSmells/bonehead.html)
45. Check	 order	 of	 computation/evaluation,	 operator	 precedence	 and	
parenthesizing	
46. Check	the	liberal	use	of	parenthesis	is	used	to	avoid	operator	precedence	
problems.
47. Check	that all	denominators	of	a	division	are	prevented	from	being	zero
48. Check	 that integer	arithmetic,	especially	division,	are	used	appropriately	
to	avoid	causing unexpected	truncation/rounding
49. Check	that the	comparison and	Boolean operators	are	correct
50. Check	 throw-catch	 expressions,	 and	 check	 that the	 error	 condition	 is	
actually	legitimate	
51. Check	that the	code is free	of any	implicit type	conversions
Exceptions
52. Check	that the	relevant	exceptions	are	caught
53. Check	that the	appropriate	action are taken	for	each	catch	block	
Flow	of	Control
54. In	 a	 switch	 statement,	 check	 that all	 cases	 are	 addressed	 by	 break	 or	
return
55. Check	that all	switch statements	have	a	default	branch
56. Check	 that all	 loops are correctly	 formed,	 with	 the	 appropriate	
initialization,	increment	and	termination	expressions
Files
57. Check	that all	files	are	properly	declared	and	opened
58. Check	that all	files	are	closed	properly,	even	in	the	case	of	an	error
59. Check	that EOF	conditions	are	detected	and	handled	correctly
60. Check	that	all	file	exceptions	are	caught	and	dealt	with	accordingly