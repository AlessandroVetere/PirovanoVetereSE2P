\section{Issues Found}
%TODO <Report the classes/code fragments that do not fullfill some points in the check list. Explain which point is not fullfilled and why>
In this section are reported all the coding choices that do not meet the \textbf{Code Inspection Checklist} given.

\subsection{Code Inspection Checklist}
%\newcounter{checklist_item_number}
%\setcounter{checklist_item_number}{1}
%\stepcounter{checklist_item_number}
\subsubsection{Naming Conventions}
\begin{itemize}
	\checklist{1}
		All class names, interface names, method names, class variables, method variables, and constants used should have meaningful names and do what the name suggests.
	\checklist{2}
		If one-character variables are used, they are used only for temporary “throwaway” variables, such as those used in for loops.
	\checklist{3}
		Class names are nouns, in mixed case, with the first letter of each word in capitalized. Examples: class Raster; class ImageSprite;
	\checklist{4}
		Interface names should be capitalized like classes.
	\checklist{5}
		Method names should be verbs, with the first letter of each addition word capitalized. Examples: getBackground(); computeTemperature().
	\checklist{6}
		Class variables, also called attributes, are mixed case, but might begin with an underscore (‘\_’) followed by a lowercase first letter. All the remaining words in the variable name have their first letter capitalized. Examples: \_windowHeight, timeSeriesData.
	\checklist{7}
		Constants are declared using all uppercase with words separated by an underscore. Examples: MIN\_WIDTH; MAX\_HEIGHT;
\end{itemize}

\subsubsection{Indention}
\begin{itemize}
	\checklist{8}
		Three or four spaces are used for indentation and done so consistently
	\checklist{9}
		No tabs are used to indent
\end{itemize}

\subsubsection{Braces}
\begin{itemize}
	\checklist{10}
		Consistent bracing style is used, either the preferred “Allman” style (first brace goes underneath the opening block) or the “Kernighan and Ritchie” style (first brace is on the same line of the instruction that opens the new block).
	\checklist{11}
		All if, while, do-while, try-catch, and for statements that have only one statement to execute are surrounded by curly braces. Example:
		Avoid this:
		\begin{lstlisting}
			if (condition) 
				doThis();
		\end{lstlisting}
		Instead do this:
		\begin{lstlisting}
			if (condition) 
			{
				doThis();
			}
		\end{lstlisting}
\end{itemize}

\subsubsection{File Organization}
\begin{itemize}
	\checklist{12}
		Blank lines and optional comments are used to separate sections (beginning comments, package/import statements, class/interface declarations which include class variable/attributes declarations, constructors, and methods).
	\checklist{13}
		Where practical, line length does not exceed 80 characters.
	\checklist{14}
		When line length must exceed 80 characters, it does NOT exceed 120 characters.
\end{itemize}

\subsubsection{Wrapping Lines}
\begin{itemize}
	\checklist{15}
		Line break occurs after a comma or an operator.
	\checklist{16}
		Higher-level breaks are used.
	\checklist{17}
		A new statement is aligned with the beginning of the expression at the same level as the previous line.
\end{itemize}

\subsubsection{Comments}
\begin{itemize}
	\checklist{18}
		Comments are used to adequately explain what the class, interface, methods, and blocks of code are doing.
	\checklist{19}
		Commented out code contains a reason for being commented out and a date it can be removed from the source file if determined it is no longer needed.
\end{itemize}

\subsubsection{Java Source Files}
\begin{itemize}
	\checklist{20}
		Each Java source file contains a single public class or interface.
	\checklist{21}
		The public class is the first class or interface in the file.
	\checklist{22}
		Check that the external program interfaces are implemented consistently with what is described in the javadoc.
	\checklist{23}
		Check that the javadoc is complete (i.e., it covers all classes and files part of the set of classes assigned to you).
\end{itemize}

\subsubsection{Package and Import Statements}
\begin{itemize}
	\checklist{24}
		If any package statements are needed, they should be the first non-comment statements. Import statements follow.
\end{itemize}

\subsubsection{Class and Interface Declarations}
\begin{itemize}
	\checklist{25}
		The class or interface declarations shall be in the following order:
		\begin{enumerate}
			\item
				class/interface documentation comment
			\item
				class or interface statement
			\item
				class/interface implementation comment, if necessary
			\item
				class (static) variables
				\begin{enumerate}
					\item first public class variables
					\item next protected class variables
					\item next package level (no access modifier)
					\item last private class variables
				\end{enumerate}
			\item
				instance variables
				\begin{enumerate}
					\item first public instance variables
					\item next protected instance variables
					\item next package level (no access modifier)
					\item last private instance variables
				\end{enumerate}
			\item
				constructors
			\item
				methods
		\end{enumerate}
	\checklist{26}
		Methods are grouped by functionality rather than by scope or accessibility.
	\checklist{27}
		Check that the code is free of duplicates, long methods, big classes, breaking encapsulation, as well as if coupling and cohesion are adequate.
\end{itemize}

\subsubsection{Initialization and Declarations}
\begin{itemize}
	\checklist{28}
		Check that variables and class members are of the correct type. Check that they have the right visibility (public/private/protected)
	\checklist{29}
		Check that variables are declared in the proper scope
	\checklist{30}
		Check that constructors are called when a new object is desired
	\checklist{31}
		Check that all object references are initialized before use
	\checklist{32}
		Variables are initialized where they are declared, unless dependent upon a computation
	\checklist{33}
		Declarations appear at the beginning of blocks (A block is any code surrounded by curly braces “{“ and “}” ). The exception is a variable can be declared in a ‘for’ loop.
\end{itemize}

\subsubsection{Method Calls}
\begin{itemize}
	\checklist{34}
		Check that parameters are presented in the correct order
	\checklist{35}
		Check that the correct method is being called, or should it be a different method with a similar name
	\checklist{36}
		Check that method returned values are used properly
\end{itemize}

\subsubsection{Arrays}
\begin{itemize}
	\checklist{37}
		Check that there are no off-by-one errors in array indexing (that is, all required array elements are correctly accessed through the index)
	\checklist{38}
		Check that all array (or other collection) indexes have been prevented from going out-of-bounds
	\checklist{39}
		Check that constructors are called when a new array item is desired
\end{itemize}

\subsubsection{Object Comparison}
\begin{itemize}
	\checklist{40}
		Check that all objects (including Strings) are compared with "equals" and not with "=="
\end{itemize}

\subsubsection{Output Format}
\begin{itemize}
	\checklist{41}
		Check that displayed output is free of spelling and grammatical errors
	\checklist{42}
		Check that error messages are comprehensive and provide guidance as to how to correct the problem
	\checklist{43}
		Check that the output is formatted correctly in terms of line stepping and spacing
\end{itemize}

\subsubsection{Computation, Comparisons and Assignments}
\begin{itemize}
	\checklist{44}
		Check that the implementation avoids “brutish programming: (see \href{http://users.csc.calpoly.edu/~jdalbey/SWE/CodeSmells/bonehead.html}{Brutish Programming})
	\checklist{45}
		Check order of computation/evaluation, operator precedence and parenthesizing
	\checklist{46}
		Check the liberal use of parenthesis is used to avoid operator precedence problems.
	\checklist{47}
		Check that all denominators of a division are prevented from being zero
	\checklist{48}
		Check that integer arithmetic, especially division, are used appropriately to avoid causing unexpected truncation/rounding
	\checklist{49}
		Check that the comparison and Boolean operators are correct
	\checklist{50}
		Check throw-catch expressions, and check that the error condition is actually legitimate
	\checklist{51}
		Check that the code is free of any implicit type conversions
\end{itemize}

\subsubsection{Exceptions}
\begin{itemize}
	\checklist{52}
		Check that the relevant exceptions are caught
	\checklist{53}
		Check that the appropriate action are taken for each catch block
\end{itemize}

\subsubsection{Flow of Control}
\begin{itemize}
	\checklist{54}
		In a switch statement, check that all cases are addressed by break or return
	\checklist{55}
		Check that all switch statements have a default branch
	\checklist{56}
		Check that all loops are correctly formed, with the appropriate initialization, increment and termination expressions
\end{itemize}

\subsubsection{Files}
\begin{itemize}
	\checklist{57}
		Check that all files are properly declared and opened
	\checklist{58}
		Check that all files are closed properly, even in the case of an error
	\checklist{59}
		Check that EOF conditions are detected and handled correctly
	\checklist{60}
		Check that all file exceptions are caught and dealt with accordingly
\end{itemize}

\subsection{Class Issues}
In this subsection are listed the issues related to the whole class and not only to a specific method.
\subsubsection{Naming Conventions}
\begin{itemize}
	\checklist{1} %Class Related
%	All class names, interface names, method names, class variables, method variables, and constants used should have meaningful names and do what the name suggests.
		\begin{itemize}
			\item
				The class has the capability of creating also \textbf{Plain Text Socket} and \textbf{Plain Text Server Socket}, even if the name \textbf{IIOPSSLSocketFactory} clearly underlines that the class has to be a \textbf{Factory} of \textbf{Secure Sockets} and \textbf{Server Sockets}.
				In order to give the architecture the ability of creating \textbf{Plain Text Socket} and \textbf{Plain Text Server Socket}, a separate class  has to be thought.
			\item
				The method
				\renderPartialCode{219}{222}
				has a name which is not meaningful at all.
				It's code could in fact be moved into the \textbf{SSLInfo constructor}.
			\item
				The method
				\renderPartialCode{209}{211}
				has a name which is not really meaningful.
				It could have been omitted and substituted by a \textbf{SSLInfo constant} inside either the \textbf{IIOPSSLSocketFactory} class or the \textbf{SSLInfo} inner class.
			\item
				The method
				\renderPartialCode{597}{602}
				has a name that is not really meaningful, and its functionality is in fact disjointed by the one of \textbf{Socket and Server Socket Creation}.
			\item
				The method 
				\renderPartialCode{537}{538}
				has a name that is not really meaningful, and its functionality is in fact disjointed by the one of \textbf{Socket and Server Socket Creation}.
			\item
				The method
				\renderPartialCode{348}{360}
				is simply terrible.
				No meaningful name, useless parameters and fuzzy functionality.
		\end{itemize}
	%\checklist{2}  %Class Related Ok
%	If one-character variables are used, they are used only for temporary “throwaway” variables, such as those used in for loops.
	%\checklist{3} %Class Related Ok
%	Class names are nouns, in mixed case, with the first letter of each word in capitalized. Examples: class Raster; class ImageSprite;
	%\checklist{4} %Class Related Ok
%	Interface names should be capitalized like classes.
%	\checklist{5} % Class Related Ok
%	Method names should be verbs, with the first letter of each addition word capitalized. Examples: getBackground(); computeTemperature().
%	\checklist{6} %Class Related Ok
%	Class variables, also called attributes, are mixed case, but might begin with an underscore (‘\_’) followed by a lowercase first letter. All the remaining words in the variable name have their first letter capitalized. Examples: \_windowHeight, timeSeriesData.
	\checklist{7} %Class Related
%	Constants are declared using all uppercase with words separated by an underscore. Examples: MIN\_WIDTH; MAX\_HEIGHT;
		The constant \textbf{\_logger} follows the naming convention of normal attributes, even if it is a constant.
		\renderPartialCode{91}{92}
\end{itemize}

\subsubsection{Comments}
\begin{itemize}
	\checklist{18} %Also Class Related
%	Comments are used to adequately explain what the class, interface, methods, and blocks of code are doing.
		The only adequate comment in the whole class is at line 337.
		\renderPartialCode{336}{337}
		The class is not adequately commented at all.
	\checklist{19} %Also Class Related
%	Commented out code contains a reason for being commented out and a date it can be removed from the source file if determined it is no longer needed.
		The commented out code at lines 103, 135 and 254 is left alone without any additional hint.
		\renderPartialCode{103}{103}
		\renderPartialCode{134}{136}
		\renderPartialCode{253}{255}
\end{itemize}

%Class Related
\subsubsection{Java Source Files}
\begin{itemize}
	\checklist{20}
%	Each Java source file contains a single public class or interface.
		The main class contains an internal class named \textbf{SSLInfo}:
		\renderPartialCode{615}{636}
%	\checklist{21} Ok
%	The public class is the first class or interface in the file.
%	\checklist{22}
%	Check that the external program interfaces are implemented consistently with what is described in the javadoc.
	\checklist{23}
%	Check that the javadoc is complete (i.e., it covers all classes and files part of the set of classes assigned to you).
	The provided \textit{Javadoc} is not complete and is not an help in understanding the class behaviour.
\end{itemize}

%Class Related
%\subsubsection{Package and Import Statements}
%\begin{itemize}
%	\checklist{24} Ok
%	If any package statements are needed, they should be the first non-comment statements. Import statements follow.
%\end{itemize}

%Class Related
\subsubsection{Class and Interface Declarations}
\begin{itemize}
%	\checklist{25} Ok
%	The class or interface declarations shall be in the following order:
%	\begin{enumerate}
%		\item
%		class/interface documentation comment
%		\item
%		class or interface statement
%		\item
%		class/interface implementation comment, if necessary
%		\item
%		class (static) variables
%		\begin{enumerate}
%			\item first public class variables
%			\item next protected class variables
%			\item next package level (no access modifier)
%			\item last private class variables
%		\end{enumerate}
%		\item
%		instance variables
%		\begin{enumerate}
%			\item first public instance variables
%			\item next protected instance variables
%			\item next package level (no access modifier)
%			\item last private instance variables
%		\end{enumerate}
%		\item
%		constructors
%		\item
%		methods
%	\end{enumerate}
	\checklist{26}
%	Methods are grouped by functionality rather than by scope or accessibility.
		The methods returning a \textit{SSLInfo Object}, for example \textit{getDefaultSslInfo()} and \textit{init()}, should be declared in the \textit{SSLInfo} inner class but here they are implemented in the main class. 
		This is an hint for improving the readability of the code and for avoiding the continuous scrolling of the class code
		\renderPartialCode{209}{209}
		\renderPartialCode{219}{222}
	\checklist{27}
%	Check that the code is free of duplicates, long methods, big classes, breaking encapsulation, as well as if coupling and cohesion are adequate.
		The constructor \textit{IIOPSSLSocketFactory()} is 75 lines of code long and it is not easily comprehensible.
		Furthermore it is full of \textit{if - else} structures with no meaningful conditions.
		A such long method worsens the readability and the instant comprehension of the method's role in the code.
		In addition it would be better to refactor and separate the atomic parts of code in additional methods.
\end{itemize}

\subsubsection{Initialization and Declarations}
\begin{itemize}
	\checklist{28}
	\begin{itemize}
		\itemBold{Lines 94 to 97} The private variables \textit{TLS, SSLv3, SSLv2, SSL} are used as simple string instead of creating an enumeration for better show the logic bond between each one and the other ones.
		\renderPartialCode{94}{97}
	\end{itemize}
	\checklist{32}
	\begin{itemize}
		\itemBold{Line 118} The \textit{clientSslInfo Object} is initialized with \textit{null} value even if it is useless.
		\renderPartialCode{118}{118}
	\end{itemize}
\end{itemize}

\subsection{Method Issues}
The checklist has also been checked analysing the assigned source code file method by method.

\subsubsection{Issues in createServerSocket}
\renderPartialCode{266}{301}
The following problems have been found in this method.

%\paragraph{Naming Conventions} Ok
%\begin{itemize}
%	\checklist{1} Ok
%	\checklist{2} Ok
%	\checklist{3} Class Related
%	\checklist{4} Class Related
%	\checklist{5} Ok
%	\checklist{6} Class Related
%	\checklist{7} Class Related
%\end{itemize}

\paragraph{Indention}
\begin{itemize}
	\checklist{8}
	 	\begin{itemize}
			\itemBold{Lines 277 to 300}
				The whole method lacks a level of indentation.
			\itemBold{Lines 287 to 295}
				Have an extra level of indentation.
		\end{itemize}
	\checklist{9}
	 	\begin{itemize}
	 		\itemBold{Lines 278, 281, 283, 286, 300}
		 		Are indented using one tab.
	 		\itemBold{Lines 279, 285, 297, 298}
		 		Are indented using one tab and four spaces.
	 		\itemBold{Line 284}
		 		Is indented using two tabs.
	 	\end{itemize}
\end{itemize}

%\paragraph{Braces} Ok
%\begin{itemize}
%	\checklist{10} Ok
%	\checklist{11} Ok
%\end{itemize}

\paragraph{File Organization}
\begin{itemize}
	\checklist{12}
		 Lines 277, 289, 296 and 299 are blank without a clear reason.
	%\checklist{13} Ok
	%\checklist{14} Ok
\end{itemize}

\paragraph{Wrapping Lines}
\begin{itemize}
	\checklist{15}
		 \begin{itemize}
		 	\itemBold{Line 279}
		 		Is broken before an operator.
		 	\itemBold{Line 288}
		 		Is broken at an open parenthesis.
		 \end{itemize}
	\checklist{16}
		\begin{itemize}
			\itemBold{Line 288}
				A lower-level break occurs.
		\end{itemize}
	\checklist{17}
		\begin{itemize}
			\itemBold{Lines 280}
				Is aligned with an extra level of indentation.
		\end{itemize}
\end{itemize}

\paragraph{Comments}
\begin{itemize}
	\checklist{18}
		The provided \textit{JavaDoc} is too short and not really explicative.
		It does not completely explain the method functionalities.
		\renderPartialCode{266}{274}
	%\checklist{19} Ok
\end{itemize}

\paragraph{Initialization and Declarations}
\begin{itemize}
	\checklist{28}
		For the parameter \textbf{type}, it would have been better to use an enumeration instead of a \textbf{String}.
	%\checklist{29} Ok
	%\checklist{30} Ok
	%\checklist{31} Ok
	\checklist{32}
		At line 285, the \textbf{serverSocket} local variable is initialized to \textbf{null} without reason.
	%\checklist{33} Ok
\end{itemize}

%\paragraph{Method Calls} Ok
%\begin{itemize}
%	\checklist{34} Ok
%	\checklist{35} Ok
%	\checklist{36} Ok
%\end{itemize}

%\paragraph{Arrays} Ok
%\begin{itemize}
%	\checklist{37} Ok
%	\checklist{38} Ok
%	\checklist{39} Ok
%\end{itemize}

%\paragraph{Object Comparison} Ok
%\begin{itemize}
%	\checklist{40} Ok
%\end{itemize}

%\paragraph{Output Format} Ok
%\begin{itemize}
	%\checklist{41} Ok
	%\checklist{42} Ok
	%\checklist{43} Ok
%\end{itemize}

\paragraph{Computation, Comparisons and Assignments}
\begin{itemize}
	\checklist{44}
		\begin{itemize}
			\itemBold{Lines 283 to 284}
				The \textbf{if} condition is not explicit and requires inference to be fully understood.
				In addition it is error prone.
				The whole problem should have been faced using an \textbf{enumeration} instead of \textbf{String} constants.
				\renderPartialCode{283}{284}
			\itemBold{Line 287}
				The initialization to \textbf{null} could have been omitted.
				\renderPartialCode{287}{287}
			\itemBold{Lines 288 to 293}
				The local variable \textbf{serverSocketChannel} is useless, and the \textbf{if} condition could have been wrapped in a boolean private method for better readability.
				\renderPartialCode{288}{293}
		\end{itemize}
%	\checklist{45} Ok
%	\checklist{46} Ok
%	\checklist{47} Ok
%	\checklist{48} Ok
%	\checklist{49} Ok
%	\checklist{50} Ok
%	\checklist{51} Ok
\end{itemize}

%\paragraph{Exceptions} Ok
%\begin{itemize}
%	\checklist{52} Ok
%	\checklist{53} Ok
%\end{itemize}

%\paragraph{Flow of Control} Ok
%\begin{itemize}
%	\checklist{54} Ok
%	\checklist{55} Ok
%	\checklist{56} Ok
%\end{itemize}

%\paragraph{Files} Ok
%\begin{itemize}
%	\checklist{57} Ok
%	\checklist{58} Ok
%	\checklist{59} Ok
%	\checklist{60} Ok
%\end{itemize}

\subsubsection{Issues in createSocket}
\renderPartialCode{303}{346}
The following problems have been found in this method.

%\paragraph{Naming Conventions} Ok
%\begin{itemize}
	%\checklist{1} Ok
	%		All class names, interface names, method names, class variables, method variables, and constants used should have meaningful names and do what the name suggests.
	%\checklist{2} Ok
	%		If one-character variables are used, they are used only for temporary “throwaway” variables, such as those used in for loops.
	%	\checklist{3}
	%		Class names are nouns, in mixed case, with the first letter of each word in capitalized. Examples: class Raster; class ImageSprite;
	%	\checklist{4} Class Related
	%		Interface names should be capitalized like classes.
	%\checklist{5} Ok
	%Method names should be verbs, with the first letter of each addition word capitalized. Examples: getBackground(); computeTemperature().
	%	\checklist{6} Class Related
	%		Class variables, also called attributes, are mixed case, but might begin with an underscore (‘\_’) followed by a lowercase first letter. All the remaining words in the variable name have their first letter capitalized. Examples: \_windowHeight, timeSeriesData.
	%	\checklist{7} Class Related
	%		Constants are declared using all uppercase with words separated by an underscore. Examples: MIN\_WIDTH; MAX\_HEIGHT;
%\end{itemize}

\paragraph{Indention}
\begin{itemize}
	\checklist{8}
	%Three or four spaces are used for indentation and done so consistently
	\begin{itemize}
		\itemBold{Lines 313 to 345}
			The whole method is not indented correctly.
		\itemBold{Lines 317, 320, 342}
			Lack an extra level of indentation, over the one mentioned above.
	\end{itemize}
	\checklist{9}
		%No tabs are used to indent
		Excluding lines 311, 322, 326, 327, 328, 332, 333, 335, 336, 338 and 346, tabs are always used to indent, in conjugation with four spaces.
\end{itemize}

%\paragraph{Braces}
%\begin{itemize}
	%\checklist{10} Ok
	%		Consistent bracing style is used, either the preferred “Allman” style (first brace goes underneath the opening block) or the “Kernighan and Ritchie” style (first brace is on the same line of the instruction that opens the new block).
	%\checklist{11} Ok
	%		All if, while, do-while, try-catch, and for statements that have only one statement to execute are surrounded by curly braces. Example:
	%		Avoid this:
	%		\begin{lstlisting}
	%			if (condition) 
	%				doThis();
	%		\end{lstlisting}
	%		Instead do this:
	%		\begin{lstlisting}
	%			if (condition) 
	%			{
	%				doThis();
	%			}
	%		\end{lstlisting}
%\end{itemize}

\paragraph{File Organization}
\begin{itemize}
	\checklist{12}
	%		Blank lines and optional comments are used to separate sections (beginning comments, package/import statements, class/interface declarations which include class variable/attributes declarations, constructors, and methods).
		\begin{itemize}
			\itemBold{Lines 312, 326, 335}
				Are blank without reason.
		\end{itemize}
	\checklist{13}
	%		Where practical, line length does not exceed 80 characters.
		\begin{itemize}
			\itemBold{Line 304}
				Is 82 characters long.
			\itemBold{Line 310}
				Is 81 characters long.
			\itemBold{Line 317}
				Is 90 characters long.
			\itemBold{Line 329}
				Is 95 characters long.
		\end{itemize}
	%\checklist{14} Ok
	%		When line length must exceed 80 characters, it does NOT exceed 120 characters.
\end{itemize}

\paragraph{Wrapping Lines}
\begin{itemize}
	\checklist{15}
	%		Line break occurs after a comma or an operator.
		\begin{itemize}
			\itemBold{Line 327, 328}
				A break occurs after an open parenthesis.
		\end{itemize}
	\checklist{16}
	%		Higher-level breaks are used.
		\begin{itemize}
			
			\itemBold{Lines 327, 328}
				Low-level break is used.
		\end{itemize}
	\checklist{17}
	%		A new statement is aligned with the beginning of the expression at the same level as the previous line.
		\begin{itemize}
			\itemBold{Line 311}
				Has an extra level of indentation.
		\end{itemize}	
\end{itemize}

\paragraph{Comments}
\begin{itemize}
	\checklist{18}
	%		Comments are used to adequately explain what the class, interface, methods, and blocks of code are doing.
		The whole method is not commented enough.
	%\checklist{19} Ok
	%		Commented out code contains a reason for being commented out and a date it can be removed from the source file if determined it is no longer needed.
\end{itemize}

\paragraph{Initialization and Declarations}
\begin{itemize}
	\checklist{28}
	%		Check that variables and class members are of the correct type. Check that they have the right visibility (public/private/protected)
		At line 401 the variable \textbf{ss} should have been declared of type \textbf{SSLServerSocket} instead of plain \textbf{ServerSocket}.
	\checklist{29}
	%		Check that variables are declared in the proper scope
		At lines 332, 332 the local variables \textbf{host} and \textbf{port} could have been used instead for better code readability.
		\renderPartialCode{314}{315}
		\renderPartialCode{332}{333}
	%\checklist{30} Ok
	%		Check that constructors are called when a new object is desired
	%\checklist{31} Ok
	%		Check that all object references are initialized before use
	\checklist{32}
	%		Variables are initialized where they are declared, unless dependent upon a computation
		Line 322, the \textbf{socket} variable is initialized to \textbf{null}, but that value is immediately overwritten and therefore the initialization is useless.
	%\checklist{33}
	%		Declarations appear at the beginning of blocks (A block is any code surrounded by curly braces “{“ and “}” ). The exception is a variable can be declared in a ‘for’ loop.
\end{itemize}

%\paragraph{Method Calls} Ok
%\begin{itemize}
	%\checklist{34}
	%		Check that parameters are presented in the correct order
	%\checklist{35}
	%		Check that the correct method is being called, or should it be a different method with a similar name
	%\checklist{36}
	%		Check that method returned values are used properly
%\end{itemize}

%\paragraph{Arrays} Ok
%\begin{itemize}
	%\checklist{37} Ok
	%		Check that there are no off-by-one errors in array indexing (that is, all required array elements are correctly accessed through the index)
	%\checklist{38} Ok
	%		Check that all array (or other collection) indexes have been prevented from going out-of-bounds
	%\checklist{39} Ok
	%		Check that constructors are called when a new array item is desired
%\end{itemize}

%\paragraph{Object Comparison} Ok
%\begin{itemize}
	%\checklist{40} Ok
	%		Check that all objects (including Strings) are compared with "equals" and not with "=="
%\end{itemize}

\paragraph{Output Format}
\begin{itemize}
	%\checklist{41} Ok
	%		Check that displayed output is free of spelling and grammatical errors
	\checklist{42}
	%		Check that error messages are comprehensive and provide guidance as to how to correct the problem
		The error message
		\renderPartialCode{342}{342}
		is not explaining anything about the error that has occurred.
	%\checklist{43}
	%		Check that the output is formatted correctly in terms of line stepping and spacing
\end{itemize}

\paragraph{Computation, Comparisons and Assignments}
\begin{itemize}
	\checklist{44}
	%		Check that the implementation avoids “brutish programming: (see \href{http://users.csc.calpoly.edu/~jdalbey/SWE/CodeSmells/bonehead.html}{Brutish Programming})
		\begin{itemize}
			\itemBold{Line 319}
				The \textbf{if} condition
				\renderPartialCode{319}{319}
				is not clear enough, invoking a dedicate boolean method and using enumeration could have delivered better results.
			\itemBold{Lines 327, 328}
				The \textbf{if} condition
				\renderPartialCode{327}{328}
				is not clear enough, invoking a dedicate boolean method and using enumeration could have delivered better results.
			\itemBold{Line 329}
				The local variable \textbf{socketChannel}
				\renderPartialCode{329}{331}
				is useless.	
		\end{itemize}
	%\checklist{45} Ok
	%		Check order of computation/evaluation, operator precedence and parenthesizing
	%\checklist{46} Ok
	%		Check the liberal use of parenthesis is used to avoid operator precedence problems.
	%\checklist{47} Ok
	%		Check that all denominators of a division are prevented from being zero
	%\checklist{48} Ok
	%		Check that integer arithmetic, especially division, are used appropriately to avoid causing unexpected truncation/rounding
	%\checklist{49} Ok
	%		Check that the comparison and Boolean operators are correct
	%\checklist{50} See \checklist{52}
	%		Check throw-catch expressions, and check that the error condition is actually legitimate
	%\checklist{51} Ok
	%		Check that the code is free of any implicit type conversions
\end{itemize}

\paragraph{Exceptions}
\begin{itemize}
	\checklist{52}
	%		Check that the relevant exceptions are caught
		\begin{itemize}
			\itemBold{Lines 340 to 345}
				The \textbf{catch} block
				\renderPartialCode{340}{345}
				is actually catching a generic \textbf{Exception} instead of the generated ones. 
		\end{itemize}
	\checklist{53}
	%		Check that the appropriate action are taken for each catch block
		\begin{itemize}
			\itemBold{Lines 340 to 345}
				The \textbf{catch} block mentioned above is only outputting a generic log and re-throwing a generic \textbf{RuntimeException}, built using the caught one.
		\end{itemize}
\end{itemize}

%\paragraph{Flow of Control} Ok
%\begin{itemize}
	%\checklist{54} Ok
	%		In a switch statement, check that all cases are addressed by break or return
	%\checklist{55} Ok
	%		Check that all switch statements have a default branch
	%\checklist{56} Ok
	%		Check that all loops are correctly formed, with the appropriate initialization, increment and termination expressions
%\end{itemize}

%\paragraph{Files} Ok
%\begin{itemize}
	%\checklist{57} Ok
	%		Check that all files are properly declared and opened
	%\checklist{58} Ok
	%		Check that all files are closed properly, even in the case of an error
	%\checklist{59} Ok
	%		Check that EOF conditions are detected and handled correctly
	%\checklist{60} Ok
	%		Check that all file exceptions are caught and dealt with accordingly
%\end{itemize}

\subsubsection{Issues in createSSLServerSocket}
\renderPartialCode{364}{430}
The following problems have been found in this method.

\paragraph{Naming Conventions}
\begin{itemize}
	\checklist{1}
	%		All class names, interface names, method names, class variables, method variables, and constants used should have meaningful names and do what the name suggests.
		\begin{itemize}
			\itemBold{Lines 376, 377}
				The difference between \textit{port} and \textit{iport} should be more highlighted through the naming choices.
				\renderPartialCode{376}{377}
			\itemBold{Line 384}
				The variable name \textit{ssf} is not really meaningful.
				\renderPartialCode{384}{384}
			\itemBold{Line 393}
				The variable name \textit{cs} is not really meaningful.
				\renderPartialCode{393}{393}
			\itemBold{Line 401}
				The variable name \textit{ss} is not really meaningful.
				\renderPartialCode{401}{401}
		\end{itemize}
	%\checklist{2} Ok
	%		If one-character variables are used, they are used only for temporary “throwaway” variables, such as those used in for loops.
	%\checklist{3} Class Related
	%		Class names are nouns, in mixed case, with the first letter of each word in capitalized. Examples: class Raster; class ImageSprite;
	%\checklist{4} Class Related
	%		Interface names should be capitalized like classes.
	%\checklist{5} Ok
	%Method names should be verbs, with the first letter of each addition word capitalized. Examples: getBackground(); computeTemperature().
	%\checklist{6} Class Related
	%		Class variables, also called attributes, are mixed case, but might begin with an underscore (‘\_’) followed by a lowercase first letter. All the remaining words in the variable name have their first letter capitalized. Examples: \_windowHeight, timeSeriesData.
	%	\checklist{7} Class Related
	%		Constants are declared using all uppercase with words separated by an underscore. Examples: MIN\_WIDTH; MAX\_HEIGHT;
\end{itemize}

\paragraph{Indention}
\begin{itemize}
	\checklist{8}
	%Three or four spaces are used for indentation and done so consistently
		\begin{itemize}
			\itemBold{Lines 368 to 430}
				The whole method is not indented correctly.
			\itemBold{Lines 398, 419, 420}
				Lack an extra level of indentation, over the one mentioned above.
		\end{itemize}
	\checklist{9}
	%No tabs are used to indent
		\begin{itemize}
			\itemBold{Lines 396, 397, 399, 418, 421, 423, 424, 427}
				These lines are indented using four spaces and one tab.
				This approach is neither consistent with the (wrong) style adopted in the whole method.
		\end{itemize}
\end{itemize}

%\paragraph{Braces}
%\begin{itemize}
	%\checklist{10} Ok "Allman"
	%		Consistent bracing style is used, either the preferred “Allman” style (first brace goes underneath the opening block) or the “Kernighan and Ritchie” style (first brace is on the same line of the instruction that opens the new block).
	%\checklist{11} Ok
	%		All if, while, do-while, try-catch, and for statements that have only one statement to execute are surrounded by curly braces. Example:
	%		Avoid this:
	%		\begin{lstlisting}
	%			if (condition) 
	%				doThis();
	%		\end{lstlisting}
	%		Instead do this:
	%		\begin{lstlisting}
	%			if (condition) 
	%			{
	%				doThis();
	%			}
	%		\end{lstlisting}
%\end{itemize}

\paragraph{File Organization}
\begin{itemize}
	\checklist{12}
	%		Blank lines and optional comments are used to separate sections (beginning comments, package/import statements, class/interface declarations which include class variable/attributes declarations, constructors, and methods).
		\begin{itemize}
			\itemBold{Lines 392, 394, 416}
				Are blank without reason.
		\end{itemize}
	\checklist{13}
	%		Where practical, line length does not exceed 80 characters.
		\begin{itemize}
			\itemBold{Line 368}
				Is 113 characters long.
			\itemBold{Line 372}
				Is 92 characters long.
			\itemBold{Line 380}
				Is 93 characters long.
			\itemBold{Line 411}
				Is 101 characters long.
		\end{itemize}
	%\checklist{14} Ok
	%		When line length must exceed 80 characters, it does NOT exceed 120 characters.
\end{itemize}

\paragraph{Wrapping Lines}
\begin{itemize}
	\checklist{15}
	%		Line break occurs after a comma or an operator.
		\begin{itemize}
			\itemBold{Lines 372 and 380}
				The line break occurs after an open rounded bracket.
				\renderPartialCode{372}{374} 
				\renderPartialCode{380}{382}
		\end{itemize}
	%\checklist{16} Ok
	%		Higher-level breaks are used.co
	%\checklist{17} Ok
	%		A new statement is aligned with the beginning of the expression at the same level as the previous line.
\end{itemize}

\paragraph{Comments}
\begin{itemize}
	\checklist{18}
	%		Comments are used to adequately explain what the class, interface, methods, and blocks of code are doing.
		The provided \textit{JavaDoc} is too short and not really explicative.
		It does not completely explain the method functionalities.
		\renderPartialCode{364}{367}
	%\checklist{19}
	%		Commented out code contains a reason for being commented out and a date it can be removed from the source file if determined it is no longer needed.
\end{itemize}

\paragraph{Initialization and Declarations}
\begin{itemize}
	%\checklist{28} Ok
	%		Check that variables and class members are of the correct type. Check that they have the right visibility (public/private/protected)
	%\checklist{29} Ok
	%		Check that variables are declared in the proper scope
	%\checklist{30} Ok
	%		Check that constructors are called when a new object is desired
	\checklist{31}
	%		Check that all object references are initialized before use
		\begin{itemize}
			\itemBold{Lines 393 and 393}
				The variable \textit{cs} is initialized to \textit{null} even if is useless.
				\renderPartialCode{393}{393}
			\itemBold{Lines 401 and 401}
				The variable \textit{ss} is initialized to \textit{null} even if is useless.
				\renderPartialCode{401}{401}
		\end{itemize}
	%\checklist{32} Ok
	%		Variables are initialized where they are declared, unless dependent upon a computation
	\checklist{33} 
	%		Declarations appear at the beginning of blocks (A block is any code surrounded by curly braces “{“ and “}” ). The exception is a variable can be declared in a ‘for’ loop.
		\begin{itemize}
			\itemBold{Lines 393 and 401}
				The \textit{cs} and \textit{ss} variables are declared at the middle of the method code.
				They have to be declared at the beginning of it.
				\renderPartialCode{393}{393}
				\renderPartialCode{401}{401}
		\end{itemize}
\end{itemize}

%\paragraph{Method Calls} Ok
%\begin{itemize}
%	\checklist{34}
	%		Check that parameters are presented in the correct order
%	\checklist{35}
	%		Check that the correct method is being called, or should it be a different method with a similar name
%	\checklist{36}
	%		Check that method returned values are used properly
%\end{itemize}

%\paragraph{Arrays} Ok
%\begin{itemize}
%	\checklist{37}
	%		Check that there are no off-by-one errors in array indexing (that is, all required array elements are correctly accessed through the index)
%	\checklist{38}
	%		Check that all array (or other collection) indexes have been prevented from going out-of-bounds
%	\checklist{39}
	%		Check that constructors are called when a new array item is desired
%\end{itemize}

%\paragraph{Object Comparison} Ok
%\begin{itemize}
	%\checklist{40} Ok
	%		Check that all objects (including Strings) are compared with "equals" and not with "=="
	%TODO Re@Albi: Per checkare se un oggetto è null è giusto usare ==, perché mettiamo che hai String pippo = null; se poi fai pippo.equals(null) ottieni una NullPointerException perché pippo è null!	
		%\begin{itemize}
		%	\itemBold{Line 371, 379, 388, 407}
		%		The comparison between \textit{null} and other objects is done through the \textit{==} operator, insted of using \textit{equals()}. 
		%		\renderPartialCode{371}{371}
		%		\renderPartialCode{379}{379}
		%		\renderPartialCode{388}{388}
		%		\renderPartialCode{407}{407}
		%\end{itemize}
%\end{itemize}

\paragraph{Output Format}
\begin{itemize}
	%\checklist{41} Ok
	%		Check that displayed output is free of spelling and grammatical errors
	\checklist{42}
	%		Check that error messages are comprehensive and provide guidance as to how to correct the problem
		In the \textit{catch blocks} the caught exceptions are not explained to the user, they are only printed out.
		\begin{itemize}
			\itemBold{Line 411}
				\renderPartialCode{411}{413}
			\itemBold{Line 423}
				\renderPartialCode{423}{423}
		\end{itemize}
	%\checklist{43} Ok
	%		Check that the output is formatted correctly in terms of line stepping and spacing
\end{itemize}

%\paragraph{Computation, Comparisons and Assignments}
%\begin{itemize}
%	\checklist{44}
	%		Check that the implementation avoids “brutish programming: (see \href{http://users.csc.calpoly.edu/~jdalbey/SWE/CodeSmells/bonehead.html}{Brutish Programming})
	%\checklist{45} Ok
	%		Check order of computation/evaluation, operator precedence and parenthesizing
	%\checklist{46} Ok
	%		Check the liberal use of parenthesis is used to avoid operator precedence problems.
	%\checklist{47} Ok
	%		Check that all denominators of a division are prevented from being zero
	%\checklist{48} Ok
	%		Check that integer arithmetic, especially division, are used appropriately to avoid causing unexpected truncation/rounding
	%\checklist{49} Ok
	%		Check that the comparison and Boolean operators are correct
	%\checklist{50} Ok
	%		Check throw-catch expressions, and check that the error condition is actually legitimate
	%\checklist{51} Ok
	%		Check that the code is free of any implicit type conversions
%\end{itemize}

\paragraph{Exceptions}
\begin{itemize}
	\checklist{52}
	%		Check that the relevant exceptions are caught
		\begin{itemize}
			\itemBold{Lines 422 to 425}
				The \textbf{catch} block
				\renderPartialCode{422}{425}
				is actually catching a generic \textbf{Exception} instead of the generated ones.
		\end{itemize}
	\checklist{53}
	%		Check that the appropriate action are taken for each catch block
		\begin{itemize}
			\item
				The problem in the above \textit{code-block} is that, for every generated exception, this block throws an \textbf{IOException}, even if the caught exception is not an \textbf{IO} one.
		\end{itemize}
\end{itemize}

%\paragraph{Flow of Control} Ok
%\begin{itemize}
%	\checklist{54}
	%		In a switch statement, check that all cases are addressed by break or return
%	\checklist{55}
	%		Check that all switch statements have a default branch
%	\checklist{56}
	%		Check that all loops are correctly formed, with the appropriate initialization, increment and termination expressions
%\end{itemize}

%\paragraph{Files} Ok
%\begin{itemize}
%	\checklist{57}
	%		Check that all files are properly declared and opened
%	\checklist{58}
	%		Check that all files are closed properly, even in the case of an error
%	\checklist{59}
	%		Check that EOF conditions are detected and handled correctly
%	\checklist{60}
	%		Check that all file exceptions are caught and dealt with accordingly
%\end{itemize}

\subsubsection{Issues in createSSLSocket}
\renderPartialCode{432}{474}
The following problems have been found in this method.

\paragraph{Naming Conventions}
\begin{itemize}
	\checklist{1} 
		\begin{itemize}
			\item
				The following piece of code contains a variable named \textbf{e2} whose name is not meaningful.
				\renderPartialCode{468}{469}
		\end{itemize}
	%		All class names, interface names, method names, class variables, method variables, and constants used should have meaningful names and do what the name suggests.
	%\checklist{2} Ok
	%		If one-character variables are used, they are used only for temporary “throwaway” variables, such as those used in for loops.
	%	\checklist{3} Ok
	%		Class names are nouns, in mixed case, with the first letter of each word in capitalized. Examples: class Raster; class ImageSprite;
	%	\checklist{4} Class Related
	%		Interface names should be capitalized like classes.
	%\checklist{5} Ok
	%Method names should be verbs, with the first letter of each addition word capitalized. Examples: getBackground(); computeTemperature().
	%	\checklist{6} Class Related
	%		Class variables, also called attributes, are mixed case, but might begin with an underscore (‘\_’) followed by a lowercase first letter. All the remaining words in the variable name have their first letter capitalized. Examples: \_windowHeight, timeSeriesData.
	%	\checklist{7} Class Related
	%		Constants are declared using all uppercase with words separated by an underscore. Examples: MIN\_WIDTH; MAX\_HEIGHT;
\end{itemize}

\paragraph{Indention}
\begin{itemize}
	\checklist{8}
	%Three or four spaces are used for indentation and done so consistently
		\begin{itemize}
			\itemBold{Line 441, 469}
				Lack a level of indentation.
		\end{itemize}
	\checklist{9}
		\begin{itemize}
			\itemBold{Line 441}
				Is indented using a tab instead of four spaces.
		\end{itemize}		
	%No tabs are used to indent
\end{itemize}

%\paragraph{Braces} Ok
%\begin{itemize}
	%\checklist{10} Ok
	%		Consistent bracing style is used, either the preferred “Allman” style (first brace goes underneath the opening block) or the “Kernighan and Ritchie” style (first brace is on the same line of the instruction that opens the new block).
	%\checklist{11} Ok
	%		All if, while, do-while, try-catch, and for statements that have only one statement to execute are surrounded by curly braces. Example:
	%		Avoid this:
	%		\begin{lstlisting}
	%			if (condition) 
	%				doThis();
	%		\end{lstlisting}
	%		Instead do this:
	%		\begin{lstlisting}
	%			if (condition) 
	%			{
	%				doThis();
	%			}
	%		\end{lstlisting}
%\end{itemize}

\paragraph{File Organization}
\begin{itemize}
	\checklist{12}
	%		Blank lines and optional comments are used to separate sections (beginning comments, package/import statements, class/interface declarations which include class variable/attributes declarations, constructors, and methods).
		\begin{itemize}
			\itemBold{Lines 446, 457}
				Are blank without reason.
		\end{itemize}
	\checklist{13}
	%		Where practical, line length does not exceed 80 characters.
		\begin{itemize}
			\itemBold{Line 448}
				Is 99 characters long.
			\itemBold{Line 455}
				Is 90 characters long.
		\end{itemize}		
	%\checklist{14} Ok
	%		When line length must exceed 80 characters, it does NOT exceed 120 characters.
\end{itemize}

\paragraph{Wrapping Lines}
\begin{itemize}
	\checklist{15}
	%		Line break occurs after a comma or an operator.
		\begin{itemize}
			\itemBold{Line 468}
				Is broken at an open parenthesis.
		\end{itemize}
	%\checklist{16} Ok
	%		Higher-level breaks are used.
	\checklist{17}
	%		A new statement is aligned with the beginning of the expression at the same level as the previous line.
		\begin{itemize}
			\itemBold{Lines 465, 469}
				Lack an extra level of indentation.
		\end{itemize}
\end{itemize}

\paragraph{Comments}
\begin{itemize}
	\checklist{18}
	%		Comments are used to adequately explain what the class, interface, methods, and blocks of code are doing.
		Comments and JavaDoc provided in this method are completely useless.
	%\checklist{19}
	%		Commented out code contains a reason for being commented out and a date it can be removed from the source file if determined it is no longer needed.
\end{itemize}

\paragraph{Initialization and Declarations}
\begin{itemize}
	%\checklist{28} Ok
	%		Check that variables and class members are of the correct type. Check that they have the right visibility (public/private/protected)
	%\checklist{29} Ok
	%		Check that variables are declared in the proper scope
	%\checklist{30} Ok
	%		Check that constructors are called when a new object is desired
	%\checklist{31} Ok
	%		Check that all object references are initialized before use
	%\checklist{32} Ok
	%		Variables are initialized where they are declared, unless dependent upon a computation
	\checklist{33}
	%		Declarations appear at the beginning of blocks (A block is any code surrounded by curly braces “{“ and “}” ). The exception is a variable can be declared in a ‘for’ loop.
		\begin{itemize}
			\itemBold{Lines 450, 451, 452, 468}
				The local variables are not initialized at the beginning of their relevant blocks.
		\end{itemize}
\end{itemize}

%\paragraph{Method Calls} Ok
%\begin{itemize}
	%\checklist{34} Ok
	%		Check that parameters are presented in the correct order
	%\checklist{35} Ok
	%		Check that the correct method is being called, or should it be a different method with a similar name
	%\checklist{36} Ok
	%		Check that method returned values are used properly
%\end{itemize}

%\paragraph{Arrays} Ok
%\begin{itemize}
	%\checklist{37} Ok
	%		Check that there are no off-by-one errors in array indexing (that is, all required array elements are correctly accessed through the index)
	%\checklist{38} Ok
	%		Check that all array (or other collection) indexes have been prevented from going out-of-bounds
	%\checklist{39} Ok
	%		Check that constructors are called when a new array item is desired
%\end{itemize}

%\paragraph{Object Comparison} Ok
%\begin{itemize}
	%\checklist{40} Ok
	%		Check that all objects (including Strings) are compared with "equals" and not with "=="
%\end{itemize}

\paragraph{Output Format}
\begin{itemize}
	%\checklist{41} Ok
	%		Check that displayed output is free of spelling and grammatical errors
	\checklist{42}
	%		Check that error messages are comprehensive and provide guidance as to how to correct the problem
		At lines from 462 to 472 the error message generated are a bit too general and not specific.
		They may not really help debugging the problem.
		\renderPartialCode{462}{472}
	%\checklist{43}
	%		Check that the output is formatted correctly in terms of line stepping and spacing
\end{itemize}

%\paragraph{Computation, Comparisons and Assignments} Ok
%\begin{itemize}
	%\checklist{44} Ok
	%		Check that the implementation avoids “brutish programming: (see \href{http://users.csc.calpoly.edu/~jdalbey/SWE/CodeSmells/bonehead.html}{Brutish Programming})
	%\checklist{45} Ok
	%		Check order of computation/evaluation, operator precedence and parenthesizing
	%\checklist{46} Ok
	%		Check the liberal use of parenthesis is used to avoid operator precedence problems.
	%\checklist{47} Ok
	%		Check that all denominators of a division are prevented from being zero
	%\checklist{48} Ok
	%		Check that integer arithmetic, especially division, are used appropriately to avoid causing unexpected truncation/rounding
	%\checklist{49} Ok
	%		Check that the comparison and Boolean operators are correct
	%\checklist{50} Ok
	%		Check throw-catch expressions, and check that the error condition is actually legitimate
	%\checklist{51} Ok
	%		Check that the code is free of any implicit type conversions
%\end{itemize}

\paragraph{Exceptions}
\begin{itemize}
	\checklist{52}
	%		Check that the relevant exceptions are caught
		At lines 462 to 472 is caught a generic \textbf{Exception} instead of the generated ones.
	\checklist{53}
	%		Check that the appropriate action are taken for each catch block
		At lines 462 to 472 a \textbf{IOException} is created in the place of the generic \textbf{Exception} caught, and it is configured and re-thrown.
		This modus operandi destroys information about the error occurred in the first place, given that the logging is poor.
\end{itemize}

%\paragraph{Flow of Control} oK
%\begin{itemize}
	%\checklist{54} Ok
	%		In a switch statement, check that all cases are addressed by break or return
	%\checklist{55} Ok
	%		Check that all switch statements have a default branch
	%\checklist{56} Ok
	%		Check that all loops are correctly formed, with the appropriate initialization, increment and termination expressions
%\end{itemize}

%\paragraph{Files} Ok
%\begin{itemize}
	%\checklist{57} Ok
	%		Check that all files are properly declared and opened
	%\checklist{58} Ok
	%		Check that all files are closed properly, even in the case of an error
	%\checklist{59} Ok
	%		Check that EOF conditions are detected and handled correctly
	%\checklist{60} Ok
	%		Check that all file exceptions are caught and dealt with accordingly
%\end{itemize}