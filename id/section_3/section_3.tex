\section{Issues Found}
%TODO <Report the classes/code fragments that do not fullfill some points in the check list. Explain which point is not fullfilled and why>

\subsection{Code Inspection Checklist}
%TODO Inserire la checklist
In this section are reported all the code choices that do not meet the \textbf{Code Inspection Checklist} given for this document.
\subsection{Class Issues}
In this subsection are listed the issues related to the whole class and not to a specific method:
\begin{itemize}
	\checklist{1}
		\begin{itemize}
			\item
				The method
				\renderPartialCode{219}{222}
				has a name which is not meaningful at all.
				It's code could in fact be moved into the \textbf{SSLInfo constructor}.
			\item
				The method
				\renderPartialCode{209}{211}
				has a name which is not really meaningful.
				It could have been omitted and substituted by a \textbf{SSLInfo constant} inside either the \textbf{IIOPSSLSocketFactory} class or the \textbf{SSLInfo} inner class.
			\item
				The method
				\renderPartialCode{597}{602}
				has a name that is not really meaningful, and its functionality is in fact disjointed by the one of \textbf{Socket and Server Socket Creation}.
			\item
				The method 
				\renderPartialCode{537}{538}
				has a name that is not really meaningful, and its functionality is in fact disjointed by the one of \textbf{Socket and Server Socket Creation}.
			\item
				The method
				\renderPartialCode{348}{360}
				is simply terrible.
				No meaningful name, useless parameters and fuzzy functionality.
		\end{itemize}		
	\checklist{20}
		The main class contains an internal class named \textbf{SSLInfo}:
		\lstinputlisting [firstline=615,lastline=636]{./common/IIOPSSLSocketFactory.java}
	\checklist{26}
		%TODO @Albi: I metodi che ritornano SSLInfo dovrebbero stare nella classe SSLInfo...
		The implementation of the class named \textbf{SSLInfo} is put in the beginning of the file while is better to put it near the methods that have the role of returning an \textbf{SSLInfo} object.
		This is an hint for raising up the readability of the code and for avoiding the continous scrolling of the class code
		\renderPartialCode{209}{209}
		\renderPartialCode{219}{222}
\end{itemize}

\subsection{Method Issues}
We have checked the constraints analysing the assigned source code file method by method:
\begin{itemize}
	\item
		\textbf{public ServerSocket createServerSocket(String type, InetSocketAddress inetSocketAddress) throws IOException}
		\renderPartialCode{266}{301}
		In this method we have found the following problems:
		\begin{itemize}
			%1 ok
			%3 cl
			\checklist{8} 
				\begin{itemize}
					\itemBold{From line 277 to line 300}
						The whole method is not indented ad all.
					\itemBold{Lines 279, 284, 285, 297, 298}
						Are indented with one space and one tab command.
					\itemBold{Lines 280, 290, 292, 294}
						Are indented with three tab commands.
					\itemBold{Lines 287, 288, 293, 295}
						Are indented with two tab commands.
					\itemBold{Line 291}
						Is indented with five tab commands.
					\itemBold{Line 289}
						Is indented with four tab commands.
				\end{itemize}
			\checklist{18}
				The provided javadoc is too short and not too much self-explicable. It doesn't completely explain the class functionalities:
				\renderPartialCode{266}{274}
	\end{itemize}
\end{itemize}