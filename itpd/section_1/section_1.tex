\section{Introduction}


\subsection{Purpose and Scope}
%TODO State the purpose and scope of the document
This document is based on the \textbf{D}esign \textbf{D}ocument. In fact the purpose of the \textbf{I}ntegration \textbf{T}est \textbf{P}lan \textbf{D}ocument is to clearly state the order in which the software components identified in the \textbf{Component View} of the \textbf{DD} have to be integrated one with each other in order to guarantee a well tested final software. Following the exposed procedure ensures that all the software components will communicate and cooperate in the proper way.

\subsection{List of Definitions and Abbreviations}
In the document are often used some technical terms whose definitions are here reported:
\begin{itemize}
	\itemBold{Integration Test Case} An atomic procedure done to test the integration of a component on the top of another one.
	\itemBold{Integration Test Suite} A collection of \textbf{Integration Test Cases}.
	\item See the correspondent section in the \textbf{RASD} and the \textbf{DD} for more definitions.
\end{itemize}
For sake of brevity, some acronyms and abbreviations are used:
\begin{itemize}
	\itemBold{In} Integration Test Suite number n.
	\itemBold{InTm} Integration Test Case number m of the Integration Test Suite number n.
	\itemBold{JS} JavaScript.
	\itemBold{UI} User Interface.
	\item See the correspondent section in the \textbf{RASD} and the \textbf{DD} for more acronyms and abbreviations.
\end{itemize}


\subsection{List of Reference Documents}
%TODO List all reference documents, for instance: The project description, the RASD, the Design Document, The documentation of any tools you plan to use for testing
\begin{itemize}
	\itemBold{\myTaxiService{} RASD v1.3} Requirements Analysis and Specification Document
	\itemBold{\myTaxiService{} DD v1.1} Design Document
	\itemBold{Assignments 4 - Test Plan} Integration testing assignment specification document
\end{itemize}
