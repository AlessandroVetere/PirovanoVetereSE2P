\section{Integration Strategy}

\subsection{Entry Criteria}
%TODO Specify the criteria that must be met before integration testing of specific elements may begin (e.g., functions must have been unit tested).

\subsection{Elements to be Integrated}
%TODO Identify the components to be integrated, refer to your design document to identify such components in a way that is consistent with your design

\subsection{Integration Testing Strategy}
%TODO Describe the integration testing approach (top-down, bottom-up, functional groupings, etc.) and the rationale for the choosing that approach

\subsection{Sequence of Component/Function Integration}
%TODO Note: The structure of this section may vary depending on the integration strategy you select in Section 2.3. Use the structure proposed below as a non mandatory guide.

\subsubsection{Software Integration Sequence}
%TODO For each subsystem: Identify the sequence in which the software components will be integrated within the subsystem. Relate this sequence to any product features/functions that are being built up.

\subsubsection{Subsystem Integration Sequence}
%TODO Identify the order in which subsystems will be integrated. If you have a single subsystem, 2.4.1 and 2.4.2 are to be merged in a single section. You can refer to Section 2.2 of the test plan example [1] as an example of what we expect.