\section{Integration Strategy}

\subsection{Entry Criteria}
%TODO Specify the criteria that must be met before integration testing of specific elements may begin (e.g., functions must have been unit tested).
Before starting the integration testing of any software element that has been designed for \myTaxiService{} system, the relevant functions of the considered element must be unit tested using JUnit framework.

\subsection{Elements to be Integrated}
%TODO Identify the components to be integrated, refer to your design document to identify such components in a way that is consistent with your design
Almost every software component in the Component View section of the Design Document has to be integrated.
The interested components are reported below for readers' convenience:
\begin{itemize}
	\item Passenger View
	\begin{itemize}
		\item PS Application View
		\item PS Web View
		\item PS Request Creator
		\item PS Receiver
	\end{itemize}
	\item Taxi Driver View
	\begin{itemize}
		\item TD Application View
		\item TD Request Creator
		\item TD Receiver
		\item TD Locator
		\item GPS Data Source
	\end{itemize}
	\item Administrator View
	\item Controller
	\begin{itemize}
		\item RESTful Service
		\item Dispatcher
		\item Location Manager
		\item Session Manager
		\item Profile Manager
		\item Taxi Driver Manager
		\item Taxi Ride Manager
		\item Queue Manager
		\item Taxi Sharing Manager
		\item Query Manager
	\end{itemize}
	\item Model
	\begin{itemize}
		\item Model Query Service
		\item Passenger DB Adapter
		\item Taxi Ride DB Adapter
		\item Zone DB Adapter
		\item Queue DB Adapter
		\item Database Driver Adapter
	\end{itemize}
\end{itemize}
Moreover we suppose that Google Maps API are already tested and that the exposed API will be tested in a separate environment using \textbf{JUnit}.

\subsection{Integration Testing Strategy}
%TODO Describe the integration testing approach (top-down, bottom-up, functional groupings, etc.) and the rationale for the choosing that approach

\subsection{Sequence of Component/Function Integration}
%TODO Note: The structure of this section may vary depending on the integration strategy you select in Section 2.3. Use the structure proposed below as a non mandatory guide.

\subsubsection{Software Integration Sequence}
%TODO For each subsystem: Identify the sequence in which the software components will be integrated within the subsystem. Relate this sequence to any product features/functions that are being built up.
We adopted a \textbf{bottom-up} testing strategy; this strategy starts from the testing of the \textbf{Model}'s features.\par
We considered \textbf{Model}, \textbf{Controller} and \textbf{Views} as \textbf{Components Subsystems}.
Firstly the \textit{tester} has to singularly test the component' \textit{sub-features} contained in the \textbf{Model Subsystem}, then, in order to simulate the behaviour of the \textbf{Interacting Software Parts}, he has to represent the \textbf{Controller Component} with a dedicated \textbf{Driver} in order to globally test the \textbf{Model}' features that have to be tested.\par In this way, the model will be completely tested and its behaviour is shown by simulating all the \textbf{Controller}'s possible actions.\par
Then the testing procedure passes through the \textbf{Controller Component} and the test sequence adopts the same strategy used to test the \textbf{Model}. The only difference, watching the procedure from a high point of view, is that in this case the \textbf{Controller} is tested using the already tested \textbf{Model} and using a \textbf{Driver} in order to simulate the \textbf{Views} actions.
The last part of the procedure is the one dedicated to the \textbf{View}. These Components(\textbf{TD View, PS View, Administrator View}) are tested using the already tested components (\textbf{Controller and the Model}) and no one component' behaviour is simulated using \textbf{Subs} or \textbf{Drivers}.
\subsubsection{Subsystem Integration Sequence}
%TODO Identify the order in which subsystems will be integrated. If you have a single subsystem, 2.4.1 and 2.4.2 are to be merged in a single section. You can refer to Section 2.2 of the test plan example [1] as an example of what we expect.
\showPercentImage{./section_2/system_integration_flowchart.png}{Subsystem Integration Flowchart}{0.7}
%prima model con controller poi integriamo le due view
