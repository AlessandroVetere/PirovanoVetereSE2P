\section{Function Points and COCOMO}
%TODO Apply Function Points to estimate the project size and then COCOMO to estimate effort and cost.
%TODO 1) table for weights
%TODO 1.2) describe 
%TODO ext input
%TODO ext ouput
%TODO etc
%TODO for each task asssign a level of complexity
%TODO 2) COCOMO
%TODO use free tools online (logo tipo tempio greco)
%
\subsection{Function Points}
Under the assumption that the dimension of the software can be characterized by correlating the kind of functionalities offered with the source lines of code (SLOC) of the software itself the Function Points approach, defined in 1975 by Allan Albrecht, consist in a technique to assess the effort needed to design and develop custom software applications.
This technique consists in combining the following program characteristics to obtain a final result:
\begin{itemize}
	\item External inputs and outputs,
	\item User interactions,
	\item External Interfaces,
	\item Files used by the system.
\end{itemize}
A weight is associated to each of these characteristics and then final calculations are done in order to obtain an indicative size of the software project in terms of SLOC.\par
The following table relates each types of functionality offered by the system with a weight that depends on the level of complexity of the functionality itself:
\begin{center}
	\begin{tabulary}{\linewidth\tymin=70pt}{Y{2cm}|Y{1.75cm}|Y{1.75cm}|Y{1.75cm}}
		\textbf{Function\newline{}Types} & \textbf{Simple\newline{}Weight} & \textbf{Medium\newline{}Weight} & \textbf{Complex\newline{}Weight} \\ \hline
		Internal Logic File & 7 & 10 & 15 \\ \hline
		External Logic File & 5 & 7 & 10 \\ \hline
		External Input & 3 & 4 & 6 \\ \hline
		External Output & 4 & 5 & 7 \\ \hline
		External Inquiry & 3 & 4 & 6 \\
	\end{tabulary}
\end{center}
These weights are then referred to as $w_{i,j}$ in the following sections, where $i$ stands for the level of complexity, and $j$ for the function type.
After this section comes the in-details description of each software function class, with the relevant partial sum indicated as $FP_{i}$, where $i$ is the abbreviation for the considered function class.
%
\subsubsection{Internal Logic File}
Internal Logic Files are homogeneous sets of data used and managed by the application.\\
Summary of the calculated weights:
\begin{equation*}
	\begin{aligned}
		&	FP_{ILF}
		& & = 9 \cdot w_{Simple,ILF} + 1 \cdot w_{Medium,ILF}\\
		&&& = 9 \cdot 7 + 1 \cdot 10\\
		&&& = 73\\
	\end{aligned}
\end{equation*}
In details:
\begin{itemize}
	\fp{Registered Passenger}{Personal Registered Passenger data. Infrequent modifications, but frequent insertions and readings.}{1}{w_{Simple,ILF}}
	\fp{Taxi Driver}{Taxi Driver data. Very infrequent modifications, infrequent insertions, frequent readings.}{1}{w_{Simple,ILF}}
	\fp{Administrator}{Administrator data. Very infrequent modifications, insertions and readings.}{1}{w_{Simple,ILF}}
	\fp{Taxi Car}{Taxi Car data. Very infrequent modifications, insertions and readings.}{1}{w_{Simple,ILF}}
	\fp{Taxi Request}{Taxi Request data. Frequent insertions and readings, infrequent modifications.}{1}{w_{Simple,ILF}}
	\fp{Taxi Reservation}{Taxi Reservation data. Frequent insertions and readings, infrequent modifications.}{1}{w_{Simple,ILF}}
	\fp{Problem}{Problem data. Infrequent insertions, readings and modifications.}{1}{w_{Simple,ILF}}
	\fp{Zone}{Zone data. Very infrequent insertions, readings and modifications.}{1}{w_{Simple,ILF}}
	\fp{GPS Data}{GPS Data, Database Server side. Very frequent insertions and readings, but very infrequent modifications.}{1}{w_{Medium,ILF}}
	\fp{Taxi Drivers Queue}{Taxi Drivers Queue data. Frequent insertions, readings and modifications.}{1}{w_{Simple,ILF}}
\end{itemize}
%
\subsubsection{External Interface File}
External Interface Files are homogeneous sets of data used by the application but generated and maintained by other applications.\\
Summary of the calculated weights:
\begin{equation*}
	\begin{aligned}
		&	FP_{ELF}
		& & = 2 \cdot w_{Simple,ELF} + 1 \cdot w_{Medium,ELF}\\
		&&& = 2 \cdot 5 + 1 \cdot 7\\
		&&& = 17\\
	\end{aligned}
\end{equation*}
In details:
\begin{itemize}
	\fp{Google Maps API}{Google Maps API related data about Travel Time and Addresses. Very frequent readings.}{2}{w_{Simple,ELF}}
	\fp{GPS Data}{GPS Data, Taxi Driver side. Very frequent insertions and readings, but very infrequent modifications.}{1}{w_{Medium,ELF}}
\end{itemize}
%
\subsubsection{External Input}
External Inputs are elementary operations to elaborate data coming from the external environment.\\
Summary of the calculated weights:
\begin{equation*}
	\begin{aligned}
		&	FP_{EI}
		& & = 4 \cdot w_{Simple,EI} + 3 \cdot w_{Medium,EI} + 1 \cdot w_{Complex,EI}\\
		&&& = 4 \cdot 3 + 3 \cdot 4 + 1 \cdot 6\\
		&&& = 30\\
	\end{aligned}
\end{equation*}
In details:
\begin{itemize}
	\fp{Login}{This operation requires a simple effort. In fact it has to perform few steps in order to conclude the procedure.}{1}{w_{Simple,EI}}
	\fp{Logout}{This operation requires a simple effort. In fact it has to perform few steps in order to conclude the procedure.}{1}{w_{Simple,EI}}
	\fp{Registration}{This operation requires a simple effort. In fact it has to perform few steps in order to conclude the procedure.}{1}{w_{Simple,EI}}
	\fp{Handle Personal Profile}{This operation requires a simple effort. In fact it has to perform few steps in order to conclude the procedure.}{1}{w_{Simple,EI}}
	\fp{Taxi Reservation}{This operation requires a medium effort. In fact it is very frequent.}{1}{w_{Medium,EI}}
	\fp{Taxi Request}{This operation requires a medium effort. In fact it is very frequent.}{1}{w_{Medium,EI}}
	\fp{Notify Problem}{This operation requires a strong effort. In fact it has to perform many elementary steps in order to handle the problem.}{1}{w_{Complex,EI}}
	\fp{End of Ride}{This operation requires a medium effort. In fact it is very frequent.}{1}{w_{Medium,EI}}
\end{itemize}
%
\subsubsection{External Output}
External Outputs are elementary operations that generate data for the external environment, and they usually include the elaboration of data from logic files.\\
Summary of the calculated weights:
	\begin{equation*}
		\begin{aligned}
		&	FP_{EO}
		& & = 9 \cdot w_{Simple,EO} + 1 \cdot w_{Complex,EO}\\
		&&& = 9 \cdot 4 + 1 \cdot 7\\
		&&& = 43\\
	\end{aligned}
\end{equation*}
In details:
\begin{itemize}
	\fp{Logout}{A redirection message needs to be sent to the user that logs out.}{1}{w_{Simple,EO}}
	\fp{Registration}{The result of this operation must be sent to the user that registers}{1}{w_{Simple,EO}}
	\fp{Login}{The result of the login must be sent to the specific user.}{1}{w_{Simple,EO}}
	\fp{Handle Personal Profile}{Informations about the result of the modification must be sent to the related user}{1}{w_{Simple,EO}}
	\fp{Taxi Reservation}{Notification about the result of the reservation must be sent to the specific passenger.}{1}{w_{Simple,EO}}
	\fp{Notify Problem}{Notification about the result of the request must be sent to the specific passenger.}{1}{w_{Simple,EO}}
	\begin{itemize}
		\fp{Automatic Taxi Request}{A new Taxi Request must be issued.}{1}{w_{Complex,EO}}
		\fp{Notification to the Passenger}{Notification about the new Taxi Request must be sent to the specific passenger.}{1}{w_{Simple,EO}}
		\fp{Notification to the Taxi Driver}{Notification about the result of the request must be sent to the specific Taxi Driver.}{1}{w_{Simple,EO}}
	\end{itemize}
	\fp{End of Ride}{Notification about the result of the operation must be sent to the relevant Taxi Driver.}{1}{w_{Simple,EO}}
\end{itemize}
%
\subsubsection{External Inquiry}
External Inquiries are elementary operations that involve input and output, without significant elaboration of data from logic files.\\
Summary of the calculated weights:
\begin{equation*}
	\begin{aligned}
		&	FP_{EIQ}
		& & = 2 \cdot w_{Simple,EIQ}\\
		&&& = 2 \cdot 3\\
		&&& = 6\\
	\end{aligned}
\end{equation*}
In details:
\begin{itemize}
	\fp{View Requests and Reservations}{In order to perform these operations the system has only to retrieve, send and render simple data.}{2}{w_{Simple,EIQ}}
\end{itemize}
%
\subsubsection{Summary}
All the calculated $FP_{i}$ sums up to $FP$, which is the total Function Points value:
\begin{equation*}
	\begin{aligned}
		&	FP
		& & = FP_{ILF} + FP_{ELF} + FP_{EI} + FP_{EO} + FP_{EIQ}\\
		&&& = 73 + 17 + 30 + 43 + 6\\
		&&& = 169\\	
	\end{aligned}
\end{equation*}
The total $FP$ value is then multiplied by a constant factor $k_{i,j}$ that depends on the programming language $i$ used to develop the software and the company gearing ratio $j$.\par
The gearing ratio is the level of a company's debt related to its equity capital, usually expressed in percentage form.
Gearing is a measure of a company's financial leverage and shows the extent to which its operations are funded by lenders versus shareholders.\par
This final calculation gives us the number of SLOC predicted for \myTaxiService{}:
\begin{equation*}
	\begin{aligned}
		&   n_{SLOC}
		& & = FP \cdot k_{Java, Avg}\\
		&&& = 169 \cdot 53\\
		&&& = 8957 \text{ SLOC}
	\end{aligned}
\end{equation*}
%
\subsection{COCOMO}
This estimation method takes into account a lot of parameters, such as final software, personnel and platform characteristics, combining them using a complex non-linear process.\par
In order to generate the \textbf{Constructive Cost Model} we decided to use an online \href{http://csse.usc.edu/tools/COCOMOII.php}{COCOMO II calculator}, using the \textbf{FP Sizing Method}.\par
We also used \href{http://csse.usc.edu/csse/research/COCOMOII/cocomo2000.0/CII_modelman2000.0.pdf}{COCOMO II - Model Definition Manual} to make better choices of the parameters we had to insert into the model.
\begin{itemize}
	\item \textbf{Software Size}
	\begin{itemize}
		\item \textbf{Unadjusted Function Points:} The value $FP$ has been taken as parameter, so \textbf{169} is the value chosen for this field.
		\item \textbf{Language:} The language of choice is \textbf{Java}, and not only Java EE, because the software is possibly a combination of different flavors of Java (Java SE, Java EE and Java for Android).
	\end{itemize}
	\item \textbf{Software Scale Drivers}
	\begin{itemize}
		\itemBold{Precedentedness} Since we had a previous experience using \textit{Java SE} for medium-size projects, but we had never used \textit{Java EE} for developing such a big application we decided to set this parameter to \textbf{Nominal}.
		\itemBold{Development Flexibility} Given that we had not strict specifications we set this parameter to \textbf{High}.
		\itemBold{Architecture/Risk Resolution} %TODO Completare
		\itemBold{Team Cohesion} After a few days dedicated to create a efficient and fast workspace, the cohesion reached an \textbf{Very High} level.
		\itemBold{Process Maturity} We understand, support and follow the process so we choose a \textbf{High} level for this parameter.
	\end{itemize}
	\item \textbf{Software Cost Drivers}
	\begin{itemize}
		\item \textbf{Product}
		\begin{itemize}
			\itemBold{Required Software Reliability}
			Given that a failure in the software system could lead to moderate problem we chose \textbf{Nominal} level.
			\itemBold{Data Base Size} Since we have a distributed application, the focus is on the lines of code instead of being on the size of the testing Database; so we chose a \textbf{Low} level parameter.
			\itemBold{Product Complexity} We made an average of the various complexity areas and we chose a \textbf{High} level parameter.
			\itemBold{Developed for Reusability} We decided to develop reusable system components, so we came up with an \textbf{High} level parameter.
			\itemBold{Documentation Match to Lifecycle Needs} The standard level of documentation is required, so the chosen level is \textbf{Nominal}.
		\end{itemize}
		\item \textbf{Personnel}
		\begin{itemize}
			\itemBold{Analyst Capability}
			\itemBold{Programmer Capability}
			\itemBold{Personnel Continuity}
			\itemBold{Application Experience}
			\itemBold{Platform Experience}
			\itemBold{Language and Toolset Experience}
		\end{itemize}
		\item \textbf{Platform}
		\begin{itemize}
			\itemBold{Time Constraint} We have no relevant time constraints, so we chose \textbf{Nominal} level for this parameter.
			\itemBold{Storage Constraint} We have no relevant storage constraint, so we chose \textbf{Nominal} level for this parameter. 
			\itemBold{Platform Volatility} Our hardware and software platforms do not change often, so we have no volatility and we chose a \textbf{Low} level for this constraint.
		\end{itemize}
		\item \textbf{Project}
		\begin{itemize}
			\itemBold{Use of Software Tools}
			\itemBold{Multisite Development}
			\itemBold{Required Development Schedule}
		\end{itemize}
	\end{itemize}
	\item \textbf{Maintenance}
	\item \textbf{Software Labor Rates}
	\begin{itemize}
		\itemBold{Cost per Person-Month (Dollars)}
	\end{itemize}	
\end{itemize}
%