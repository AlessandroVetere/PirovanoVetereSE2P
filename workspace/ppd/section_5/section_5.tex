\section{Project Risks}
%TODO Define the risks for the project, their relevance and the associated recovery actions.
%TODO No recruitment problems if we implement the application (like in the second semester). there are problems if we do not implement the application
We do not hire others to develop and test the project.
No recruitment problems.
%
\subsection{Risk Types}
%TODO risks types and examples
\begin{itemize}
	%They threaten the project plan. If they become real, it is likely that the project schedule will slip and that costs will increase
	\item Project Risks
	% They threaten the quality and timeliness of the software to be produced. If they become real, implementation may become difficult or impossible
	\item Technical Risks
	%They threaten the viability of the software to be built. If they become real, they jeopardize the project or the product.
	\item Business Risks
	\begin{itemize}
		%building an excellent product or system that no one really wants
		\itemBold{Market Risk}
		
		
		%building a product that no longer fits into the overall business strategy for the company
		\itemBold{Strategic Risk}
		
		
		%building a product that the sales force doesn't understand how to sell
		\itemBold{Sales Risk}
		\begin{enumerate}
			\risk{The project, once developed, comes out as too complex and not easily marketable.}{MEDIUM}{Ciao}{Ciao}
			\item The project's user interfaces are not user-friendly and not longer usable.\newline
			\textbf{probability} LOW
			\item The project, once developed, demonstrates useless and a complete misunderstanding of the requirements.\newline
			\textbf{probability} LOW
		\end{enumerate}
		
		
		%Losing the support of senior management due to a change in focus or a change in people
		\itemBold{Management Risk}
		\begin{enumerate}
			\item The company changes the management group and the new one decides to change the project's focus or the project's work force.\newline
			\textbf{probability} LOW
		\end{enumerate}
		
		
		%losing budgetary or personnel commitment
		\itemBold{Budget Risk}
		\begin{enumerate}
			\item The company decides to assign less money then promised for the development of \myTaxiService{}. \newline
			\textbf{probability} LOW
			\item One or both the developers of \myTaxiService{} decide to stop the development for an overlapping with other personal commitments.\newline
			\textbf{probability} LOW
		\end{enumerate}
	\end{itemize}
\end{itemize}
%
\subsection{Risk Management Strategies}
%TODO strategies to help manage risk examples
%Reactive vs. Proactive Risk Strategies

%Reactive risk strategies
%! "Don't worry, I'll think of something"
%! The majority of software teams and managers rely on this approach
%! Nothing is done about risks until something goes wrong
%! The team then flies into action in an attempt to correct the problem rapidly (fire fighting)
%! Crisis management is the choice of management techniques

%Proactive risk strategies
%! Steps for risk management are followed (see next slide)
%! Primary objective is to avoid risk and to have a contingency plan in place
%to handle unavoidable risks in a controlled and effective manner



%Steps for Risk Management
%1) Identify possible risks; recognize what can go wrong
%2) Analyze each risk to estimate the probability [L,M,H] that it will occur and the impact (i.e., damage) that it will do if it does occur
%3) Rank the risks by probability and impact - Impact may be negligible, marginal, critical, and catastrophic
%4) Develop a contingency plan to manage those risks having high probability and high impact
%