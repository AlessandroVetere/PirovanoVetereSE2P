\section{Project Tasks and Scheduling}
%TODO Identify the tasks for your project and their schedule. Do so retrospectively, assuming that the project has started in October 2015 as it really happened.
%
\subsection{Project Tasks}
%TODO Descrizione dei vari tasks
%TODO describe the number of hours spent for the various tasks (RASD, DD etc.)
Several tasks has been identified in our project.
The are summarized all together in the following table, which associates a label, a description and a completion state to each task:
\begin{center}
	\begin{tabulary}{\linewidth\tymin=70pt}{Y{2cm}|Y{6cm}|Y{2.25cm}}
		\textbf{Task} & \textbf{Description} & \textbf{Completed?}\\ \hline
		T1a & RASD - Writing & Yes \\ \hline
		T1b & RASD - Review & Yes \\ \hline
		T2a & DD - Writing & Yes \\ \hline
		T2b & DD - Review & Yes \\ \hline
		T3a & ITPD - Writing & Yes \\ \hline
		T3b & ITPD - Review & Yes \\ \hline
		T4a & PPD - Writing & Yes \\ \hline
		T4b & PPD - Review & No \\ \hline
		T5 & Implementation & No \\ \hline
		T6 & Unit Testing & No \\ \hline
		T7 & Integration Testing & No \\ \hline
		T8 & System Testing & No \\ \hline
		T9 & User Acceptance - Alpha Testing & No \\ \hline
		T10 & User Acceptance - Beta Testing & No \\ \hline
		T11 & Release To Market & No \\
	\end{tabulary}
\end{center}
%
\subsection{Tasks Scheduling}
%TODO Breve introduzione allo scheduling -> come deadline del progetto possiamo usare quella data da COCOMO II
\lipsum[100]
%
\subsubsection{Tasks, Durations and Dependencies}
%TODO Spiegazione della tabella
\lipsum[100]
\begin{center}
	\begin{tabulary}{\linewidth\tymin=70pt}{Y{2cm}|Y{2.5cm}|Y{2cm}|Y{2cm}}
		\textbf{Task} & \textbf{Effort\newline(person-days)} & \textbf{Duration\newline(days)} & \textbf{Dependencies} \\ \hline
		T1 & 15 & 10 & \\ \hline
		T2 & 153 & 123 & T1 \\
	\end{tabulary}
\end{center}
%
\subsubsection{Gantt Diagram}
%TODO 2) diagramma di gantt costruito dalla tabella
\lipsum[100]
\begin{center}
	\begin{ganttchart}[hgrid=true,vgrid={draw=none, dotted}, x unit=4mm]{1}{24}
		\gantttitle{2016}{24} \\
		\gantttitlelist{1,...,12}{2} \\
		\ganttmilestone{M1 Start  -}{0} \\
		\ganttbar{T1}{1}{2} \\
		\ganttbar{T2}{1}{6} \\
		\ganttmilestone{M2 End}{24}
	\end{ganttchart}
\end{center}
%