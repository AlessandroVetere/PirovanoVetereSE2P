\section{Project Tasks and Scheduling}
%TODO Identify the tasks for your project and their schedule. Do so retrospectively, assuming that the project has started in October 2015 as it really happened.
%
\subsection{Project Tasks}
%TODO Descrizione dei vari tasks
%TODO describe the number of hours spent for the various tasks (RASD, DD etc.)
Several tasks has been identified in our project.
The are summarized all together in the following table, which associates a label, a description and a completion state to each task:
\begin{center}
	\begin{tabulary}{\linewidth\tymin=70pt}{Y{2cm}|Y{6cm}|Y{2.25cm}}
		\textbf{Task} & \textbf{Description} & \textbf{Completed?}\\ \hline
		T1a & RASD - Writing & Yes \\ \hline
		T1b & RASD - Presentation & Yes \\ \hline
		T2a & DD - Writing & Yes \\ \hline
		T2b & DD - Presentation & Yes \\ \hline
		T3a & ITPD - Writing & Yes \\ \hline
		T3b & ITPD - Presentation & Yes \\ \hline
		T4a & PPD - Writing & Yes \\ \hline
		T4b & Final Presentation & No \\ \hline
		T5 & Implementation & No \\ \hline
		T6 & Unit Testing & No \\ \hline
		T7 & Integration Testing & No \\ \hline
		T8 & System Testing & No \\ \hline
		T9 & User Acceptance - Alpha Testing & No \\ \hline
		T10 & User Acceptance - Beta Testing & No \\ \hline
		T11 & Release To Market & No \\
	\end{tabulary}
\end{center}
%
\subsection{Tasks Scheduling}
In this section is solved the scheduling problem for the identified tasks.
We provide a table and a diagram in order to have the most effective visualization of the scheduling.
%
\subsubsection{Tasks, Durations and Dependencies}
%TODO Spiegazione della tabella
\lipsum[100]
\begin{center}
	\begin{tabulary}{\linewidth\tymin=70pt}{Y{1cm}|Y{2.5cm}|Y{4cm}|Y{2.5cm}}
		\textbf{Task} & \textbf{Effort [$people \cdot hours$]} & \textbf{Duration [$hours$]} & \textbf{Dependencies} \\ \hline
		T1a &  86 & $22[day] \cdot 4[hours/day] = \textbf{88[hours]}$  & \\ \hline
		T1b & 6 & $1[day] \cdot 6[hours/day] = \textbf{6[hours]}$] & T1a \\ \hline
		T2a & 74 & $22[days] \cdot 4[hours/day] = \textbf{88[hours]}$ & T1b \\ \hline
		T2b & 6 & $1[day] \cdot 6[hours/day] = \textbf{6[hours]}$ & T2a \\ \hline
		T3a & 30 & $14[days] \cdot 4[hours/day] = \textbf{56[hours]}$ & T2b \\ \hline
		T3b & 6 & $1[day] \cdot 6[hours/day] = \textbf{6[hours]}$ & T3a \\ \hline
		T4a & 32 & $10[days] \cdot 4[hours/day] = \textbf{40[hours]}$ & T3b \\ \hline
		T4b & 10 & $1[day] \cdot 10[hours/day] = \textbf{10[hours]}$ & T4a \\ \hline
		T5 & 480 & $80[days] \cdot 4[hours/day] = \textbf{320[hours]}$ & T4b \\ \hline
		T6 & 160 & $20[days] \cdot 4[hours/day] = \textbf{80[hours]}$ & T5 \\ \hline
		T7 & 140 & $20[days] \cdot 4[hours/day] = \textbf{80[hours]}$ & T6 \\ \hline
		T8 & 70 & $20[days] \cdot 4[hours/day] = \textbf{80[hours]}$ & T7 \\ \hline
		T9 & 80 & $20[days] \cdot 4[hours/day] = \textbf{80[hours]}$ & T8 \\ \hline
		T10 & 80 & $20[days] \cdot 4[hours/day] = \textbf{80[hours]}$ & T9 \\ \hline
		T11 & 10 & $20[days] \cdot 4[hours/day] = \textbf{80[hours]}$ & T10 \\ \hline
	\end{tabulary}
\end{center}
%
\subsubsection{Gantt Diagram}
%TODO 2) diagramma di gantt costruito dalla tabella
\lipsum[100]
\begin{center}
	\begin{ganttchart}[hgrid=true, vgrid={*3{white}, dotted}, x unit=1.7mm]{1}{60}
		\gantttitle{2015}{12} \gantttitle{2016}{48} \\
		\gantttitlelist{10,...,12}{4} \gantttitlelist{1,...,12}{4} \\
		\ganttmilestone{M1 Start}{0} \\
		\ganttbar{T1}{1}{2} \\
		\ganttbar{T2}{1}{6} \\
		\ganttmilestone{M2 End}{24} \\
		\ganttlink{elem0}{elem1} \\
		\ganttlink{elem2}{elem3}
	\end{ganttchart}
\end{center}
%