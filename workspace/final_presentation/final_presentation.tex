\documentclass{../common/latex_classes/pdf_presentation}
%
\usepackage{tikz}
\usetikzlibrary{shapes,arrows}
%
\newcommand{\showGeneralDiagram}[2]{\showPercentImage{../dd/section_2/general_diagram_dd.png}{#1}{#2}}
\newcommand{\showComponentDiagram}[2]{\showPercentImage{../dd/section_2/component_diagram/component_diagram.png}{#1}{#2}}
\newcommand{\showDeploymentDiagram}[2]{\showPercentImage{../dd/section_2/deployment_diagram/deployment_diagram.png}{#1}{#2}}
\newcommand{\showSequenceDiagram}[3]{\showPercentImage{../dd/section_2/sequence_diagrams/#1_sequence_diagram.png}{#2}{#3}}
\newcommand{\showRPMockupImage}[3]{\showPercentImage{../rasd/section_3/mockups/png_passenger/#1.png}{#2}{#3}}
\newcommand{\showTDMockupImage}[3]{\showPercentImage{../rasd/section_3/mockups/png_driver/#1.png}{#2}{#3}}
\newcommand{\showAdminMockupImage}[3]{\showPercentImage{../dd/section_4/mockups/administrator_mockups/#1.png}{#2}{#3}}
%
\title{Final Presentation}
%
\pdfinfo{
	/Author (Alberto Pirovano;Alessandro Vetere)
	/Title  (Software Engineering 2 - Final Presentation)
	%D:YYYYMMDDHHmmss
	/CreationDate (D:20160229090000)
	/Subject (Final Presentation)
	/Keywords (Final;Presentation;Software;Engineering)
}
%
\AtBeginSubsection[]
{
	\begin{frame}<beamer>{Outline}
		\tableofcontents[currentsection,currentsubsection]
	\end{frame}
}
%
\begin{document}
	\titleToc{}
	\section{Introduction}
	\begin{frame}{Assignment}
		The project is called \textbf{myTaxiService}, and is a complex software system to better manage and improve a preexisting taxi service in a town.\\
		In order to rationalize, clarify, and put in a \textbf{well structured document} all the revelant contents of that high level description, along with several additional contents of ours, a \textbf{RASD} was chosen as the best form of such a document.\\
	\end{frame}
	\begin{frame}{Tools used}
		We composed this document using some tools such as:
		\begin{enumerate}
			\item\textbf{TexStudio} for compiling \LaTeX{} document
			\item\textbf{StarUML} for drawing diagrams
			\item\textbf{Alloy Analizer 4.2} for checking model consistency
			\item\textbf{Balsamiq mockups 3.0} for building mockups
			\item\textbf{SourceTree} for allowing team collaboration
			\item\textbf{Github} for storing the project 
			\item\textbf{Asta Professional} for drwaing diagrams
			\item\textbf{Skype} for team collaboration
		\end{enumerate}		
	\end{frame}
	\section{RASD}
	\subsection{What, why and how?}
	\begin{frame}{What is a RASD?}
		\textbf{RASD} stands for \textbf{R}equirement \textbf{A}nalysis and \textbf{S}pecification \textbf{D}ocument, and is often mentioned as \textbf{SRS} (\textbf{S}oftware \textbf{R}equirements \textbf{S}pecification) in the literature.\\
		The standards that defines it, \textbf{IEEE Std 830-1998}, infact refers to that kind of document using the term \textbf{SRS}.\\
		Essentially, it is a document in which are analyzed and specified all the \textbf{requirements} of a software system that is going to be developed.
	\end{frame}
	\subsection{What does our RASD contain?}
	\begin{frame}{Contents overview}
		Our \textbf{RASD} consists in:
		\begin{enumerate}
			\item Title
			\item Table of Contents
			\item \textbf{3 Main Sections}
			s	\item 1 Appendix
		\end{enumerate}
		It was intended to be as much compilant with \textbf{IEEE Std 830-1998} as possible, but still there were deviations from the standard to better fit our project assignment (i.e. the inclusion of Alloy models...).
	\end{frame}
	\begin{frame}{Section 1 - Introduction}
		It is a general overview of the \textbf{RASD} document.\\
		Furthermore it specifically provides both a description of high-level software functionalities, and a set of information about the organization of the document.\\
		In particular, this section specifies the \textbf{goals} and the \textbf{actors} involved in the application.\\ 
		Moreover, it helps to better understand the used \textbf{acronyms}, \textbf{abbreviations}, and gives a lot of useful \textbf{definitions}.
	\end{frame}
	\begin{frame}{Section 2 - Overall Description}
		The aim of this section is describing the \textbf{product} that is going to be developed at a sufficient level of details.\\
		Moreover, it contains descriptions of all the \textbf{environmental} \textbf{elements} and \textbf{constraints} that are going to interact with the product during the development and once deployed.\\
		It serves also as a \textbf{background} for the description of software requirements, and helps in making them easier to understand for the public of the document (i.e. engineers, stakeholders…).
	\end{frame}
	\begin{frame}{Section 3 - Specific requirements}
		In this section are treated the \textbf{software requirements} at a level of detail sufficient for developers and engineers to create a software architecture that satisfies them, sufficient for testers to test those requirements and sufficient for stakeholders to have a general idea about how the finished product would work.\\
		In order to better explain how to write a software that complies with all the \textbf{requirements} mentioned, this section is enriched with several \textbf{UML diagrams}, \textbf{mockups} and \textbf{Alloy Models}. 
	\end{frame}
	\begin{frame}{Section 3 - UML Diagrams}
		In particular, this section is provided with a variety of diagrams: each type has a different purpose.\\
		\begin{itemize}
			\item\textbf{UML Use Case:} Show the \textbf{supported use cases} in relation with the \textbf{involved actors}.
			\item\textbf{UML Sequence Diagram:} Indicating, for each \textbf{use case} required, the \textbf{interaction} between the \textbf{actors involved} and the \textbf{system}.
			\item\textbf{UML State chart:} Explaining the \textbf{different states} in which:\\
			\begin{itemize}
				\item\textit{The Taxi driver (TD)} can be during the use of myTaxiService.
				\item\textit{The application} can be during the Registered passenger (RP) navigation flow.
			\end{itemize}
			\item\textbf{UML Class diagram:} Pointing out the different \textbf{software entities} involved in the application and the \textbf{relationships} between them.
		\end{itemize}
	\end{frame}
	\begin{frame}{Section 3 - Mockups}
		In addition, are displayed the most important screens of the three available different GUI.
		\begin{enumerate}
			\item \textbf{myTaxiService} web site for \textbf{Registered} and \textbf{Non Registered Passengers}.
			\item \textbf{myTaxiService} application for \textbf{Registered} and \textbf{Non Registered Passengers}.
			\item \textbf{myTaxiService} application for \textbf{Taxi Drivers}.
		\end{enumerate}
	\end{frame}
	\begin{frame}{Section 3 - Alloy Models}
		In order to meet the project assignment, we integrated our \textbf{RASD} with with some images of the models built with \textbf{Alloy Analyser 4.2} using the \textbf{Alloy} modeling language.\\
		The tool didn't find a proof of the inconsistency of our \textbf{models}, and that along with the generation and verification of interesting worlds, made us aware of the \textbf{consistency} of those models within a reasonable level of confidence.\\
		We integrated a screenshot for both of the generated worlds, one being simple and readable and the other being more “real”.
	\end{frame}
	\section{DD}
	\subsection{Architectural Design}
	\begin{frame}{Overview Diagram}
		\showGeneralDiagram{Overview Diagram}{1.0}
	\end{frame}
	\begin{frame}{High level components and their interaction}
		The system is composed of many \textbf{distributed} components: those will communicate with a \textbf{Client-Server} style and through \textbf{Point to Point} messaging system. 
		\begin{itemize}
			\item The \textbf{Client-Server} style is used to give the many Clients connected to the Server the opportunity of sending different requests (e.g. a \textbf{Taxi Ride Reservation} or \textbf{Taxi Ride Request}).
			\item The \textbf{Point to Point} bidirectional communication channel is made necessary to enable the Server the delivery of various messages and requests to the Clients:
			\begin{itemize}
				\item Generic notifications
				\item Service messages
				\item The request of serving a Taxi Ride (to a Taxi Driver)
				\item The request of an updating GPS Data (to a Taxi Driver)
			\end{itemize}
		\end{itemize}
	\end{frame}
	\begin{frame}{Layers}
		The selected software architecture follows the principles of the \textbf{Model View Controller} architectural pattern, therefore three main software components have been identified and those are:
		\begin{itemize}
			\item The \textbf{Model}
			\item The \textbf{View}
			\item The \textbf{Controller}
		\end{itemize}
		\textbf{Model}, \textbf{View} and \textbf{Controller} are then mapped to three different relevant software layers.
	\end{frame}
	\begin{frame}{Layers - View}
		This layer processes \textbf{Clients} commands, and converts them into requests addressed to the \textbf{Controller} layer.
		The \textbf{View} is connected to the \textbf{Controller} through a communication facility (e.g. The Internet).
		We imagined four different types of \textbf{View}, each one designed specifically to access \myTaxiService{} system in a different way and by a specific kind of user:
		\begin{itemize}
			\item Passenger Web View
			\item Passenger Application View
			\item Taxi Driver Application View
			\item Administrator View
		\end{itemize}
	\end{frame}
	\begin{frame}{Layers - Controller}
		This second Layer is split in two families of components with specialized functionalities:
		\begin{itemize}
			\item \textbf{Networking Components Family:}
			\begin{itemize}
				\item Groups the \textbf{Communication Components} that are involved in sending messages to the various Views, following the logic implements in the Business Components Family. 
				\item Dispatches a particular request to the relative \textbf{View}.
			\end{itemize}
			\item \textbf{Business Components Family:} In this family are included all the software components that implement the system logic.
			Their role is:
			\begin{itemize}
				\item Processing requests 
				\item Generating either \textbf{synchronous responses} (e.g registration or login procedure) or \textbf{asynchronous events} (e.g adding a \textbf{Taxi Ride} and sending a \textbf{Taxi Driver}).
			\end{itemize}
		\end{itemize}
	\end{frame}
	\begin{frame}{Layers - Model}
		The third and last Layer is the \textbf{Model}. It:
		\begin{itemize}
			\item Guarantees a high level interface to store and manage all the \myTaxiService{} relevant data.
			\item Abstracts a \textbf{Relational Database} in a software component that is in direct connection with the \textbf{Controller}
			%(Model Query Service in our UML Component Diagram proposed in a next section).
		\end{itemize}
		It has the responsibility of \textbf{receiving} and \textbf{handling} all the model updating needs of the \textbf{Business Components}. 
		%Moreover it has to guarantee \textbf{parallelism} and \textbf{high execution speed}.
	\end{frame}
	\begin{frame}{Tiers}
		The system is divided in \textbf{four} different tiers:
		\begin{itemize}
			\item \textbf{Clients:} The distributed clients of the application.
			\item \textbf{Web Server:} An outer server that dynamically generates web pages, receives requests, dispatches messages and contacts other servers.
			\item \textbf{Application Server:} The most important Tier of the system. Here are done all the logics and calculations that constitute the core part of \myTaxiService{}.
			\item \textbf{Database Server:} In this Tier it is hosted the Database that allows data persistence.
		\end{itemize}
	\end{frame}
	\begin{frame}{Component View}
		Several components has been designed to provide all the functionalities needed for \myTaxiService{} to work.
		\textbf{Five} mayor subsystems have been identified:
		\begin{itemize}
			\item Passenger View, Taxi Driver View, Administrator View
			\item Controller
			\item Model
		\end{itemize}
	\end{frame}
	\begin{frame}{Component View - Diagram}
		This diagram maps system \textbf{features into different software components}, and show \textbf{how these components interact} in order to \textbf{deliver the required functionalities}.
		It helps showing \textbf{Layers organization} and the \textbf{MVC implementation}.
		\showComponentDiagram{UML Component Diagram}{0.4}
	\end{frame}
	\begin{frame}{Deployment View}
		The best way found to \textbf{deploy} the software components identified, is to consider \textbf{7 different nodes} (8 if considering the Google Server contacted to use Google Maps API):
		\begin{itemize}
			\item Passenger Web Browser, Passenger Smartphone, Taxi Driver Smartphone, Administrator Workstation
			\item Web Server
			\item Application Server
			\item Database Server
		\end{itemize}
	\end{frame}			
	\begin{frame}{Deployment View - Diagram}
		The following diagram shows how \textbf{software components} are mapped into the \textbf{physical system}.
		\showDeploymentDiagram{UML Deployment Diagram}{0.3}
	\end{frame}
	\begin{frame}{Runtime View}
		In this subsection are proposed some of the most meaningful \textbf{UML Sequence Diagrams} with respect to show how software components interacts in order to deliver a specific functionality.
		The chosen functionalities are:
		\begin{itemize}
			\item Taxi Driver Registration (done by an Administrator)
			\item Handling of a Taxi Reservation (done by the Taxi Ride Manager)
		\end{itemize}
		There are other functionalities whose \textbf{UML Sequence Diagram} is not reported here for space and time constraints:
		\begin{itemize}
			\item Passenger Login
			\item Passenger Registration
			\item Queue Management
			\item Taxi Ride Request Handling
			\item Taxi Driver Report Problem
			\item Taxi Driver Position Update
		\end{itemize}
	\end{frame}
	\begin{frame}{Taxi Driver Registration}
		\showSequenceDiagram{admin_td_reg}{UML Sequence Diagram}{0.7}
	\end{frame}
	\begin{frame}{Handling of a Taxi Reservation}
		\showSequenceDiagram{tr_timer_expired}{UML Sequence Diagram}{0.9}
	\end{frame}
	\begin{frame}{Selected architectural styles and patterns - MVC}
		Several architectural styles and patterns were chosen in order to build \myTaxiService{} as a modern software.
		The main pattern that was recursively adopted is the \textbf{Model View Controller} architectural pattern:
		\begin{itemize}
			\item \textbf{System Level:} All the clients that use \myTaxiService{} (i.e. the Passengers, the Taxi Drivers, and the Administrator) are seen as Views, that following the Cocoa MVC pattern, are connected to a Controller, the Web Server, that through the Application Server is itself connected to the Model hosted on the Database Server.
			\item \textbf{Client Level}
			\item \textbf{Server Level}
			\begin{itemize}
				\item Web Server
				\item Application Server
				\item Database Server
			\end{itemize}
		\end{itemize}
	\end{frame}
	\begin{frame}{Selected architectural styles and patterns - Client-Server}
		The \textbf{Client-Server} style is used for all the requests done by the various clients connected to the Web Server of \myTaxiService{}.
		The \textbf{Taxi Driver Application} and the \textbf{Passenger Application} can use a standardized \textbf{Client-Server} protocol via \textbf{HTTPS} that follows the principle of a \textbf{RESTful Service}.
		The \textbf{Administrator} application its connected via \textbf{RPC} to the \textbf{Web Server} and can perform more critical requests, like the registration of a new \textbf{Taxi Driver} into the system.
		It is required that the \textbf{Administrator} application opens a \textbf{RPC} connection to the \textbf{Web Server} to start the communication.
	\end{frame}
	\begin{frame}{Selected architectural styles and patterns - Point to Point}
		A \textbf{Point to Point} bidirectional messaging system is established between the \textbf{Clients} and the \textbf{Web Server} at the \textit{boot} of the client application.
		The client should explicitly request a connection to the server that is listening for clients' connections.
		It is the connection over \textbf{Web Socket} protocol that allows the \textbf{Web Server} to send asynchronous messages and requests to which the client can respond using the same channel.
		The main reasons why this protocol is used are sending a \textbf{Taxi Ride} proposal to a given \textbf{Taxi Driver}, that can either accept or deny the proposal, and allowing the server to ask the \textbf{Taxi Driver} an updated geolocation data.
	\end{frame}
	\begin{frame}{Selected architectural styles and patterns - Conclusion}
		The \textbf{Client-Server} style and \textbf{Point to Point} bidirectional messaging system are used to implement properly the \textbf{MVC} pattern in this three \textbf{Layers}, four \textbf{Tiers} system.
	\end{frame}
	\begin{frame}{Other design decisions - HTTPS and Web Socket}
		Several technologies have been chosen in order to best fit the needs of the system to be.
		Not all the required functionalities of \myTaxiService{} are already mapped onto specific products because in those cases the choice done would matter less.
		But for the cases in which a technology has already been proposed, it is because a clear design decision was mandatory.
		As for the communication protocols between clients (excluded the Administrator client) and the server have been chosen:
		\begin{itemize}
			\itemBold{HTTPS} The secure version of \textbf{HTTP} was a mandatory choice as security and privacy concerns are of major importance nowadays.
			\itemBold{Web Socket} This innovative socket technology has been chosen although is relatively new because it implements a full duplex socket communication channel using web technology and therefore using the port 80, which is in almost every case not blocked by any firewall.
		\end{itemize}
	\end{frame}
	\begin{frame}{Other design decisions - Internet and Firewalls}
		For what concerns the network reachability has been chosen to make discoverable only the Web Server assigning it a public IP.
		All the other servers in \myTaxiService{} system are reachable only within the enterprise network.
		Between the \textbf{Web Server} and the external network is installed a firewall that controls all the incoming connections.
		In particular it must accept only incoming \textbf{HTTPS} connections, \textbf{Web Socket} connections and \textbf{RPC} connections.
		A firewall is also used to protected the \textbf{Database Server} from the \textbf{Application Server} in the unlikely case that the \textbf{Application Server} is attacked through the \textbf{Web Server} or the \textbf{Application Server} for some reasons stops working correctly and start behaving in a way that will damage the application \textbf{Model}.
	\end{frame}
	\subsection{Algorithm Design}
	\configureJava{}
	\begin{frame}{Queue Management}
		What will follow are slides containing algorithms (in form of Java methods, without loss of generality) that explain how the association of a Taxi Ride to an Available Taxi Driver is managed, and how that specific Taxi Driver is found.
	\end{frame}
	\begin{frame}{Queue Management - Manage Taxi Ride}	
		\lstinputlisting[basicstyle={\fontsize{5.5}{6.5}\ttfamily}, firstline=67, lastline=99]{../dd/section_3/java/src/mts/queue/QueueManager.java}
	\end{frame}
	\begin{frame}{Queue Management - Get Taxi Driver}
		\lstinputlisting[basicstyle={\tiny\ttfamily}, firstline=38, lastline=65]{../dd/section_3/java/src/mts/queue/QueueManager.java}
	\end{frame}
	\begin{frame}{Geolocation - A First Approach}
		Another interesting design choice that has been made concerns the way in which the \textbf{GPS coordinates} obtained from a given \textbf{Taxi Driver} are mapped into a specific \textbf{Zone}.
		It could have been possible, to do such a thing:
		\begin{enumerate}
			\item Obtain \textbf{GPS Data} via \textbf{Web Socket} from the selected \textbf{Taxi Driver}.
			\item Calculate the nearest \textbf{Address} of the given \textbf{GPS Data} using \textbf{Google Maps HTTPS API}.
			\item Query the \textbf{Model} to obtain the \textbf{Zone} to which belongs the given \textbf{Address}.
		\end{enumerate}
		But this solution requires to have a precomputed data structure that associates every \textbf{Address} in the \textbf{City} to the corresponding \textbf{Zone} (that could have been a relational table with as many rows inside as \textbf{Addresses} in the \textbf{City}, each address associated with the corresponding \textbf{Zone}), that is heavy to manage and maintain, although if correctly installed and filled, it gives for certain good performances.
	\end{frame}
	\begin{frame}{Geolocation - The Chosen Approach}
		A less heavy weight solution has been found: this solution expects every \textbf{Zone} of the \textbf{City} to be divided in several convex \textbf{Polygons}, for instance \textbf{Triangles}, that have interesting properties for our application.
		In \myTaxiService{}, \textbf{Zones} of regular shape are intended to be designed, and therefore the number of \textbf{Triangles} in which a \textbf{Zone} should be decomposed is very limited.
		So, such a flow is followed:
		\begin{enumerate}
			\item Obtain \textbf{GPS Data} via \textbf{Web Socket} from the selected \textbf{Taxi Driver}
			\item For each \textbf{Zone}, check if the the \textbf{Point} that the \textbf{Longitude} and \textbf{Latitude} from \textbf{GPS Data} identify is contained inside any \textbf{Triangle} in which the \textbf{Zone} is divided. If it is so, then the \textbf{Zone} is found. If that's not the case, then another \textbf{Zone} could contain the given Point. If no \textbf{Zone} contains the Point, then we can assume that the \textbf{Point} refers to \textbf{GPS Data} that identify a geographical point outside of the \textbf{City}.
		\end{enumerate}
		The computation of the \textbf{Point in Triangle} test is simple and efficient (e.g. using barycentric coordinates).
	\end{frame}
	\subsection{User Interface Design}
	\begin{frame}{GUI Design}
		In this section we provide the \textbf{most important and meaningful mockups} for every class of screens we have designed.
		In particular we identified \textbf{three classes} of graphical user interfaces:
		\begin{itemize}
			\itemBold {Passenger Mockups} both Web based and Mobile Application based.
			\itemBold {Taxi Driver Mockups} only Mobile Application based.
			\itemBold {Administrator Mockups} only Desktop Application based.
		\end{itemize}
	\end{frame}
	\begin{frame}{Passenger Mockups}
		In the following slides are shown \textbf{sequence of graphical states} that the application has to \textbf{render} in order to \textbf{create} and \textbf{handle} a Taxi Ride.
		Once logged in, the \textbf{Registered Passenger} will be redirected in his/her personal \textbf{Home page}, where he/she will be able to request or reserve a \textbf{Taxi Ride} and manage his/her personal profile.
	\end{frame}
	\begin{frame} {Passenger Mockups - Overview}
		Here the \textbf{Registered Passenger} can perform different actions:
		\begin{itemize}
			\item Request a Ride
			\item Reserve a Ride
			\item Logout
			\item Modify his profile
			\item Throw away a selected Ride
		\end{itemize}
		By clicking the \textbf{"Request a ride"} and \textbf{"Reserve a ride"} buttons the user is allowed to perform the relative actions.\\
		\textbf{Once requested a ride, the "Request a ride" button is disabled, in order to prevent multiple useless requests.}
	\end{frame}
	\begin{frame} {Passenger Mockups - Personal Homepage}
		This is an example of the \textbf{Registered Passenger}'s \textbf{Home page}. The screen is divided in two parts:
		\begin{itemize}
			\item The \textbf{left} one contains \textbf{Taxi Requests}
			\item The \textbf{right} one contains \textbf{Taxi Reservations}
		\end{itemize} 
		\showRPMockupImage{RegisteredPassengerHomePage}{Empty RP home page.}{0.65}
	\end{frame}
	\begin{frame} {Passenger Mockups - Request a Ride}
		\showRPMockupImage{RequestTaxi}{Taxi request.}{0.9}
	\end{frame}
	\begin{frame} {Passenger Mockups - Reserve a Ride}
		\showRPMockupImage{ReserveTaxi}{Taxi reservation.}{0.9}
	\end{frame}
	\begin{frame} {Passenger Mockups - Personal Homepage with \textbf{Taxi Rides}}
		Below is shown a common \textbf{state} with one \textbf{Taxi Request} active and one \textbf{Taxi Reservation} booked. 
		Through the trash icon the user is allowed to cancel a selected \textbf{Taxi Ride}.
		\showRPMockupImage{RegisteredPassengerHomePage2}{Populated RP home page.}{0.75}
	\end{frame}
	\begin{frame}{Taxi Driver Mockups}
		In the following slides are shown \textbf{the sequence of the graphical states} that the application has to \textbf{render} in order to make the \textbf{Taxi Driver} able to \textbf{handle} a \textbf{Taxi Ride}.
	\end{frame}
	\begin{frame} {Taxi Driver Mockups - Overview}
		The \textbf{Taxi Driver} personal screen is divided into \textbf{two} sections:
		\begin{itemize}
			\item Pending Rides Space
			\item Serving Ride Space
		\end{itemize}
		When the system sends a \textbf{Taxi Ride} to a specific \textbf{Taxi Driver}, it is placed in the \textbf{Pending Requests} space.
		Here the \textbf{Taxi Rider} can accept or deny it. 
		If it is accepted the \textbf{Taxi Request} is moved from the previous to the second space.
		Once the ride is finished, the \textbf{Taxi Driver} has to push the \textbf{Notify End Of Ride} button, in order to notify the system the ending of the given ride.
		Through the \textbf{Report Problem} button, the \textbf{Taxi Driver} has the possibility, \textit{in every moment of his/her working time}, to signal an accident or a problem.
		In order to better handle the problem, the \textbf{Taxi Driver} is asked to \textbf{signal} if the problem is solvable or not.
		\begin{itemize}
			\item If it is solvable, then the system \textbf{does not assign} a new \textbf{Taxi Driver}.
			\item If it is not solvable, then the system \textbf{assigns} the incomplete ride to the next \textbf{Taxi Driver} in the \textbf{Zone Queue}.
		\end{itemize}
	\end{frame}
	\begin{frame} {Taxi Driver Mockups - No Requests}
		\showTDMockupImage{TaxiDriverScreen3}{Taxi Driver Homepage without pending \textbf{Taxi Requests}.}{0.2}
	\end{frame}
	\begin{frame} {Taxi Driver Mockups - A Pending Request}
		\showTDMockupImage{TaxiDriverScreen1}{Taxi Driver Homepage with a \textbf{PENDING} \textbf{Taxi Request}.}{0.2}
	\end{frame}
	\begin{frame} {Taxi Driver Mockups - A Active Request}
		\showTDMockupImage{TaxiDriverScreen2}{Taxi Driver Homepage with an \textbf{ACTIVE} \textbf{Taxi Request}.}{0.2}
	\end{frame}
	\begin{frame}{Administrator Mockups}
		The system architecture does not admit the usage of a textual interface (e.g. a CLI). 
		For this reason we decided to \textbf{provide a thin desktop interface to the Administrator}.
		Thus, the Administrator can perform his actions using an intuitive, fast, and lightweight GUI.
		\showAdminMockupImage{admin_mockup_1}{Example of an Administrator screen.}{0.5}
	\end{frame}
	\section{ITPD}
	\begin{frame}{ITPD}
		This document is mainly based on the \textbf{Design Document}.In fact the purpose of the \textbf{Integration Test Plan Document} is to clearly state the order in which the software components identified in the \textbf{Component View} of the \textbf{DD} have to be integrated one with each other in order to guarantee a well tested final software.
	\end{frame}
	\begin{frame}{Entry Criteria}
		Before starting the integration testing of any software component that has been designed for \textbf{myTaxiService} system, the internal functions of the considered component must be unit tested using an appropriate framework.\\
		In particular every software component in the \textbf{Component View} section of the \textbf{Design Document} has to be integrated.\\
		Moreover we suppose that \textbf{Google Maps API} are well tested by \textbf{Google} and thus we can use them without testing any further.\\
		For what concerns the other external software elements, we assume that the \textbf{GPS Data Source} module in the \textbf{Taxi Driver View} uses the \textbf{GPS Drivers} of the underlying operating system that are already tested, and the same is assumed for the \textbf{Database Driver Adapter} in the \textbf{Model} referring to Database Drivers.
	\end{frame}
	\begin{frame}{Integration Testing Strategy}
		The \textbf{bottom-up integration testing approach} has been chosen, because for a medium sized project like \textbf{myTaxiService}, it is best to proceed step by step in a careful yet coherent integration strategy.\\
		The usage of the selected approach will forge a robust application with efforts concentrated in testing the \textbf{Server} parts before all.\\
	\end{frame}
	\begin{frame}{Convention adopted - Blocks}
		\begin{itemize}
			\itemBold{Yellow} This block is not dependent on any lower level component in \textbf{myTaxiService} and therefore it is integrated as a starting point in the current diagram.
			\itemBold{Blue} This block is going to be fully integrated on the top of its parents.
			\itemBold{Green} This block is not going to be fully integrated within the current diagram but needs further integration testing in subsequent diagrams.
			\itemBold{Red} This block represents a stub component, that replaces the real component mocking its functionalities.
		\end{itemize}
	\end{frame}
	\begin{frame}{Convention adopted - Arrows}
		It is a precedence symbol. \\
		It helps the tester to follow the right order in the whole integration process.\\
		It starts from a block and ends into another block. The block from which it starts is called parent and the other one child. In particular it means that the child block can be integrated only if its parents are already integrated. \\
		Moreover if a block is pointed by several arrows, its integration process can begin only when all the parent blocks are integrated.
	\end{frame}
	\begin{frame}
		%
		\begin{center}
			\begin{tikzpicture}[node distance=2cm, auto]
			% Place nodes
			\node [block, fill=yellow!20] (model) {Model};
			\node [block, right of=model, node distance=2.5cm] (controller) {Controller};
			\node [block, right of=controller, node distance=3cm] (passenger_view) {Passenger View};
			\node [block, above of=passenger_view] (taxi_driver_view) {Taxi Driver View};
			\node [block, below of=passenger_view] (administrator_view) {Administrator View};
			% Draw edges
			\path [line] (model) -- (controller);
			\path [line] (controller) -- (passenger_view);
			\path [line] (controller) |- (taxi_driver_view);
			\path [line] (controller) |- (administrator_view);
			\end{tikzpicture}
			\captionof{figure}{Software Integration Sequence Diagram}
		\end{center}
		%
	\end{frame}
	\begin{frame}
		%
		\begin{center}
			\begin{tikzpicture}[node distance=2cm, auto]
			% Place nodes
			\node [block, fill=yellow!20] (datbase_driver_adapter) {Database Driver Adapter};
			\node [block, right of=datbase_driver_adapter, node distance=5cm] (taxi_driver_db_adapter) {Taxi Driver DB Adapter};
			\node [block, above of=taxi_driver_db_adapter] (taxi_ride_db_adapter) {Taxi Ride DB Adapter};
			\node [block, above of=taxi_ride_db_adapter] (passenger_db_adapter) {Passenger DB Adapter};
			\node [block, below of=taxi_driver_db_adapter] (zone_db_adapter) {Zone DB Adapter};
			\node [block, below of=zone_db_adapter] (queue_db_adapter) {Queue DB Adapter};
			\node [block, right of=taxi_driver_db_adapter, node distance=5cm, fill=green!20] (model_query_service) {Model Query Service};
			% Draw edges
			\path [line] (datbase_driver_adapter) -- (taxi_driver_db_adapter);
			\path [line] (datbase_driver_adapter) |- (taxi_ride_db_adapter);
			\path [line] (datbase_driver_adapter) |- (passenger_db_adapter);
			\path [line] (datbase_driver_adapter) |- (zone_db_adapter);
			\path [line] (datbase_driver_adapter) |- (queue_db_adapter);
			\path [line] (taxi_driver_db_adapter) -- (model_query_service);
			\path [line] (taxi_ride_db_adapter) -| (model_query_service);
			\path [line] (passenger_db_adapter) -| (model_query_service);
			\path [line] (zone_db_adapter) -| (model_query_service);
			\path [line] (queue_db_adapter) -| (model_query_service);
			\end{tikzpicture}
			\captionof{figure}{Software Integration Sequence Diagram - Model}
		\end{center}
		%
	\end{frame}
	\begin{frame}
		%
		\begin{center}
			\begin{tikzpicture}[node distance=2cm, auto]
			% Place nodes
			\node [block, fill=green!20] (query_manager) {Query Manager};
			\node [block, above of=query_manager, fill=green!20] (model_query_service) {Model Query Service};
			\node [block, below of=query_manager] (taxi_sharing_manager) {Taxi Sharing Manager};
			\node [block, left of=taxi_sharing_manager, node distance=3.5cm, fill=green!20] (profile_manager) {Profile Manager};
			\node [block, left of=query_manager, node distance=3.5cm, fill=green!20] (session_manager) {Session Manager};
			\node [block, right of=taxi_sharing_manager, node distance=3.5cm] (location_manager) {Location Manager};
			\node [block, above of=location_manager, fill=red!20] (taxi_driver_view_stub) {TD Locator Stub};
			\node [block, below of=location_manager, fill=green!20] (taxi_ride_manager) {Taxi Ride Manager};
			\node [block, below of=taxi_sharing_manager, fill=green!20] (queue_manager) {Queue Manager};
			\node [block, left of=queue_manager, fill=red!20, node distance=3.5cm] (dispatcher_stub) {Dispatcher Stub};
			\node [block, below of=queue_manager, fill=green!20] (taxi_driver_manager) {Taxi Driver Manager};
			% Draw edges
			\path [line] (model_query_service) -- (query_manager);
			\path [line] (query_manager) -- (taxi_sharing_manager);
			\path [line] (query_manager) -- (profile_manager);
			\path [line] (query_manager) -- (session_manager);
			\path [line] (query_manager) -- (location_manager);
			\path [line] (taxi_driver_view_stub) -- (location_manager);
			\path [line] (location_manager) -- (queue_manager);
			\path [line] (dispatcher_stub) -- (queue_manager);
			\path [line] (dispatcher_stub) -- (taxi_driver_manager);
			\path [line] (location_manager) -- (taxi_ride_manager);
			\path [line] (queue_manager) -- (taxi_driver_manager);
			\end{tikzpicture}
			\captionof{figure}{Software Integration Sequence Diagram - Controller Business Components}
		\end{center}
		%
	\end{frame}
	\begin{frame}
		%
		\begin{center}
			\begin{tikzpicture}[node distance=2cm, auto]
			% Place nodes
			\node [block, fill=green!20] (dispatcher) {Dispatcher};
			\node [block, above of=dispatcher, fill=red!20] (td_view_stub) {TD Receiver Stub};
			\node [block, below of=dispatcher, fill=red!20] (passenger_view_stub) {PS Receiver Stub};
			\node [block, right of=td_view_stub, node distance=4cm, fill=green!20] (queue_manager) {Queue Manager};
			\node [block, below of=queue_manager, fill=green!20] (taxi_driver_manager) {Taxi Driver Manager};
			\node [block, below of=taxi_driver_manager, fill=green!20] (session_manager) {Session Manager};
			\node [block, right of=taxi_driver_manager, node distance=4cm, fill=green!20] (restful_service) {RESTful Service};
			\node [block, below of=restful_service, fill=green!20] (profile_manager) {Profile Manager};
			\node [block, above of=restful_service, fill=green!20] (taxi_ride_manager) {Taxi Ride Manager};		
			% Draw edges
			\path [line] (taxi_driver_manager) -- (restful_service);
			\path [line] (taxi_ride_manager) -- (restful_service);
			\path [line] (session_manager) -- (restful_service);
			\path [line] (profile_manager) -- (restful_service);
			\path [line] (dispatcher) -- (taxi_driver_manager);
			\path [line] (dispatcher) -- (queue_manager);
			\path [line] (td_view_stub) -- (dispatcher);
			\path [line] (passenger_view_stub) -- (dispatcher);
			\end{tikzpicture}
			\captionof{figure}{Software Integration Sequence Diagram - Controller Networking Components}
		\end{center}
		%
	\end{frame}
	\begin{frame}
		%
		\begin{center}
			\begin{tikzpicture}[node distance=2cm, auto]
			% Place nodes
			\node [block] (psreq) {PS Request Creator};
			\node [block, fill=green!20, left of=psreq, node distance=3.5cm] (restful_service) {RESTful Service};
			\node [block, right of=psreq, above of=psreq] (psw) {PS Web View};
			\node [block, right of=psreq, below of=psreq] (papp) {PS Application View};
			\node [block, below of=psw, right of=psw] (psr) {PS Receiver};
			\node [block, right of=psr, fill=green!20, node distance=2.5cm] (dp) {Dispatcher};
			% Draw edges
			\path [line] (restful_service) -- (psreq);
			\path [line] (psreq) |- (psw);
			\path [line] (psreq) |- (papp);
			\path [line] (psw) -| (psr);
			\path [line] (papp) -| (psr);
			\path [line] (psr) -- (dp);
			\end{tikzpicture}
			\captionof{figure}{Software Integration Sequence Diagram - Passenger View}
		\end{center}
		%
	\end{frame}
	\begin{frame}
		%
		\begin{center}
			\begin{tikzpicture}[node distance=4.5cm, auto]
			% Place nodes	
			\node [block] (treq) {TD Request Creator};
			\node [block, above of=treq, fill=green!20, node distance=2cm] (rs) {RESTful Service};
			\node [block, right of=treq] (tapp) {TD Application View};
			\node [block, right of=tapp] (tr) {TD Receiver};
			\node [block, below of=treq, node distance=2cm,, fill=yellow!20] (gps) {GPS Data Source};
			\node [block, right of=gps] (tl) {TD Locator};
			\node [block, below of=tr, fill=green!20, node distance=2cm] (dispatcher) {Dispatcher};
			% Draw edges
			\path [line] (rs) -- (treq);
			\path [line] (treq) -- (tapp);
			\path [line] (tapp) -- (tr);
			\path [line] (gps) -- (tl);
			\path [line] (tr) -- (dispatcher);
			\path [line] (tl) -- (dispatcher);
			\end{tikzpicture}
			\captionof{figure}{Software Integration Sequence Diagram - Taxi Driver View}
		\end{center}
		%
	\end{frame}
	\begin{frame}
		%
		\begin{center}
			\begin{tikzpicture}[node distance=2cm, auto]
			% Place nodes	
			\node [block, fill=green!20] (qm) {Query Manager};
			\node [block, right of=qm, node distance=4cm] (admw) {Administrator View};
			% Draw edges
			\path [line] (qm) -- (admw);
			\end{tikzpicture}
			\captionof{figure}{Software Integration Sequence Diagram - Administrator View}
		\end{center}
		%
	\end{frame}
	\section{PPD}
	\begin{frame}{PPD}
		PPD
	\end{frame}
\end{document}