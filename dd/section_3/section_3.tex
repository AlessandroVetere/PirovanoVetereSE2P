\section{Algorithm Design}
%TODO Quando Finisce la ride, al tassista viene indicato in che zona si trova e gli viene quindi ordinato di spostarsi fisicamente nell'unico punto di ritrovo presente nella sua zona (i punti di ritrovo sono unici per ogni zona) -> VA NEL RASD O NEL DD? SE NEL DD DOVE?

This section presents two of the most important algorithms used in \myTaxiService{} software system.
The algorithms are reported with a level of detail that is enough high to give software developer a predefined route to follow when developing the system.

%TODO Javadoc on interfaces
\subsection{Taxi Rides Serving Management}
In this subsection is presented some Java code that implements in a naive but meaningful way the algorithm that is used to handle the association of a Taxi Ride to a Taxi Driver.
\subsubsection{Queue Manager}
\inputJava{queue/QueueManager}
\subsubsection{Dispatcher}
\inputJava{queue/Dispatcher}
\subsubsection{Location Manager}
\inputJava{queue/LocationManager}
\subsubsection{Query Manager}
\inputJava{queue/QueryManager}
\subsubsection{Taxi Driver}
\inputJava{queue/TaxiDriver}
\subsubsection{Taxi Driver Status}
\inputJava{queue/TaxiDriverStatus}
\subsubsection{Taxi Ride}
\inputJava{queue/TaxiRide}
\subsubsection{Zone}
\inputJava{queue/Zone}

%TODO Javadoc and algorithm explanation
\subsection{Geolocation}
This subsection presents a naive implementation of the algorithm used to associate a given GPS Data to a Zone, or to show that the GPS Data is not contained in the City.
\subsubsection{Point}
\inputJava{zones/Point}
\subsubsection{Triangle}
\inputJava{zones/Triangle}
\subsubsection{Zone}
\inputJava{zones/Zone}
\subsubsection{GPS Data}
\inputJava{zones/GPSData}
\subsubsection{Zones}
\inputJava{zones/Zones}