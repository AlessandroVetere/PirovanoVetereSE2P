\section{Requirements Traceability}
%TODO Explain how the requirements you have defined in the RASD map into the design elements that you have defined in this document.
In this section are mapped all the functional requirements identified in the \textbf{RASD}, grouped by the \textbf{Use Case} they refer to.

\subsection{End of Ride}
\reqTracTab{
		FR35 & 
		%A taxi driver must be capable of notifying the end of a taxi ride he is serving.
		The Taxi Driver pushes a button on the TD Application View that calls the TD Request Creator, which sends a request to the RESTful Service. The request is then forwarded to the Taxi Ride Manager by the RESTful Service. The Taxi Ride Manager handles the operation. \\
		\hline
		FR36 &
		%The system must automatically re-enqueue the taxi driver that notifies the end of his current taxi ride in the queue of his current zone.
		When a Taxi Driver has notified the end of his current Ride, he is associated to a Available status. Then the Queue Manager is waiting for Taxi Drivers to become Available to enqueue them into their current Zone. \\
		\hline
		FR37 &
		%A notification must be sent from the system to the taxi driver to inform him about the result of his action.
		The response to the request made to achieve FR35 contains the result of the Taxi Driver action. \\
		\hline
		FR38 &
		%The system must peacefully handle a connection error and notify the user of the problem.
		The TD Request Creator and the RESTful Service are capable of handling connection error problem without making the relevant system crash. \\
		\hline
	}

\subsection{Login}
\reqTracTab{
		FR7 &
		%The system must check the correctness of the user's inserted data.
		The validation of \textbf{Passenger} inserted data is done by the \textbf{Session Manager} after checking the eventually duplication of the user session.
		\\
		\hline
		FR8 &
		%The system must correctly handle internally the login token assignment to the user.
		The \textbf{Session Manager} generates the \textbf{Login Token} once it is sure of the correctness of the login data. This token is delivered by the \textbf{Dispatcher}.
		\\
		\hline
		FR9 &
		%The system must notify the user about the result of his login.
		The notification of the result of the login procedure is produced in the \textbf{Session Manager} and is sent by the \textbf{Dispatcher}.
		\\
		\hline
		FR10 &
		%The system must peacefully handle a connection error and notify the user of the problem.
		The internal software architecture of the \textbf{Session Manager} is built in order to catch exceptions and to handle them in a proper way. 
		The \textbf{Point to Point} communication channel is used for delivering from the \textbf{Server} to the \textbf{Client} some notification about error's state.
		\\
		\hline
	}
	
\subsection{Logout}
\reqTracTab{
	FR11 &
	%The system must offer the possibility to logout to a given RP.
	The \textbf{Passenger} can perform the \textbf{Logout} procedure by interacting with the \textbf{Passenger Web or Application View}.
	\\
	FR12 &
	%After the logout, the system must redirect the RP to the \myTaxiService{} home page.
	All the logic about the navigation inside the application and in general about the building of the \textbf{View} is contained into the \textbf{Passenger Web or Application View}.It is defined through client side languages such as \textbf{Javascript} or \textbf{HTML} directives.
	\\
	FR13 &
	%The system must peacefully handle a connection error and notify the user of the problem.
	The internal software architecture of the \textbf{Passenger Web or Application View} is built in order to catch exceptions and to handle them in a proper way. 
	The \textbf{Point to Point} communication channel is used for delivering from the \textbf{Server} to the \textbf{Client} some notification about error's state.
	\\
}