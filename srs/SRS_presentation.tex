\documentclass{../common/latex_classes/pdf_presentation}

\title{RASD for \myTaxiService{}}

\subject{A brief presentation of myTaxiService RASD}

\begin{document}
	
	\titleToc{}

	\section{What, why and how?}
	
	\subsection{What is a RASD?}
	
	\begin{frame}{What is a RASD?}
		\textbf{RASD} stands for \textbf{R}equirement \textbf{A}nalysis and \textbf{S}pecification \textbf{D}ocument, and is often mentioned as \textbf{SRS} (\textbf{S}oftware \textbf{R}equirements \textbf{S}pecification) in the literature.\\
		The standards that defines it, \textbf{IEEE Std 830-1998}, infact refers to that kind of document using the term \textbf{SRS}.\\
		Essentially, it is a document in which are analyzed and specified all the \textbf{requirements} of a software system that is going to be developed.
	\end{frame}
	
	\subsection{Why did we write a RASD?}
	
	\begin{frame}{Why did we write a RASD?}
		A project was assigned to us in form of a high level description from our professor, \textbf{Raffaella Mirandola}.\\
		The project is called \textbf{myTaxiService}, and is a complex software system to better manage and improve a preexisting taxi service in a town.\\
		In order to rationalize, clarify, and put in a \textbf{well structured document} all the revelant contents of that high level description, along with several additional contents of ours, a \textbf{RASD} was chosen as the best form of such a document.\\
	\end{frame}
	
	\subsection{How did we write it?}
	
	\begin{frame}{How did we write it?}
		We composed this document using some tools such as:
		\begin{enumerate}
			\item\textbf{TexStudio} for compiling \LaTeX{} document
			\item\textbf{StarUML} for drawing diagrams
			\item\textbf{Alloy Analizer 4.2} for checking model consistency
			\item\textbf{Balsamiq mockups 3.0} for building mockups
			\item\textbf{SourceTree} for allowing team collaboration
			\item\textbf{Github} for storing the project 
		\end{enumerate}		
	\end{frame}
	
	\section{What does our RASD contain?}
	
	\subsection{Contents overview}
	
	\begin{frame}{Contents overview}
		Our \textbf{RASD} consists in:
		\begin{enumerate}
			\item Title
			\item Table of Contents
			\item \textbf{3 Main Sections}
		s	\item 1 Appendix
		\end{enumerate}
		It was intended to be as much compilant with \textbf{IEEE Std 830-1998} as possible, but still there were deviations from the standard to better fit our project assignment (i.e. the inclusion of Alloy models...).
	\end{frame}
	
	\subsection{Section 1}
	
	\begin{frame}{Section 1 - Introduction}
		It is a general overview of the \textbf{RASD} document.\\
		Furthermore it specifically provides both a description of high-level software functionalities, and a set of information about the organization of the document.\\
		In particular, this section specifies the \textbf{goals} and the \textbf{actors} involved in the application.\\ 
		Moreover, it helps to better understand the used \textbf{acronyms}, \textbf{abbreviations}, and gives a lot of useful \textbf{definitions}.
	\end{frame}
	
	\subsection{Section 2}
	
	\begin{frame}{Section 2 - Overall Description}
		The aim of this section is describing the \textbf{product} that is going to be developed at a sufficient level of details.\\
		Moreover, it contains descriptions of all the \textbf{environmental} \textbf{elements} and \textbf{constraints} that are going to interact with the product during the development and once deployed.\\
		It serves also as a \textbf{background} for the description of software requirements, and helps in making them easier to understand for the public of the document (i.e. engineers, stakeholders…).
	\end{frame}
	
	\subsection{Section 3}
	
	\begin{frame}{Section 3 - Specific requirements}
		In this section are treated the \textbf{software requirements} at a level of detail sufficient for developers and engineers to create a software architecture that satisfies them, sufficient for testers to test those requirements and sufficient for stakeholders to have a general idea about how the finished product would work.\\
		In order to better explain how to write a software that complies with all the \textbf{requirements} mentioned, this section is enriched with several \textbf{UML diagrams}, \textbf{mockups} and \textbf{Alloy Models}. 
	\end{frame}
	
	\begin{frame}{Section 3 - UML Diagrams}
		In particular, this section is provided with a variety of diagrams: each type has a different purpose.\\
		\begin{itemize}
			\item\textbf{UML Use Case:} Show the \textbf{supported use cases} in relation with the \textbf{involved actors}.
			\item\textbf{UML Sequence Diagram:} Indicating, for each \textbf{use case} required, the \textbf{interaction} between the \textbf{actors involved} and the \textbf{system}.
			\item\textbf{UML State chart:} Explaining the \textbf{different states} in which:\\
											\begin{itemize}
												\item\textit{The Taxi driver (TD)} can be during the use of myTaxiService.
												\item\textit{The application} can be during the Registered passenger (RP) navigation flow.
											\end{itemize}
			\item\textbf{UML Class diagram:} Pointing out the different \textbf{software entities} involved in the application and the \textbf{relationships} between them.
		\end{itemize}
	\end{frame}
	
	\begin{frame}{Section 3 - Mockups}
		In addition, are displayed the most important screens of the three available different GUI.
		\begin{enumerate}
			\item \textbf{myTaxiService} web site for \textbf{Registered} and \textbf{Non Registered Passengers}.
			\item \textbf{myTaxiService} application for \textbf{Registered} and \textbf{Non Registered Passengers}.
			\item \textbf{myTaxiService} application for \textbf{Taxi Drivers}.
		\end{enumerate}
	\end{frame}
	
	\begin{frame}{Section 3 - Alloy Models}
		In order to meet the project assignment, we integrated our \textbf{RASD} with with some images of the models built with \textbf{Alloy Analyser 4.2} using the \textbf{Alloy} modeling language.\\
		The tool didn't find a proof of the inconsistency of our \textbf{models}, and that along with the generation and verification of interesting worlds, made us aware of the \textbf{consistency} of those models within a reasonable level of confidence.\\
		We integrated a screenshot for both of the generated worlds, one being simple and readable and the other being more “real”.
	\end{frame}
	
\end{document}