\section{Introduction}	% Produces section heading.  Lower-level sections are begun with similar \subsection and \subsubsection commands.
What follows is an introduction to \myTaxiService{} RASD document.
\subsection{Purpose}
The purpose of this document is to serve as a guide to software designers, developers and testers who are responsible for the engineering of the \myTaxiService{} project.\par
Through UML use cases and diagrams, it will show the interaction between the different agents involved in the application, to better explain the environment in which the software will work when installed in the real world: this should give the engineers all the necessary informations to design, develop and test the software.
In addition, it is an important tool for stakeholders to check whether the product matches their initial demands, and to serve as a further contractual basis.
\subsection{Scope}
This document provides an overall description of \myTaxiService{} software system, showing its functional and non-functional requirements, constraints and external interfaces.\par
It consists of natural language descriptions, UML diagrams and Alloy models of the different features given.
\subsubsection{Actors}
Below are listed the two main actors that will interact with the application once deployed:
\begin{itemize}
	\itemBold{Taxi driver}	Owner of a vehicle who is given the permission to provide the service.
	\itemBold{Non registered passenger} A person that needs to move from a position to another one among the city and wants to use \myTaxiService{} in order to do so, but has not registered yet to the service.
	\itemBold{Registered passenger} A formerly non registered passenger that has registered to \myTaxiService{}.
\end{itemize}
\subsubsection{Goals}
This new service pretends to achieve various goals, the main ones being:
\begin{itemize}
	\itemBold{G1} A non registered passenger can only register once to the service.
	\itemBold{G2} A registered passenger is registered to the service.
	\itemBold{G3} A registered passenger can login to the service only when not logged in.
	\itemBold{G4} A registered passenger can logout from the service only when logged in.
	\itemBold{G5} A registered passenger can request a taxi ride only when logged to the service.
	\itemBold{G6} A registered passenger can reserve a taxi ride only when logged to the service.
	\itemBold{G7} A registered passenger can view his taxi requests when logged to the service.
	\itemBold{G8} A registered passenger can cancel a taxi request only if he is viewing it.
	\itemBold{G9} A registered passenger can view his taxi reservations when logged to the service.
	\itemBold{G10} A registered passenger can cancel a taxi reservation only if he is viewing it.
	\itemBold{G11} A registered passenger can handle his profile only when logged to the service.
	\itemBold{G12} A taxi driver is registered to the service.
	\itemBold{G13} A taxi driver can login to the service only when not logged in.
	\itemBold{G14} A taxi driver can logout to the service only when not logged in.
	\itemBold{G15} A taxi driver can accept to give a ride to a registered user that requested one.
	\itemBold{G16} A taxi driver can deny to give a ride to a user that requested one.
	\itemBold{G17} A taxi driver can notify the end of a ride.
	\itemBold{G18} A taxi driver that denies availability for a ride is put last in his current zone waiting queue.
	\itemBold{G19} A registered passenger can only take a ride from a taxi driver who is first in his current zone waiting queue.
	\itemBold{G20} A taxi driver belongs to at maximum one waiting queue at a time.
	\itemBold{G21} Both a non registered passenger and a registered passenger can use either a web application or a mobile application to access the service.
	\itemBold{G22} Further services can be built on the top of the existing one through a set of given APIs.
\end{itemize}
\subsection{Definitions, acronyms, and abbreviations}
Here are listed some of the necessary definitions, acronyms and abbreviations to make better understandable the content of the document:
\subsubsection{Acronyms}
As commonly used, some acronyms are included in the document and here explained:
\begin{itemize}
	\itemBold{IEEE} Institute of Electrical and Electronics Engineers.
	\itemBold{SRS} Software Requirements Specification, document based on IEEE 830.
	\itemBold{RASD} Requirements Analysis and Specification Document, also known as SRS.
	\itemBold{UML} Unified Modelling Language.
	\itemBold{Wi-Fi} Wireless Fidelity, technology for local area wireless computer networking technology based on IEEE 802.11 standard.
	\itemBold{WiMAX} Worldwide Interoperability for Microwave Access, technology based on IEEE 802.16 standard.
	\itemBold{API} Application Programming Interface, a set of public accessible functions, protocols and tools provided by a specific application for building software applications.
	\itemBold{GPS} Global Positioning System.
	\itemBold{DBMS} DataBase Management System.
	\itemBold{SDK} Software Development Toolkit.
	\itemBold{URI} Uniform Resource Identifier, a string of characters used to identify the name of a resource.
	\itemBold{URL} Uniform Resource Locator, a specific type of URI that reference to a web resource that specifies its location on a computer network and a mechanism for retrieving it.
	\itemBold{HTTP} The Hypertext Transfer Protocol is an application protocol for distributed, collaborative, hypermedia information systems; HTTP is the foundation of data communication for the World Wide Web.
	\itemBold{HTTPS} Communication protocol over HTTP within a connection encrypted by Transport Layer Security.
	\itemBold{GUI} Graphical User Interface.
	\itemBold{LAN} Local Area Network.
	\itemBold{MAN} Metropolitan Area Network.
\end{itemize}
\subsubsection{Definitions}
What follows is a list of definitions of some terms used in the document:
\begin{itemize}
	\itemBold{Ethernet} Family of computer networking technologies for LANs and MANs, first standardized in 1983 as IEEE 802.3.
	\itemBold{Bluetooth} Wireless technology standard for exchanging data over short distances, once based on IEEE 802.15.1 standard.
	\itemBold{Internet} The Internet is the global system of interconnected computer networks that use the Internet protocol suite (TCP/IP) to link billions of devices worldwide.
	\itemBold{Service} The functionalities offered by \myTaxiService{} via software applications to passengers and taxi drivers.
	\itemBold{System} The whole hardware and software parts that together deliver \myTaxiService{} as a service to passengers and taxi drivers.
	\itemBold{Server} The centralized part of the system installed in a dedicated facility, that communicates with clients via a network.
	\itemBold{Client} A part of the system that is not the server, and communicates with the server via network to access a desired service.
	\itemBold{Smartphone} A mobile phone with an advanced mobile operating systems, which combines features of a personal computer operating system with other features useful for mobile or handheld use.
	\itemBold{Application} A computer program (i.e. a piece of software) that performs a group of coordinated functions, tasks, or activities for the benefit of the user.
	\itemBold{Smartphone Application} Also referred as \textbf{mobile application}, is an application designed to run on smartphones.
	\itemBold{World Wide Web} Also referred as \textbf{web}, is an information space where documents and other web resources are identified by URLs, interlinked by hypertext links, and can be accessed via the Internet.
	\itemBold{Web Browser} Software application for retrieving, presenting, and traversing information resources on the World Wide Web.
	\itemBold{Web Page} A web document that is suitable for the World Wide Web and the web browser.
	\itemBold{Web Server} An information technology that processes requests via HTTP and stores, processes and deliver web pages to clients.
	\itemBold{Web Application} A client-server software application in which the client (or user interface) runs in a web browser.
\end{itemize}
\subsubsection{Abbreviations}
For the sake of brevity, some abbreviations are adopted:
\begin{itemize}
	\itemBold{RP} Registered passenger.
	\itemBold{TD} Taxi driver.
	\itemBold{PTRS} Pending Taxi Ride Set.
	\itemBold{TRQ} Taxi Request.
	\itemBold{TRS} Taxi Reservation.
\end{itemize}	
\subsection{Overview}
This document is structured in 3 main parts:
\begin{itemize}
	\itemBold{Section 1} Overview of the RASD document. Specifically provides both a description of high-level software functionalities, and a set of information about the organization of the document.
	\itemBold{Section 2} Provides a high-level description of the software requirements, not describing specifically the main aspect of them, but only providing a background of those requirements. The main purpose of this section is make the requirements easy to understand.
	\itemBold{Section 3} Shows all of software requirements to a level of detail sufficient to enable designers to design a system to satisfy those requirements, and testers to test the system requirements.
\end{itemize}
\subsection{References}
This section contains references to external sources that have been taken into account while writing this document.
\begin{itemize}
	\item Specification Document: Software Engineering 2 Project, AA 2015-2016, Assignment 1 and 2
	\item IEEE Std 830-1998 IEEE Recommended Practice for Software Requirements Specifications
	\item Applied Software Project Management Andrew Stellman \& Jennifer Greene
	\item Wikipedia, the free encyclopedia
\end{itemize}