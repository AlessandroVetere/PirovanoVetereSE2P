\section{Introduction}      % Produces section heading.  Lower-level
                             % sections are begun with similar
                             % \subsection and \subsubsection commands.
TODO: Introduction to \myTaxiService{} SRS document.
\subsection{Purpose}
The purpose of this document is to serve as a guide to software designers, developers and testers who are responsible for the engineering of the \myTaxiService{} project.
It should give the engineers all of the information necessary to design, develop and test the software.
In addition, it is an important tool for \myTaxiService{} stakeholders to check whether the product matches their initial demands, and to serve as a further contractual basis.
\subsection{Scope}
This document provides an overall description of \myTaxiService{} software system, showing its functional and non-functional requirements, constraints and external interfaces.
It consists of natural language descriptions, UML diagrams and Alloy models of the different scenarios affected.
\subsubsection{Actors}
\begin{itemize}
	\item Taxi driver: all the owners of a vehicle that have given the availability join the service.
	\item User: all the people that need to move from a position to another one around the city.
\end{itemize}
\subsubsection{Goals}
\begin{itemize}
	\item G1 A user can request a taxi ride
	\item [G2] A user can reserve a taxi ride
	\item [G3] A taxi driver can confirm it's availability to take a call
	\item [G4] A taxi driver can deny it's availability to take a call
	\item [G5] A user can register to the service
	\item [G6] A user can login to the service
	\item [G7] A taxi driver that denies availability is put last in his current zone waiting queue
	\item [G8] A user can only take a ride from a taxi driver who is first in his current zone queue
	\item [G9] A taxi driver belongs to at maximum one taxi queue at a time
	\item [G10] A user can use either a web application or a mobile application to access the service
	\item [G11] Further services can be built on the top of the existing one
\end{itemize}
\subsection{Definitions, acronyms, and abbreviations}
\begin{itemize}
	\item SRS: Software Requirements Specification
	\item UML: Unified Modelling Language
	\item IEEE: Institute of Electrical and Electronics Engineers
\end{itemize}
\subsection{Overview}
This document is structured in 3 main parts:
\begin{itemize}
	\item Section 1: Overview of the SRS document. Specifically provides both a description of high-level software functionalities, and a set of information about the organization of the document.
	\item Section 2: Provides a high-level description of the software requirements, not describing specifically the main aspect of them, but only providing a background of those requirements. The main purpose of this section is make the requirements easy to understand.
	\item Section 3: Shows all of software requirements to a level of detail sufficient to enable designers to design a system to satisfy those requirements, and testers to test the system requirements.
\end{itemize}
\subsection{References}
This section contains references to external sources that have been taken into account while writing this document.
\begin{itemize}
   \item Specification Document: Software Engineering 2 Project, AA 2015-2016, Assignment 1 and 2
   \item IEEE Std 830-1998 IEEE Recommended Practice for Software Requirements Specifications
   \item Applied Software Project Management Andrew Stellman \& Jennifer Greene
\end{itemize}