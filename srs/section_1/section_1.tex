%TODO Add /par where needed
\section{Introduction}      % Produces section heading.  Lower-level sections are begun with similar \subsection and \subsubsection commands.
What follows is an introduction to \myTaxiService{} SRS document.
\subsection{Purpose}
The purpose of this document is to serve as a guide to software designers, developers and testers who are responsible for the engineering of the \myTaxiService{} project.
Moreover, through UML use cases and diagrams, it will show the interaction between the different agents involved in the application, in order to better explain the environment in which the software will work when installed in the real world.
This document should give the engineers all the necessary informations to design, develop and test the software.
In addition, it is an important tool for stakeholders to check whether the product matches their initial demands, and to serve as a further contractual basis.
\subsection{Scope}
This document provides an overall description of \myTaxiService{} software system, showing its functional and non-functional requirements, constraints and external interfaces.
It consists of natural language descriptions, UML diagrams and Alloy models of the different features given.
\subsubsection{Actors}
Below are listed the two main actors that will interact with the application once deployed:
\begin{itemize}
	%TODO Check actors
	\item\textbf{Taxi driver:}	Owner of a vehicle who is given the permission to provide the service.
	\item\textbf{Non registered passenger:} A person that needs to move from a position to another one among the city and wants to use \myTaxiService{} in order to do so, but has not registered yet to the service.
	\item\textbf{Registered passenger:} A former non registered passenger that has registered to \myTaxiService{}.
\end{itemize}
\subsubsection{Goals}
This new service will achieve various goals, the main ones being:
\begin{itemize}
	%TODO Check goals
	\item\textbf{G1:} A non registered passenger can only register to the service.
	\item\textbf{G2:} A registered passenger can request a taxi ride only when logged to the service.
	\item\textbf{G3:} A registered passenger can reserve a taxi ride only when logged to the service.
	\item\textbf{G4:} A taxi driver can accept to give a ride to a registered user that requested one.
	\item\textbf{G5:} A taxi driver can deny to give a ride to a user that requested one.
	\item\textbf{G6:} A registered passenger can login to the service.
	\item\textbf{G7:} A taxi driver that denies availability is put last in his current zone waiting queue.
	\item\textbf{G8:} A user can only take a ride from a taxi driver who is first in his current zone waiting queue.
	\item\textbf{G9:} A taxi driver belongs to at maximum one waiting queue at a time.
	\item\textbf{G10:} A non registered passenger or a registered passenger can use either a web application or a mobile application to access the service.
	\item\textbf{G11:} Further services can be built on the top of the existing one.
\end{itemize}
\subsection{Definitions, acronyms, and abbreviations}
\begin{itemize}
	\item \textbf{SRS:} Software Requirements Specification, document based on IEEE 830.
	\item\textbf{UML:} Unified Modelling Language.
	\item\textbf{IEEE:} Institute of Electrical and Electronics Engineers.
	\item\textbf{Wi-Fi:} Wireless Fidelity, technology for local area wireless computer networking technology based on IEEE 802.11 standard.
	\item\textbf{WiMAX:} Worldwide Interoperability for Microwave Access, technology based on IEEE 802.16 standard.
	\item\textbf{Bluetooth:} Wireless technology standard for exchanging data over short distances, once based on IEEE 802.15.1 standard.
	%TODO Complete
	\item\textbf{Java:}
	\item\textbf{Android:}
	\item\textbf{iOs:}
	%TODO Add more definitions
\end{itemize}
\subsection{Overview}
This document is structured in 3 main parts:
\begin{itemize}
	\item\textbf{Section 1:} Overview of the SRS document. Specifically provides both a description of high-level software functionalities, and a set of information about the organization of the document.
	\item\textbf{Section 2:} Provides a high-level description of the software requirements, not describing specifically the main aspect of them, but only providing a background of those requirements. The main purpose of this section is make the requirements easy to understand.
	\item\textbf{Section 3:} Shows all of software requirements to a level of detail sufficient to enable designers to design a system to satisfy those requirements, and testers to test the system requirements.
\end{itemize}
\subsection{References}
This section contains references to external sources that have been taken into account while writing this document.
\begin{itemize}
	\item Specification Document: Software Engineering 2 Project, AA 2015-2016, Assignment 1 and 2
	\item IEEE Std 830-1998 IEEE Recommended Practice for Software Requirements Specifications
	\item Applied Software Project Management Andrew Stellman \& Jennifer Greene
\end{itemize}