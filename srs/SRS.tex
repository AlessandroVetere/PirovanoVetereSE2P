\documentclass{article}      % Specifies the document class
                             % The preamble begins here.
%\newcommand{\newCommandName}[parametersNumber]{a_string_that_replaces_the_"\newCommandName{params}"}
%Refer to params as #1, #2 etc.
\newcommand{\myTaxiService}[0]{\mbox{\emph{myTaxiService}}}
\newcommand{\setDepth}[1]{\setcounter{tocdepth}{#1}\setcounter{secnumdepth}{#1}}
\newcommand{\sectionBaseDir}[1]{./section_#1} %\sectionBaseDir{1} ./section_1
\newcommand{\sectionTexPath}[1]{\sectionBaseDir{#1}/section_#1.tex}
\newcommand{\inputSection}[1]{\newpage\input{\sectionTexPath{#1}}}
\setDepth{4}
\title{Software Requirements Specifications}    % Declares the document's title.
\author{Alberto Pirovano \and Alessandro Vetere}      % Declares authors' names.
\date{\today}	% This command produces today's date.
\begin{document}	% End of preamble and beginning of text.
\maketitle
%TODO Make a meaningful abstract
\begin{abstract} % This is the start of the "abstract" environment.
	% The abstract environment is inteded to contain a summary of contents.
	Software Requirements Specifications of \myTaxiService{} project.
\end{abstract} % Here ends the "abstract" environment. 
\newpage % Starts a new page
\tableofcontents             % Prints the TOC
\inputSection{1}
\inputSection{2}
\inputSection{3}
\end{document}               % End of document.