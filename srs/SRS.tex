\documentclass{../common/latex_classes/pdf_document}      % Specifies the document class
                             % The preamble begins here.

\newcommand{\inputSystemFeature}[1]{\input{\sectionBaseDir{3}/system_features/#1.tex}}
\newcommand{\showGeneralDiagram}{\showImage{\sectionBaseDir{2}/general_diagram.png}{General Diagram}}
\newcommand{\showUmlImage}[2]{\showImage{\sectionBaseDir{3}/uml/#1.png}{#2}}
\newcommand{\showAlloyImage}[3]{\showPercentImage{\sectionBaseDir{3}/alloy/#1.png}{#2}{#3}}
\newcommand{\showRPMockupImage}[2]{\showImage{\sectionBaseDir{3}/mockups/png_passenger/#1.png}{#2}}
\newcommand{\showTDMockupImage}[2]{\showPercentImage{\sectionBaseDir{3}/mockups/png_driver/#1.png}{#2}{0.3}}

\title{Requirements Analysis and Specification Document for \myTaxiService{}}    % Declares the document's title.

\begin{document}	% End of preamble and beginning of text.
\maketitle
%\begin{abstract} % This is the start of the "abstract" environment.
	% The abstract environment is inteded to contain a summary of contents.
%	Requirements Analysis and Specification Document of \myTaxiService{} project.
%\end{abstract} % Here ends the "abstract" environment. 
\newpage % Starts a new page
\tableofcontents             % Prints the TOC
\inputSection{1}
\inputSection{2}
\inputSection{3}
\inputSection{4}
\end{document}               % End of document.