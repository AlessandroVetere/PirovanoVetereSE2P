\section{Specific requirements}
This section of the SRS should contain all of the software requirements to a level of detail sufficient to
enable designers to design a system to satisfy those requirements. These requirements must be: Correct, Unambiguous, Complete, Consistent, Ranked, Verifiable, Modifiable and Traceable.
\subsection{External interface requirements}
TODO
\subsubsection{User interfaces}
TODO
\subsubsection{Hardware interfaces}
TODO
\subsubsection{Software interfaces}
TODO
\subsubsection{Communications interfaces}
TODO
\subsection{System Features}
TODO: This template illustrates organizing the functional requirements for the product by system features, the major services provided by the product. You may prefer to organize this section by use case, mode of operation, user class, object class, functional hierarchy, or combinations of these, whatever makes the most logical sense for your product.
\subsubsection{System Feature 1}
TODO
\paragraph{Description and Priority}
TODO: Provide a short description of the feature and indicate whether it is of High, Medium, or Low priority. You could also include specific priority component ratings, such as benefit, penalty, cost, and risk (each rated on a relative scale from a low of 1 to a high of 9).
\paragraph{Stimulus/Response Sequences}
TODO: List the sequences of user actions and system responses that stimulate the behavior defined for this feature. These will correspond to the dialog elements associated with use cases.
\paragraph{Functional Requirements}
TODO: Itemize the detailed functional requirements associated with this feature. These are the software capabilities that must be present in order for the user to carry out the services provided by the feature, or to execute the use case. Include how the product should respond to anticipated error conditions or invalid inputs. Requirements should be concise, complete, unambiguous, verifiable, and necessary. Use “TBD” as a placeholder to indicate when necessary information is not yet available.
TODO: Each requirement should be uniquely identified with a sequence number or a meaningful tag of some kind.
\begin{itemize}
	\item REQ-1:  
	\item REQ-2:
\end{itemize}
\paragraph{Scenarios}
\paragraph{UML Diagrams}
\paragraph{Alloy Models}
\subsection{Performance requirements}
TODO
\subsection{Design constraints}
TODO
\subsection{Software system attributes}
TODO
\subsection{Other requirements}
TODO