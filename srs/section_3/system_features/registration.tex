\subsubsection{Registration}
This feature is related to goal \textbf{G1}, \textbf{G2} and \textbf{G12}, and to \textbf{Registration} use case.
\paragraph{Use Case}
\begin{itemize}
	\itemBold{Actors} Non registered passenger (referred as Bob for brevity).
	\itemBold{Pre condition} Bob is not registered to \myTaxiService{} and wants to do so.
	\itemBold{Post condition} Bob is now registered to \myTaxiService{} and therefore he is a member of registered passenger users class.
	\itemBold{Event Flow}
	\begin{itemize}
		\item Bob reaches either \myTaxiService{} web page or application, and selects the registration input control.
		\item Bob fills the registration page with his personal data, accepts the treatment of his personal data and selects the confirmation input control.
		\item The system acknowledges the registration of Bob to the service and sends him a confirmation email.
	\end{itemize}
	\itemBold{Exceptions}
	\begin{itemize}
		\item The browser or the application is closed at any time: the whole registration process is invalidated.
		\item The personal data inserted in the registration page are not valid: on confirmation the system asks Bob to correct the necessary informations.
		\item Bob is already registered: on confirmation the system neglects registration.
	\end{itemize}
\end{itemize}
\paragraph{Description and Priority}
This feature is of high priority, it is essential for the system to be.\par
\paragraph{Functional Requirements}
\begin{itemize}
	\itemBold{FR1} A non registered passenger can register only once.
	\itemBold{FR2} Personal data inserted during registration must be valid.
	\itemBold{FR3} In no case can a registration process be resumed after browser or application closure.
	\itemBold{FR4} The system must acknowledge the non registered passenger whether the registration process ends successfully or fails.
	\itemBold{FR5} The system must send an email to the non registered passenger that has completed the registration.
	\itemBold{FR6} After the registration is completed successfully, the non registered passenger involved in the process must be considered a registered passenger.
\end{itemize}
\paragraph{Scenarios}
\begin{itemize}
	\item Carl wants to use the new come \myTaxiService{} because his friends told him how useful it is when it comes to move in town via taxi.
	Therefore he opens the \myTaxiService{} application he just installed in his smartphone and presses the registration button.
	He fill all the required fields with his data, accepts \myTaxiService{} conditions and then a dialog confirms his registration to the service.
	An email also arrives at his address.
	Being now a registered user, he can start using \myTaxiService{} in town.
	\item Alice is a taxi driver and she got his taxi driver login credentials on the day she was hired, so she don't need to register to \myTaxiService{} at all.
\end{itemize}
\unbreakableBlock{\paragraph{UML Sequence Diagram}
\showUmlImage{RegistrationSequenceDiagram}{Registration UML Sequence Diagram}}