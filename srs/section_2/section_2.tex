\section{Overall Description}
The aim of \myTaxiService{} is to improve taxi service usage and management in a large city by simplifying the access of passengers to the service and optimizing the management of taxi queue.\par
Overall, \myTaxiService{} will lead to several benefits for taxi drivers, passengers and the government of the city.
\subsection{Product Perspective}
The software \myTaxiService{} is a completely new product, not based on previous ones.\par
However, it relies on location data received via Internet from each taxi driver smartphone application: all the involved cars already have a GPS Bluetooth antenna installed inside, that communicates their position to the driver's smartphone and then to the service.\par
Being a partially distributed application, \myTaxiService{} requires a fully operative Internet connection in order to work properly, both on server and client side: no service is intended to be provided offline.\par
This software provides two separate end user interfaces accessible via Internet, and a separate administator interface that is only accessible via a LAN network.\par
All the data generated by this software are stored in a database, accordingly to current normatives and laws about privacy and personal data management.\par
In addition, several APIs are provided in order to permit further improvements and expansions of the software: in this way additional services like taxi sharing could be built on the top of the existing ones.
\showGeneralDiagram{}
The high-level diagram proposed above is intended to be a general indication for each type of professionalism that will use this document.\par
Therefore it aims to provide a high-level overview of the architecture of the software they are going to develop.\par
In particular, in this architecture are involved four classes of users; Taxi drivers, Passengers, registered or not, external developers and server administrators.\par
The first three are allowed to use \myTaxiService{} through smartphone or personal computer, remotely connected to the Web Application Server using HTTPS Internet navigation protocol.\par
The Web Application Server, as well as providing the services requested by users and ensuring programmative APIs to external developers, in turn uses the API provided by Google Maps to translate the GPS data sent from Taxi drivers' smartphone in real addresses, in such a way to associate every taxi driver to an area of ​​the city and add it to the relative queue.\par
In addition, the Web Application Server has an administrator interface to manage failures and anomalies. To ensure the ACID properties, the database of \myTaxiService{} is provided of a database server with a relational DBMS, connected with an Ethernet LAN to the Application Server.
\subsection{Product Features}
This project provides interesting features for both passengers and taxi drivers.\par
Non registered passengers can only register to \myTaxiService{}, whereas registered passengers can request a taxi through the system, that answers informing the passenger about the code of the incoming taxi and the waiting time.\par
On the other hand, taxi drivers should inform the system about their availability and confirm that they are going to take care of a certain call.\par
The system then guarantees a fair management of taxi queues: when a request arrives from a certain zone, the system forwards it to the first taxi queueing in that zone.
If the taxi confirms, then the system will send a confirmation to the passenger.
If not, then the system will forward the request to the second in the queue and will, at the same time, move the first taxi in the last position in the queue.
\subsection{User Classes and Characteristics}
As already mentioned, this software is dedicated both to passengers and taxi drivers, but the main focus of the software is about passengers' satisfaction, being them end users of the software, whereas taxi drivers are only intermediate users, trained to work with the application provided on their smartphones.\par
Therefore the graphical interface provided to passengers, both on the web and mobile application, should be more attractive and well-finished, while the taxi drivers interface should more usable, and in general may be less attractive.\par
Passengers can request a taxi either through a web application or a mobile application.\par
Taxi drivers use a mobile application to inform the system about their availability and to confirm that they are going to take care of a certain call.\par
Under a certain point of view, we could consider even an additional user class which is the one of the developers that will be using the project API to develop further services based on the provided ones.
This last class needs an extensive documentation about the programmatic interfaces they will integrate in their development, but no proper graphical or textual user interface is provided for them, so no it will be no mentioned further in the document.
\subsection{Operating Environment}
This software will operate on different machines in different environments.\par
Its distributed part should be installed on smartphones and be always connected to the Internet.\par
Taxi drivers must have a Bluetooth capable smartphone because their part of the software requires an always active Bluetooth connection to run properly.\par
The server runs in a dedicated facility, within a controlled and protected environment, provided with a persistent Internet connection.
\subsection{Design and Implementation Constraints}
Being the software deployed in a complex and non controllable environment, it could encounter some issues.\par
The most common issue that could affect \myTaxiService{} is an Internet congestion: if the whole network slows down or become unavailable, the service could encounter major problems.\par
That issue could easily target passengers or taxi drivers when using their mobile application to communicate with the server.
For instance, they could be connected to a mobile Wi-Fi hotspot, to a WiMAX hotspot, or to a mobile 2G/3G/4G network cell, without any assurance of the bandwidth availability or network stability, thus leading to slow communication and possible timeouts.\par
A Bluetooth connection is required only on taxi drivers smartphones, in order for them to communicate with the GPS antenna installed in their cars, so taxi drivers smartphones cannot get more than 10 meters away from their cars while working, in order to mantain active the wireless link between them and the GPS antennas.\par
No external application is required by \myTaxiService{}, nor any interface with external applications is provided.\par
Object oriented, imperative and declarative programming languages will be used for \myTaxiService{} development, as well as markup and modelling languages.\par
\subsection{Assumptions and Dependencies}
The software is going to interact with computer browsers and smartphones, GPS Bluetooth antennas and external ISPs, and we assume that they meet their specification.