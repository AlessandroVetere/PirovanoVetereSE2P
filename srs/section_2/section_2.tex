%TODO Add /par where needed
\section{Overall Description}
The aim of \myTaxiService{} is to improve taxi service usage and management in a large city: this is achieved by simplifying the access of passengers to the service and optimizing the management of taxi queue.\par
Overall, \myTaxiService{} will lead to several benefits for taxi drivers, passengers and the government of the city.
\subsection{Product Perspective}
The software \myTaxiService{} is a completely new product, not based on previous ones.\par
However, it relies on location data received via internet from each taxi driver smartphone application: all the involved cars already have a GPS Bluetooth antenna installed inside, that communicates their position to the driver's smartphone and then to the service.
Being a partially distributed application, \myTaxiService{} requires a fully operative internet connection in order to work properly, both on server and client side.
For server side we intend all the hardware and software 
No service is intended to be provided offline.
This software has two separate end user interfaces accessible by the network, and a separate administator interface that is only accessible via a LAN network.\par
All the data generated by this software are stored in a database, accordingly to current normatives and laws about privacy and personal data management.\par
In addition, several programmatic API are provided in order to permit further improvements and expansions of the software: in this way additional services like taxi sharing could be built on the top of this API.
\subsection{Product Features}
This project provides interesting features for both passengers and taxi drivers.\par
Non registered passengers can only register to \myTaxiService{}, whereas registered passengers can request a taxi through the system, that answers informing the passenger about the code of the incoming taxi and the waiting time.
On the other hand, taxi drivers should inform the system about their availability and confirm that they are going to take care of a certain call.\par
The system then guarantees a fair management of taxi queues: when a request arrives from a certain zone, the system forwards it to the first taxi queueing in that zone.
If the taxi confirms, then the system will send a confirmation to the passenger.
If not, then the system will forward the request to the second in the queue and will, at the same time, move the first taxi in the last position in the queue.
\subsection{User Classes and Characteristics}
As already mentioned, this software is dedicated both to passengers and taxi drivers, but the main focus of the software is about passengers' satisfaction, being them end users of the software, whereas taxi drivers are only intermediate users, trained to work with the application provided on their smartphones.
Therefore the graphical interface provided to passengers, both on the web and mobile application, should be more attractive and well-finished, while the taxi drivers interface should more usable, and in general may be less attractive.
Passengers can request a taxi either through a web application or a mobile application.
Taxi drivers use a mobile application to inform the system about their availability and to confirm that they are going to take care of a certain call.
Under a certain point of view, we could consider even an additional user class which is the one of the developers that will be using the project API to develop further services based on the provided ones.
This last class needs an extensive documentation about the programmatic interfaces they will integrate in their development, but no proper graphical interface is provided for them, so no it will be no mentioned further in the document.
\subsection{Operating Environment}
This software will operate on different machines in different environments.\par
Its distributed part should be installed on Android and iOs smartphones and be always connected to the internet.
Taxi drivers must have a Bluetooth capable smartphone because their part of the software requires an always active Bluetooth connection to run properly.
The server runs on Linux in a controlled and protected environment, with a persistent internet connection.
\subsection{Design and Implementation Constraints}
Being the software deployed in a complex system, it could encounter some issues.
The major issue that could affect \myTaxiService{} is an internet congestion: if the network slows down or become unavailable, the whole service could encounter major problems.
For instance, that issue could easily target passengers or taxi drivers when using their mobile application outdoor to communicate with the main server.
They could be connected to a mobile WiFi hotspot, to a WiMAX hotspot, or to a mobile 2G/3G/4G network cell, without any assurance of the bandwidth availability or network stability, thus leading to slow communication and possible timeouts.
A Bluetooth connection is required only on taxi drivers smartphones, in order for them to communicate with the GPS antenna installed in their cars, so taxi drivers smartphones cannot get more than 15 meters away from their cars when working, in order to mantain active the wireless link between them and the GPS antennas.
No external application is required by \myTaxiService{}, nor any interface with external applications is provided.
%TODO Inserire i requisiti hardware e software
%TODO Mi sa che questa parte va nel design document, dato che noi lo facciamo esplicitamente separato.
Language used for development would be:
\begin{itemize}
	\itemBold{Java} For the server side of the software and the Android application development.
	\itemBold{Objective C} For the iOS application development.
	\itemBold{HTML, CSS and JSP} For the web server and web application development.
	\itemBold{MySQL} For the interaction with the DBMS.
\end{itemize}
Development tools would be:
\begin{itemize}
	\itemBold{Android Studio} For the server side of the software and the Android application development.
	%TODO iOS IDE
	\itemBold{Apple iOS IDE} For the iOS application development.
	\itemBold{Eclipse} For the web server and web application development.
\end{itemize}
Technologies used by \myTaxiService{} are:
%TODO Are we using a three tier or two tier architecture? -> This is design IMHO, and goes in DD
\begin{itemize}
	%TODO Which web server are we using?
	\itemBold{Webserver} For the webserver.
	\itemBold{DBMS} For the iOS application development.
	\itemBold{Linux} Server operative system.
\end{itemize}
%TODO Describe any items or issues that will limit the options available to the developers. These might include: corporate or regulatory policies; hardware limitations (timing requirements, memory requirements); interfaces to other applications; specific technologies, tools, and databases to be used; parallel operations; language requirements; communications protocols; security considerations; design conventions or programming standards (for example, if the customer’s organization will be responsible for maintaining the delivered software).
\subsection{Assumptions and Dependencies}
We are going to interact with Android smartphones, iOS smartphones, GPS Bluetooth antennas and external ISPs, and we are going to assume that they meet their specification.
In particular we are referring to Android and iOS SDKs.
%TODO Better specify which toolkits are being used, without getting too much into details.
%TODO List any assumed factors (as opposed to known facts) that could affect the requirements stated in the SRS. These could include third-party or commercial components that you plan to use, issues around the development or operating environment, or constraints. The project could be affected if these assumptions are incorrect, are not shared, or change. Also identify any dependencies the project has on external factors, such as software components that you intend to reuse from another project, unless they are already documented elsewhere (for example, in the vision and scope document or the project plan).