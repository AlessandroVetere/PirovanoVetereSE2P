\section{Overall Description}
The aim of \myTaxiService{} is to improve taxi service usage and management in a large city.
The project will simplify the access of passengers to the service and optimize the management of taxi queues.
This software will lead to several benefits for taxi drivers, passengers and the government of the city.
The software will simplify the access to the service therefore increasing the number of users, giving benefit to either the drivers and the government.
Also, it will optimize the management of taxi queues shortening the passengers’ waiting time and distributing equally users to taxi drivers.
\subsection{Product Perspective}
The \myTaxiService{} software is a completely new product, not based on previous ones.
However, it relies on GPS data received via internet from each taxi driver smartphone: all the cars already have a GPS Bluetooth antenna installed inside.
Being a distributed application, \myTaxiService{} requires a fully working internet connection in order to work properly.
The software has two separate end user interfaces, and a separate administator interface, not accessible by the network.
Data generated by this software are stored in a database.
In addition, several programmatic API are provided in order to permit further improvements and expansions. In this way services like taxi sharing could be built on the top of this API.
\subsection{Product Features}
\myTaxiService{} provides functions for both passengers and taxi drivers.
A passenger can make a taxi request to the system, that has to answers informing the passenger about the code of the incoming taxi and the waiting time.
On the other hand, taxi drivers should inform the system about their availability and confirm that they are going to take care of a certain call. The system guarantees a fair management of taxi queues. When a request arrives from a certain zone, the system forwards it to the first taxi queuing in that zone. If the taxi confirms, then the system will send a confirmation to the passenger. If not, then the system will forward the request to the second in the queue and will, at the same time, move the first taxi in the last position in the queue.
\subsection{User Classes and Characteristics}
This software is dedicated both to passengers and taxi drivers. The main focus is on the passenger, which is a final user of the software, whereas the taxi driver is only an intermediate user. Therefore passengers interface should be more attractive and well-finished, while the taxi drivers interface should more usable.
Passengers can request a taxi either through a web application or a mobile application.
Taxi drivers use a mobile application to inform the system about their availability and to confirm that they are going to take care of a certain call.
Under a certain point of view, we could consider even an additional user class which is the one of the developers that will be using the project API to develop further services based on the provided ones.
This last class needs an extensive documentation about the programmatic interfaces they will integrate in their development, but no proper user interface is provided for them.
\subsection{Operating Environment}
This software will operate on different machines in different environments. It's distributed part should be installed on Android and iOs smartphones and always connected to the internet. Taxi drivers must have a Bluetooth capable smartphone because their part of the software requires an always active Bluetooth connection too.
The backend runs on Linux in a controlled and protected environment, with a persistent internet connection.
\subsection{TODO: Design and Implementation Constraints}
The major issue that could affect \myTaxiService{} is an internet congestion. If the network slows down, the whole service could encounter major problems.
For instance, that issue could easily target the users or the taxi drivers when using their mobile application to communicate with the main server.
They could be connected to a mobile WiFi hotspot, to a WiMAX hotspot, or to a mobile 2G/3G/4G network, without any assurance of the bandwidth availability or network stability, thus leading to slow communication and possible timeouts.
A Bluetooth connection is required only on taxi drivers smartphones, in order for them to communicate with the GPS antenna installed in their cars.
No other applications are required for \myTaxiService{} to work, nor any interface with external applications is necessary.
Language used would be Java for the server part and the Android applications, and Objective C for iOs development; HTML, CSS, Javascript and PHP will be used along Java for the webserver development.

TODO: Describe any items or issues that will limit the options available to the developers. These might include: corporate or regulatory policies; hardware limitations (timing requirements, memory requirements); interfaces to other applications; specific technologies, tools, and databases to be used; parallel operations; language requirements; communications protocols; security considerations; design conventions or programming standards (for example, if the customer’s organization will be responsible for maintaining the delivered software).
\subsection{TODO: Assumptions and Dependencies}
TODO: List any assumed factors (as opposed to known facts) that could affect the requirements stated in the SRS. These could include third-party or commercial components that you plan to use, issues around the development or operating environment, or constraints. The project could be affected if these assumptions are incorrect, are not shared, or change. Also identify any dependencies the project has on external factors, such as software components that you intend to reuse from another project, unless they are already documented elsewhere (for example, in the vision and scope document or the project plan).